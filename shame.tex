\documentclass[10pt, fleqn]{article}
\usepackage{mydefaults}

\begin{document}


Стыд - одна из самых мощных разрушительных сил в человеческой жизни.
Сам по себе стыд не плох, даже наоборот, здоровый стыд нужен нам, что бы знать о наших ограничениях как человек. Стыд даёт нам разрешение быть человеком. Стыд даёт нам понять, что мы, люди, совершаем ошибки, что мы не боги. Здоровый стыд является фундаментом таких понятий как смирение и духовность.


Стыд, как здоровая эмоция, может быть трансформирована в стыд как состояние бытия. В таком виде стыд забирает на себя всю личность человека. Испытывать стыд как состояния жизни значит верить в то, что ты ущербен, что ты неполноценен как человек. Как только стыд становится идентификацией человека, стыд становится токсичным.

Токсичный стыд невыносим, поэтому всегда требует укрытия - ложное <<я>> (a false self). Раз человек чувствует свою ущербность, то он создаёт ложное я, которое не ущербно. Идентифицировать себя через ложное <<я>> значит перестать существовать как аутентичный человеческое существо. Ложное я считает пытается быть <<более чем>> или <<менее чем>> человек. Процесс формирования ложного <<я>> Alive Miller называет <<убийством души>>. 

Токсичный стыд настолько разрушает функции аутентичной личности, что синдромы стыда начинают развиваться из ложного <<я>>. Каждый синдром имеет свои характерные шаблоны. Токсичный стыд становиться корнем неврозов, расстройств личности, политического насилия, войн и криминала.

Библия описывает стыд как основу и последствия падения Адама. Адам, как символ всего человечества, не был удовлетворён своим состоянием бытия, он хотел стать больше чем человеком. Он отказался от принятия своих ограничений, таким образов потеряв свой здоровый стыд. Библия говорит, что первородный грех является желанием быть кем-то другим, кем мы являемся, быть большее, чем человеком. В своём токсичном стыде Адам желал обрести ложное <<я>>.
После того, как Адам отверг свою аутентичность, он спрятался. До того, как Адам отказался от своей аутентичности, он не стыдился своей наготы.
Нагота символизирует истинное аутентичное <<я>>.

Метафоричное описание Адама и Евы является описанием человеческого состояния. Безусловная любовь и принятие себя кажется самой сложной задачей человечества. Отказываясь от истинного <<я>>, мы пытаемся создать более совершенное <<я>> или сдаться и стать меньше чем человеком. Результатом является пожизненная секретность и сокрытие. Это является главной причиной нашего страдания.

Тотальная любовь к себе и принятие является единственной основой для счастья. Без этого мы обречены на нервное создание ложного <<я>>, жить которым требует огромной затрат энергии.

\ж{Множество лиц стыда}

Из-за своего до речевого происхождения, стыд очень сложно определить. Это здоровая человеческая эмоция, которая может стать реальной болезнью души.
Есть два вида стыда: питательный (?)(nourishing) стыд и токсичный стыд. Токсичный стыд проявляется как мучительное внутреннее переживание неожиданного раскрытия. Это разделяет нас с собой и с другими. Мы открещиваемся от себя и это требует создание укрытия. Токсичный стыд любит тьму и секретность.
Из-за секретности токсичного стыд, его бывает сложно обнаружить за множеством ложных лиц.


\ж{Стыд как разрешение быть человеком}

Здоровое чувство стыда даёт нам знать, что мы не всемогущи. Ограничения это наша естественная натура. Множество проблем проистекает из непонимание нашей ограниченности.
Как и все эмоции стыд призван мотивировать нас на удовлетворение наших нужд.
Одна из наших базовых нужд - структура. Мы обеспечиваем свою структуру через создание границ личности, в которой мы безопасно действуем. Структура даёт нашей жизни форму. Границы и форма дают нам безопасность и более эффективное использование энергии.
Без границ мы теряем ограничения и легко запутываемся. Мы идём то в одну, то в другую сторону, зря тратя энергию. Мы теряем свой путь.
Здоровый стыд держит нас приземлённым (grounded?).
Здоровый стыд даёт нам знать, что мы не боги, что нам нужна помощь, также зная свои ограничения мы можем эффективнее использовать энергию.

\ж{Стыд как стадия развития}

Согласно словам Erik Erikson, чувство стыда является частью второго этапа психологического развития. В первой стадии ребёнку нужно установить чувство базового доверия.
Мы должны знать в самом начале, что мы можем доверять миру. Мир впервые приходит к нам в виде первичных опекунов (чаще всего родители). Нам нужно знать, что мы можем рассчитывать на кого-то вокруг нас, кто будет помогать нам в предсказуемой форме. Если у нас были такие опекуны, которые были предсказуемы, трогали нас и зеркалили наше поведение, мы формируем базовое чувство доверия. Когда существует безопасность и доверие мы начинаем формировать межличностную связь, которая создаёт мост взаимности. Такой мост очень важен, т.к. позволяет сформировать чувство собственной значимости. Единственный способ ребёнку сформировать чувство себя - через взаимоотношения с другими.
В эту раннюю стадию развития, мы можем знать себя только через отражении глаз наших главных опекунов.

\ж{Межличностный мост}

Отношения между ребёнком и опекуном постепенно развивается из взаимного интереса, а также обмена доверием. Доверие формирует благодаря нашему ожиданию взаимности. С ростом доверия формируется эмоциональная связь. Эмоциональная связь позволяет ребёнку рисковать и исследовать мир. Главный опекун становится важным в том смысле, что его любовь, уважение и забота становятся очень важны для нас. Мы позволяем себе быть уязвимыми в том, что позволяем себе нуждаться в другом человеке.
После этапа формирования базового доверия, ребёнок начинает формировать стыд. Стыд может быть как здоровым, так и токсичным.

\ж{Формирование здорового стыда}

Примерно в возрасте 15 месяцев у ребёнка начинает развиваться мускулатура. Ему нужно исследовать, пробовать, сжимать и отпускать, а для этого ему придётся отделиться от главного опекуна.
Задача этого этапа развития является поиск баланса между автономностью и стыдом.
Этот промежуток называют <<ужасными двумя>>, т.к. в этот период ребёнок исследует: трогает, пробует на вкус и тестирует. Они хотят делать все по-своему, поэтому когда их прерывают, они злятся.

\ж{Нужды ребёнка}

Что нужно ребёнку больше всего, это твёрдый, но понимающий опекун, который удовлетворяет свои нужды через супруга. Такому опекуну нужно самому решить свои проблемы, также иметь чувство ответственности за себя. В таком случае, такой опекун может быть доступен ребёнку и удовлетворять его нужды.
Ребёнку нужная здоровая модель стыда и других эмоций. Ему нужно внимание и время опекуна, а также хорошие границы. Ребёнку нужно иметь опекуна, который может установить границы. Внешний контроль должен быть обнадёживающим. Ребёнку нужно знать, что межличностный мост не будет разрушен его новыми стремлениями делать что-то своим путём.
Если ребёнок защищён твёрдыми, но сострадательными границами, если он может исследовать, тестировать и злиться без разрушения межличностного моста, т.е. любви, тогда ребёнок развит здоровое чувство стыда.

\ж{Стыд как смущение и покраснение}

В неловком положении человека ловят врасплох, он раскрыт, когда он к этому не был готов.
В таких случаях мы испытываем чувство здорового стыда, что может выражаться в виде покраснения.
Покраснения является манифестацией ограничений человека. Мы понимаем, что сделали ошибку. Это напоминает нам о наших человеческих границах.

\ж{Стыд как стеснительность}

Стеснительность это естественные границы, которые охраняют нас от раскрытия или травм от незнакомцев. Это тоже здоровый вид стыда.
Стеснительность может стать серьёзной проблемой, когда её корни идут из токсического стыда.

\ж{Стыд как базовая потребность в сообществе}

Людям нужны другие люди. Это наша потребность.
В детстве нам нужны родители, что удовлетворить наши потребности. Наш здоровый стыд служит напоминанием того, что нам нужна помощь других для удовлетворения потребностей.

\ж{Стыд как источник творчества и обучения}

Один из главных блоков творческой силы является уверенность в том, что мы правы. Эта уверенность останавливает нас от поиска новой информации.
Здоровое чувство стыда никогда не позволяет нам верить, что мы все знаем.

\ж{Стыд как источник духовности}

Что такое духовность? Это жизнь в которой мы растём и раскрываемся, это любовь, правда, добродетель, красота, отдача и забота.
Духовность это цельность и завершённость. Это конечная потребность человека.
Здоровый стыд является фундаментом нашей духовности. Стыд сигнализирует о наших ограничениях и даёт нам знать, что мы не боги. Здоровый стыд направляет нас в сторону поиска смысла, показывает нам, что есть что-то большее чем мы.

\ж{Токсичный стыд}

Описание неврозов и расстройства характера Scott Peck:
Невротик возлагает на себя слишком много обязательств; человек с расстройством характера берет на себя недостаточно обязательств. Когда у невротика возникает конфликт с внешним миром, он предполагает, что это он виноват. Когда конфликт с миром у человека с расстройством характера, то он предполагает, что это мир виноват.

Большинство из нас имеют немного невроза и расстройств характера. Главная проблема в том, что бы уяснить наши обязательства. Для этого нужно иметь хорошие отношения с самим собой, чего у людей с токсичным стыдом совершенно нет. Даже наоборот, токсичный стыд приводит к враждебным отношениям с самим собой.

\ж{Невротический синдром стыда}

Токсичный стыд - это всеобъемлющее чувство о том, что я ущербен как человек. Этот стыд больше не является эмоцией, он стал идентификацией человека.
Человек с такой идентификацией будет охранять себя от раскрытия другим, но ещё важнее, он будет остерегаться открыть настоящего себя самому себе.
Токсичный стыд невыносим из-за боли, которая вызывается при раскрытии своих ошибок самому себе.
В токсичном стыде <<я>> становится объектом собственного презрения, объектом, которому нельзя доверять.
Токсичный стыд переживается как внутренняя борьба, болезнь души.
Присутствует стыд о стыде. Человек с готовностью признает вину, обиду или страх, но не стыд. Токсичный стыд характеризуется ощущением одиночества и изоляции в полном смысле. Человека постоянно преследует чувство отсутствия и пустоты.

\ж{Стыд как идентификация - интернализация стыда}

Любая человеческая эмоция может быть интернализирована. В таком случае, эмоция становится идентификацией: <<злой человек>>, <<грустный>> и т.д. Человек не просто испытывает эмоцию, эмоция стала частью характера.

Процесс интернализации в трёх этапах:
\begin{enumerate}
\item Идентификация с ненадёжными и, имеющими корни стыда, моделями.
\item Травма при отказе в удовлетворении потребностей и связывание чувств, нужд и желаний со стыдом.
\item Взаимосвязь импринтов в памяти, которые формируют коллаж стыда.
\end{enumerate}

Интернализация это постепенный процесс и происходит он в течении какого-то времени. Каждому человеку приходиться бороться с этапами этого процесса, но вся интернализация требует присутствия всех трёх этапов.

\begin{enumerate}
\item Идентификация с моделями (образами), имеющие корни стыда

Идентификация это нормальный человеческий процесс. Это даёт нам чувство безопасности и защиты. Также является базовой потребностью.
Первые люди с кем мы идентифицируемся это наши главные опекуны (родители).
Когда опекуны являются обладателями токсичного стыда, ребёнок также идентифицируется с ними. Это первая ступень интернализации стыда.

\item Отказ (Abandonment)

Стыд интернализируется при отказе. Отказ является точным определением того, как человек теряет свою аутентичную личность и перестаёт существовать психологически.
Отзеркаливание (Mirroring) является очень важным фактором в жизни ребёнка и совершает в основном главными опекунами. Отказ включает в себя потерю этой функции.
Родитель, который закрыт эмоционально, не может зеркалить и подтверждать эмоции ребёнка.
Без человека, который будет отражать наши эмоции, мы не можем узнать, кем являемся.
Кроме недостатка отзеркаливания, отказ также включает в себя:
Пренебрежение потребностью развития зависимости.
Оскорбление любого вида.
Enmeshment (wiki). (что-то вроде исполнения роли супруга ребёнком).

\item Связывание стыда с чувствами, потребностями и желаниями

Связывание стыда с чувствами, потребностями и желаниями является главным фактором трансформации здорового стыда в токсичный. Быть связанным стыдом значит каждый раз переживая какое-либо чувство, потребность или желание вы сразу испытываете стыд. Вся динамика человека повязана на чувствах, потребностях и желания. Когда они будут повязаны стыдом, то он становится вашим ядром личности.

\item Взаимосвязь импринтов в памяти, которые формируют коллаж стыда

Стыдливые ситуации накапливаются в вашей памяти. В памяти сохраняются картины и звуки событий, формируя коллаж. Из-за того, что ребёнок не имеет поддержки или времени на горевание по-поводу боли утраты взаимопонимания, его эмоции будут подавлены.
Дети также запоминают поведение родителей в их худшие моменты. Когда они (родители) более всего теряют контроль, они сильнее всего угрожают выживанию ребёнка. Тревожные сигналы ребёнка очень глубоко регистрируют это поведение. Любое последующее стыдное переживание, которое даже отдалённо напоминает эти старые травмы, легко цепляется словами или картинами из этой травмы. Далее эти старые и новые воспоминания записываются в одну конструкцию, постепенно включая в себя все больше воспоминаний.
Со временем, для запуска чувства стыда нужно совершенно не много: слово, выражение лица или сцена. Иногда, внешней стимуляции и не нужно, достаточно просто вспомнить что-то из прошлого и это воспоминание запустит болезненную эмоцию.
Стыд как эмоция застыла и встроилась в ядро личности. Стыд глубоко интернализирован.
\end{enumerate}



\ж{Стыд как самоотчуждения и изоляции}

Самоотчуждение это ощущение чужеродности части себя. Например, если вам не разрешалось испытывать гнев, то ваш гнев станет чужеродной частью вас. Вы будете испытывать токсический стыд, когда будете злиться. Гнев был отвергнут вами, но психологически невозможно отказаться от энергии гнева, т.к. это энергия самосохранения и самозащиты. Без этой энергии вы станете тряпкой и люди будут использовать вас.
Когда стыд интернализирован, вы прекращаете быть собой. Вы постоянно критикуете своё поведение, что является мучительным. Парализирующее постоянное наблюдение за собой является причиной пассивности и бездействия.
<<Отрубленные>> части <<я>> проецируются во взаимоотношениях. Чаще всего это является фундаментом для ненависти и осуждения.
Подобная отчуждённость часто создаёт ощущение нереальности.

\ж{Стыд как ложное <<я>>}

Раскрытие <<я>> самому себе лежит в основе невротического стыда, что требует создание ложного <<я>>. Ложное <<я>> всегда более чем или менее чем человек: перфекционист или тупица, герой семьи или козёл отпущения. Когда ложное <<я>> было создано, аутентичное <<я>> уходит в укрытие.
Парадоксально, что невротический стыд является корнем как супер-достигатора, так и лентяя, могущественного и жалкого - двух противоположностей.

\ж{Стыд как со-зависимость}

Со-зависимость это состояние потери самости. Счастье снаружи, а не внутри и не могут быть созданы изнутри.

\ж{Стыд как пограничная личность}

Kaufman считает, что токсичный стыд является если и не источником, то как минимум одной из причин таких синдромов как зависимости, клиническая депрессия, шизоидные явления и пограничные личности.

\ж{Стыд как ядро и топливо всех зависимостей}

Невротический стыд является корнем всех компульсивных/аддиктивных поведений.
Одиночество и обида, усиленная стыдом, избегается через взаимодействие с аддиктивным веществом или же поведением. Это поведение создаёт ещё больше стыда, что замыкает полный цикл.

\ж{Стыд и вина}

Нужно чётко понимать разницу между стыдом и виной (вина также может быть здоровой и токсичной). Вина это эмоция, которая является результатом поведения, противоречащего нажим убеждениям и ценностям. Вина предполагает усвоенные правила и формируется позже стыда. По словам Эриксона, в третье стадии психологического развития достигается баланс между инициативой и виной. Эта стадия начинается в возрасте 3 лет.
Вина не влияет на прямую на идентификацию человека и не уменьшает чувство собственного достоинства.
<<Я чувствую вину за последствия своих действий>> - человек подтверждает свои ценности и возможность исправления ошибок, обучения и роста. Вина является болезненной реакцией сожаления и ответственности за свои действия, но стыд - это болезненное ощущение о самом себе как личности. Возможность исправления видится невозможной для стыдливого человека, т.к. стыд является делом идентичности, не поведения. Стыд только подтверждает негативные мысли о себе.
Martin Buber говорит, что то, что лечит в любой модели терапии это взаимоотношение <<Я и Ты>>. Как только межличностный мост будет установлен, пациент сможет принять <<принятие без осуждения>> терапевта.
Я рекомендую всем людям, с токсичным стыдом, идти в группы. Группы обеспечивают чувство значимости и важности, чего не может представить терапия один-на-один.
Cermak: групповая терапия хороша тем, что люди в ней ведут себя обычном образом, т.е. критикуют, не доверяют другим, контролируют и т.д. Это позволяет прийти к пониманию, что подобное поведение отражает их подсознательное желание защиты от токсичного стыда, что позволяет группе стать лабораторией для испытания альтернативного поведения.

\ж{Расстройства характера на основе стыда}

\begin{enumerate}
\item Нарциссическое расстройство характера

По словам James Masterson, главными клиническими характеристиками нарциссического расстройства личности являются: грандиозность, отсутствие интереса и эмпатии к другим, несмотря на стремление получить их восхищение и одобрение.
Нарцисс бесконечно мотивирован достигать совершенства во всем. Такая личность стремиться к приобретению богатства, власти и красоты, а также ищет других, кто будет зеркалить и восхищаться его грандиозностью.
Под всем этим внешним фасадом находиться пустота, наполненная завистью и гневом. Ядро этой пустоты это интернализированный стыд.

\item Параноидальная личность

Параноидальная защита развилась как способ справиться с чрезмерным стыдом. Такой человек становится сверхбдительным к предательству и унижению, которое, как он знает, обязательно придёт. Параноидальная личность интерпретирует невинные события как угрозу для себя, поэтому живёт на стороже.
Harry Stack Sullivan называет ощущения параноика к себе как <<чувство безнадёжной неполноценности>>. Источник недостатков и ошибок параноик видит в других, но не в себе.

\item Преступное поведение

Криминальное поведение это чаще всего отыгрывание или реконструкция событий. Это значит, что человек, совершающий криминальное действие, был ранее жертвой подобного поведения. Дети из неблагополучных семей страдают виктимизацией и чаще всего далее берут на себя роль жертвы или преступника и повторяют те события. Эта реконструкция называется <<навязчивым повторением>>.

\item Физическое насилие

Преступники, совершающие физическое насилие, ранее были жертвами подобного насилия и чувствовали себя беспомощными и униженными. Родители, которые в детстве были жертвами насилия, далее в жизни реконструируют эти же события, но теперь уже со своими детьми, т.к. импринты в их памяти запускают реакцию в ответ на похожее поведение их детей. Получает цикл насилия.
Почему родители хотят играть роль своих родителей, которые их били и унижали? Ответ в динамике идентификации. Когда дети страдают от физической и психологической боли, они хотят выйти из этой ситуации как можно быстрее. Поэтому они перестают идентифицироваться с самим собой и берут на себя роль своего угнетателя в попытке завладеть его властью и силой. Формируя идентификацию родителя, человек становится одновременно слабым плохим ребёнком и сильным преступным родителем. Внутренний образ жестокого родителя запускает старую сцену и управляет всем процессом. Физическое насилие действует как триггер для навязчивого повторения насилия в сторону себя, супруга или детей. Интернализированый стыд поддерживает этот процесс.
Жертва насилия также может остаться жертвой. Martin Seligman провёл множество исследования так называемой <<наученной беспомощности>>. Суть в том, что случайные и непредсказуемые акты насилия создают состояние пассивности в которой жертва больше не верит, что она может что-то сделать и что у неё есть выбор.
Простое объяснение привязанности к насилию в том, что человека избивают все больше и больше, что приводит к большему стыду. Больший интернализированный стыд приводит к твёрдым убеждениям в своей неполноценности и дефектности. Чем крепче убеждённость в этом, тем меньше выбора. Интернализированый стыд разрушает границы личности, а без них у него нет никакой защиты.

\item Сексуальное насилие

Сексуальные насильники чаще всего секс-зависимые. Часто они реконструируют сцены собственного сексуального насилия. Сексуальное насилие генерирует интенсивное и парализующее чувство стыда, что чаще всего приводит к расщеплению личности.
Сексуальный насильник трансформирует свои стыд, унижение и беспомощность во время сексуального насилия. Это временно освобождает от стыда.
\end{enumerate}

\ж{Грандиозность}
Токсичный стыд также надевает маску грандиозности. Грандиозность это расстройство воли. Может проявляться как нарциссическое увеличение своего образа или как червеобразная беспомощность. Каждая крайность отказывается от роли человека. Каждый преувеличивает: один выше чем человек, другой хуже чем человек. Надо понимать, что беспомощный также выражает грандиозность, т.к. считает, что ничто и никто не может ему помочь.
Грандиозность это результат отключения воли человека. Воля отключается в основном через посрамление эмоций. Пристыженные и заблокированные эмоции останавливают полную интеграцию интеллектуального смысла. Когда случается эмоциональное событие, эмоция должна быть выпущена для того, что бы интеллект, суждение и разум смогли понять и сделать какие-то выводы. Эмоции регулируют мысли. Когда эмоции связаны стыдом, их энергия замораживается, что останавливает взаимодействие между разумом и волей.
Воля это аппетит к действию. Без разума, воля слепа и не имеет контекста.

Проблемы:
\begin{itemize}
\item Воля хочет то, что невозможно.
\item Воля пытается контролировать все.
\item Воля ощущает себя всемогущим или черве подобным.
\item Воля хочет крайности - все или ничего
\end{itemize}

\ж{Токсичный стыд как духовное банкротство}
Проблема токсичного стыда в конечном счёте является проблемой духовной. Я думаю, что мы не являемся материальными существами на духовном пути, мы духовные существа, которым нужен земной опыт для достижения полной духовности.
Духовный образ жизни это тот, который улучшает и расширяет жизнь, т.е. духовность заключается в росте, улучшении и расширении жизни. Духовность это бытие. Бытие в упорной борьбе побеждает небытие. Бытие это о том, почему существует вообще что-то. Бытие является основой всех существ.

\ж{Дегуманизация}

Токсичный стыд, который отчуждает себя от себя, дегуманизирует человека. Испанский философ Ortega Y. Gasset говорил, что человек это единственное существо, которое живёт изнутри наружу. Быть человеком, значит иметь внутренний мир и жизнь, исходящую из него. Когда у человека нет внутренней жизни, он дегуманизируется.
Токсичный стыд с его крайностями - более чем человек, менее чем человек - дегуманизирует. Требование ложного <<я>> для сокрытия аутентичного <<я>>, требует действий и достижений. Все силы тратятся на производительность и достижения, а не на бытие. Бытие не требует измерений, оно самом по себе имеет ценность. Бытие берет начало во внутреннем миру, который всегда в изобилии.
Токсичный стыд заставляет смотреть наружу для достижения счастья и одобрения, из-за того, что внутренний мир дефективен и неполноценен. Токсичный стыд это духовное банкротство.

\ж{Токсичный стыд как безнадёжность}

Токсичный стыд имеет такое качество как неисправимость. Если я неполноценен, дефективен и ущербен, значит я ничего не могу сделать на счёт этого. Такое убеждение ведёт к бессилию.
Токсичный стыд также вызывает цикличность, т.е. стыд порождает стыд.

\ж{Функциональная автономия}

Когда токсичный стыд интернализирован, он становится автономным, что значит, он может быть запущен внутренне, без всяких стимулов. Достаточно просто представить ситуацию и почувствовать глубокий стыд. Человек может быть один, но при этом вызвать спираль стыда через внутренний диалог.
Именно это качество токсичного стыда делает его настолько разрушительным.
Вы могли уже заметить, на сколько драматичным может быть динамика стыда. Осознавая это и зная о различных масках стыда, мы можем получить какой-то контроль над ним.

\ж{Источник токсичного стыда}

Токсичный стыд в основном формируется в близких взаимоотношениях. Если вы не цените кого-либо, трудно испытывать стыд из-за его слов или действий в вашу сторону.
Если наши главные опекуны (родители) имеют корни токсичного стыда, они будут действовать соответственно этому и передадут токсичный стыд нам. Не существует способа привить здоровую самооценку, если у самого человека её нет.

Токсичный стыд тянется из поколений в поколение. Стыдливые люди находят других стыдливых людей и женятся. Как пара каждый из них несёт в себе токсичный стыд от своих семей. Основным результатом этого будет недостаток интимности, т.к. сложно подпустить к настоящему себе другого человека, если считаешь, что настоящий <<я>> дефективен и неполноценен. Отсутствие интимности поддерживается плохой общительностью, постоянными ссорами, играми, манипуляциями, обвинениями, соперничеством и т.д.

Когда у таких родителей рождается ребёнок, шансом на здоровую психику у него нет. Работа родителей в том, что бы подавать пример, как быть мужчиной или женщиной; как относиться к близким людям; как воспринимать и выражать эмоции; как честно бороться; как строить физические, эмоциональные и интеллектуальные границы; как общаться; как справляться с жизненными проблемами; как быть дисциплинированным; как любить себя и других. Родители с токсичным стыдом ничего из этого делать не умеют. Они просто не знают, т.к. их тоже этому не научили.

Детям нужно время и внимание, что является проявлением любви. Это значит быть рядом с ребёнком и уделять внимание его потребностям. Также ребёнку нужно, что бы его слушали. Дети прямо заявляют о своих потребностях. Что бы родители могли их слушать, они сами должны быть эмоционально взрослыми и уметь удовлетворять свои потребности.

Нуждающийся, стыдливый родитель в принципе не может позаботиться о потребностях ребёнка. ребёнка стыдят каждый раз, когда он проявляет нужду, т.к. нужда ребёнка сталкивается с нуждой родителя. ребёнок вырастает и становится взрослым, но под маской взрослого поведения находиться ребёнок, который был отвергнут. Этот нуждающийся ребёнок ненасытен. У такого взрослого находится, так называемая, дыра в душе. Он не может насытиться как взрослый, т.к. на самом деле неудовлетворёнными остаются потребности ребёнка.

Подобные семьи действуют по законам социальных систем. Когда социальная система дисфункциональна, она жестка и закрыта. Все члены этой семьи будто находятся в трансе. Они заботятся о нуждах системы для удержания баланса.

\ж{Дисфункциональные семьи}

\ж{Семья как социальная система}

Семья, как социальная система, следует определённым организмическим законам. Один из законов заключается в том, что целое важнее частей. На примере организма человека это объясняется так: части тела по отдельности не составляют живой организм; организм это взаимодействие всех этих частей.

Вся семья важнее её отдельных членов. Семья определяется взаимодействием её членов, нежели просто группой людей. Как социальная система семья имеет свои компоненты, роли, правила и нужды.

Главный компонент семьи как системы это брак. Если брак здоровый, значит семья тоже здоровая. Если нездоровый - вся система дисфункциональна. Если система семьи нездорова, то в действие вступает другой закон - закон динамического гомеостаза. Это закон, отвечающий за баланс.

Это закон значит, что если какая-либо часть системы выходит из строя, то остальные части будут пытаться вернуть всю систему в состояние баланса.

Дети в дисфункциональных семья берут на себя жёсткие роли, необходимые для баланса всей системы. Например, если ребёнок ощущает свою ненужность, он попытается сбалансировать семью став супер-ответственным, идеальным, избегающим проблем. Эту роль я называю <<Потерянный ребёнок>>.
<<Маленький отец>> - роль в семье, которую бросил отец. <<Под защитой>> - роль ребёнка, которого защищают от насилия со стороны главных опекунов.
Надо понимать, что все подобные роли являются прикрытием для токсического стыда.

В здоровых семьях роли выбираются и являются гибкими. Члены семьи могут отказаться от любой роли. В дисфункциональных семьях роли жёстко закреплены. Каждый играет свою жёсткую роль, при этом вся система остаётся в неподвижности. Дисфункциональные семьи находятся в состоянии некого транса. Ядро токсического стыда держит систему в заморозке. Все члены прячут свои аутентичные личности за масками ролей.

\ж{Стыд, передающийся через поколения}

Одним из самых разрушительных аспектов токсического стыда является его передача через поколения. Из-за природы секретности токсического стыда, проблема не разрешается. Секреты это то, чего семьи стыдятся. Это может быть суициды, убийства, инцесты, аборты, зависимости, финансовые потери и т.д. Все эти секреты будут проявляться через токсический стыд.

Генерируемые стыдом боль и страдания создают автоматические и бессознательные защитные механизмы. Фрейд называл эти защиты разными именами - отрицание, идеализация родителей, подавление эмоций и диссоциация от эмоций. Главное то, что мы не можем знать того, чего не знаем. Из-за того, что мы не осознаем этим механизмы защиты, мы теряем связь со стыдом, обидой и болью, которые они защищают. Мы не можем вылечить то, чего не чувствуем. Без восстановления этих чувств, токсический стыд продолжает тянуться через поколения.

\ж{Родительский образ в семьях на основе стыда}

Как вы видите, основным источником токсического стыда являются семейные системы и передача через поколения неразрешённых секретов. Как же заключаются подобные браки? Люди, имеющие корни своей личности в токсическом стыде, сами выбирают друг друга. Каждый ищет и надеется, что другой человек позаботиться о его внутреннем ребёнке. Каждый из них ненасытен и неполноценен. Их ненасытность произрастает из неудовлетворённых детских потребностей. Когда два взрослых ребёнка встречаются и влюбляются, каждый ожидает от другого удовлетворения его потребностей. Химия любви заставляет их почувствовать свою полноценность и единство, но как только чувства проходят, наружу выходят защитные механизмы и начинается война за удовлетворение тех детских потребностей.

\ж{Правила в семьях на основе стыда}

Каждая семейная система имеет свои правила: о празднованиях и социализации; о прикосновениях и сексуальности; о болезнях и об уходе за здоровьем; о содержании дома и трате денег. Одними из главных правил являются правила о чувствах, межличностных коммуникациях и воспитании.

Токсический стыд сознательно передаётся через правила осуждения и пристыжения.

\ж{Правила дисфункциональных семей}
\begin{enumerate}
\item Контроль - человек должен всегда контролировать все взаимодействия, чувства и собственное поведение.
\item Перфекционизм - всегда будь правильным во всем, что ты делаешь. Это правило включает в себя наложенную шкалу измерения. Все члены семьи живут согласно идеальному образу, которого невозможно достигнуть.
\item Вина - когда дела идут плохо, обвиняй себя или кого-то другого. Обвинения также являются прикрытием для стыда. Обвинения поддерживают баланс в дисфункциональных семьях, когда был нарушен контроль.
\item Отрицание пяти свобод - пять свобод, впервые определены Вирджинией Сатир, описывают полную личную функциональность. Каждая свобода имеет отношение к базовым человеческим силам: сила воспринимать; думать и интерпретировать; чувствовать; желать и выбирать; и сила воображения. В дисфункциональных семьях правило перфекционизма запрещает полное выражение этих сил. Оно говорит, что ты не можешь воспринимать, думать, чувствовать, желать и воображать как ты хочешь. Ты должен делать это так, как требует наши идеалы.
\item Молчание - это правило запрещает полное выражение любых чувств, потребностей и желаний. В семьях, основанных на стыде, члены семье скрывают свои настоящие чувства, потребности и желания.
\item Не делай ошибок - ошибки раскрывают дефектную уязвимую личность. Признать ошибку, значит открыться для осуждения. Прикрывай все свои ошибки, а если кто-то ошибётся, пристыди его.
\item Ненадёжность - не ожидай надёжности в отношениях. Не верь никому, и ты не разочаруешься. Родители не удовлетворили свою потребность зависимости, поэтому их дети также не удовлетворят её. Цикл недоверия продолжится.
\end{enumerate}

\ж{Правила <<Ядовитой педагогики>> Alice Miller}
\begin{enumerate}
\item Взрослые являются хозяевами зависимого от них ребёнка.
\item Они определяют что правильно, а что нет.
\item ребёнок ответственен за гнев родителей.
\item Родитель всегда должен быть защищён.
\item Жизнеутверждающие чувства ребёнка несут угрозу для для самодержавного взрослого.
\item Воля ребёнка должна быть сломлена как можно скорее.
\item Все это должно произойти в очень раннем возрасте, что бы ребёнок этого не заметил и не раскрыл взрослых.
\end{enumerate}

Ядовитая педагогика оправдывает такие методы подавления ребёнка как: физическое насилие, вранье, двуличие, манипуляции, угрозы, отказ от любви, изоляция и принуждение до уровня пыток. Все это приводит к токсическому стыду.

\ж{Травма отказа (заброшенности)}
Что входит в понятие заброшенности и отказа (abandonment): физическое отсутствие родителей, недостаток поглаживаний, нарциссическая депривация, пренебрежение потребностью в формировании связи (зависимости), исполнение роли супруга (enmeshment), также все виды насилия.

\ж{Физическое отсутствие}

Ребёнку нужна предсказуемость и структура. Ему важно рассчитывать на присутствие кого-либо.
Ребёнку нужны оба родителя. Что бы разорвать связь с матерью, ему нужен отец, для формирования связи с ним. Формирование связи включает в себя проведения вместе времени, обмен чувствами, поглаживания, тепло, демонстрирование желание быть друг с другом.
До возраста 7-8 лет дети полностью эгоцентричны, это значит, что все что с ними происходит они воспринимает на свой счёт. Даже смерть родителя воспринимается как отказ родителя любят ребёнка.
Если ребёнок не проводит достаточно времени с родителями, то он считает, что дело в нем, что это его вина, иначе, родители бы проводили с ним больше времени.
Дети эгоцентричны, потому что у них не было достаточно времени для формирования эго границ. Эго границы это внутренняя сила, которая охраняет внутреннее пространство. Без границ, человек не имеет защиты. Крепкие границы эго это как дверь с защёлкой внутри дома. Слабые - как дверь с защёлкой снаружи дома. Эго ребёнка это дом без дверей.
Надёжные границы эго формируются под воздействием идентификации с родителями, которые сами должны иметь надёжные границы эго. Если же у самих родителей эго недоразвито, то и ребёнок смоделирует подобное эго.

\ж{Эмоциональная заброшенность и нарциссическая депривация}

Детям нужно зеркалирование. Зеркалирование значит, что кто-то всегда находится рядом с ними и отражает то, кем они действительно являются. В первые три года жизни, каждому из нас нужно, что бы нами восхищались и принимали всерьёз. Нужно, что бы нас принимали до конца. Получения этих отражений Alice Miller называет базовым нарциссическим набором.

Подобный набор является результатом правильного зеркалирования родителями с надёжными эго границами. Когда это так, происходит следующее:
\begin{enumerate}
\item Агрессивные импульсы ребёнка нейтрализуются, т.к. не они не являются угрозой родителям.
\item Стремление ребёнка к самостоятельности не воспринимается как угроза родителям.
\item Ребёнку позволяется переживать и выражать такие импульсы как ревность, ярость, сексуальность, неповиновение, т.к. сами родители не отказались от этих эмоций.
\item Ребёнку не нужно угождать родителям и он может развивать свои потребности в своём темпе.
\item ребёнок может зависеть и использовать родителей, т.к. они отделены от него.
\item Самостоятельность родителей и хорошие эго границы позволяют ребёнку отделить себя и объект представления.
\item Из-за того, что ребёнку позволяется выражать двойственные чувства, он учиться воспринимать себя и родителей как <<хорошими и плохими>>, нежели отделять определённые части как хорошие, отделяя их от плохих.
\item Возникновение истиной объективной любви становится возможным, т.к. родители любят ребёнка как отдельный от них объект. 
\end{enumerate}

Что случается, когда родители стыдливые и нуждающиеся? Получается так, что такие родители неспособны передать этот отражённый нарциссический набор ребёнку, т.к. сами родители в своё время не получили его. Такие родители это взрослые дети, которые ищут <<родителя>> или объект, который будет полностью в их распоряжении. Для таких родителей, чаще всего таким объектом становятся собственные дети.

То, что стыдливая мать не смогла получить от своей матери, она находит в своём ребёнке. Ребёнок всегда будет в её распоряжении, он не может убежать от неё, полностью центрирован на ней, находиться под её контролем, и полностью предлагает своей восхищение и внимание.

Дети отлично улавливают этот момент, т.к. беря на себя роль поставщика нарциссического поощрения для своих стыдливых родителей, они закрепляют их любовь и защищают себя от заброшенности. Процесс переворачивается вверх ногами и теперь ребёнок обеспечивает потребности родителей. Подобная смена ролей парадоксальна, т.к. закрепляя любовь родителей и избегая заброшенности, ребёнок фактически оказывается заброшенным. Теперь некому зеркалировать чувства ребёнка и удовлетворять его потребности. Любой ребёнок, выросший в подобных условиях, был смертельно ранен нарциссической депривацией.

Чаще всего такие дети талантливы, успешны, ими восхищаются и хвалят за их таланты и достижения. Любой кто посмотрит на них подумает, что эти люди счастливы. Они выглядят сильными и уверенными в себе. Но правда в том, что все эти восхищения и похвалы ни чем не помогают их внутренней пустоте.
Как только грандиозность пройдёт, они тонут в глубоком чувстве вины и стыда.
Детьми их хвалили за достижения и производительность, а не за их аутентичную личность. Истинная личность была заброшена.

Ещё одним следствием эмоциональной заброшенности является потеря чувства собственного <<я>>. Когда ребёнка использовали как поставщика нарциссического одобрения для других, он формируется так, что выражает только то, что от него ждут и в конечном итоге сливается со своей деятельностью и теряет связь со своим истинным <<я>>.

Возможно самым разрушительным последствием эмоциональной заброшенности является Фантазия Связи. ребёнок, которому было отказано в переживании связи со своими эмоциями, вначале сознательно, а затем подсознательно (через идентификацию с родителями) становится зависимым от своих родителей.

Alice Miller писала:
<<Он не может полагаться на свои эмоции, у него нет опыта переживания их через метод проб и ошибок, у него нет чувства его истинных потребностей и он отчуждён от самого себя.>>

Такой человек не может отделиться от своих родителей. У него есть фантазия связи с ними. Он находиться в иллюзии того, что между ними присутствует взаимоотношения, основанные на любви. На самом деле он слился с ними. Далее в жизни эта фантазия будет перенесена на другие отношения.

Человек с фантазией связи находится в зависимости от подтверждений супруга, его детей, его друзей. Такой человек никогда не имел настоящей связи или настоящих взаимоотношений с кем-либо. В нем нет реальной личности к которой можно относится. Реальная личность прячется от <<голосов>> родителей, которые принимали его только за его <<правильное>> поведение и достижения. Одиночество родительского дома заменяется изоляцией внутри себя.

Грандиозность чаще всего является последствием этого. Грандиозным человеком восхищаются и он не может без этого жить. Если его таланты пропадут, это будет катастрофа. Он должен быть идеальным, иначе придёт депрессия. Большинство самых талантливых людей страдают депрессией и это не удивительно, т.к. депрессия это признак потерянного и забытого ребёнка внутри.

Человек освобождается от депрессии, когда его самооценка базируется на аутентичных чувствах, а не на обладании определёнными качествами.

Эмоциональная заброшенность также передаётся через поколения. ребёнок, нарциссически депривированных родителей, вырастая также будет использовать своих детей для своего нарциссического удовлетворения. Цикл продолжится.
Когда эмоционально заброшенные дети описывают своё детство, оно всегда выходит практически без чувств. Они вспоминают своё раннее детство без какой-либо симпатии к тому ребёнку, какими они когда-то были. Часто они могут выражать презрение или иронию, даже цинизм. В общем, у них нет полного представления о реальных эмоциях, серьёзного понимания превратностей детства и никакой концепции их истинных потребностей, кроме потребности в достижениях.

\ж{Отказ через насилие}

Любые формы насилия это форма отказа. Когда совершается насилие над детьми, нет никого, кто мог бы помочь им. Предполагается, что подобное поведение совершается ради ребёнка, но на самом деле, все это проекция проблем взрослых. Подобная сумасшедшая трансформация взывает стыд. Каждый акт насилия вызывает в ребёнке стыд, из-за их эгоцентричности, они видят себя виноватыми в этом насилии.
Из-за полной зависимости от родителей в вопросе выживания, дети идеализируют их. Идеализация гарантирует выживание, т.к. если родители сумасшедшие, то ребёнок не выживет, значит они не сумасшедшие, а виноват я.
У ребёнка нет шансов против формирования токсического стыда в таких условиях.

\ж{Сексуальное насилие (злоупотребление)}

Сексуальное насилие сильнее всего вызывает стыд. В это понятие входит не только прямое сексуальное насилие с применением рук.

Формы сексуального злоупотребления:

\begin{enumerate}
\item Физическое сексуальное насилие - включает в себя прикосновения руками в сексуальной форме. Это может быть сексуальное объятие или поцелуй, любая форма сексуальных прикосновений или ласк, оральный или анальный секс, мастурбация жертве или принуждение к мастурбации насильнику, сексуальный акт.

\item Открытое сексуально злоупотребление - включает в себя вуайеризм и эксгибиоционизм. Критерием подобного злоупотребления считается наличие возбуждённости родителя. ребёнок может переживать сексуальное возбуждение в присутствии родителей и это не будет сексуальным злоупотреблением, если родитель не был инициатором этого. Все зависит от родителей и их сознательного или бессознательного использования детей для своей сексуальной стимуляции.

\item Прикрытое сексуальное злоупотребление
\begin{enumerate}
\item Словесное - включает в себя неприличные сексуальные разговоры: обзывание мужской фигурой женщин <<шлюхами>> или другими именами; обзывание женской фигурой мужчин неприличными именами. Также это включает попытки родителей узнать все о сексуальной жизни ребёнка.
Прикрытым сексуальным злоупотреблением также считается не получение адекватной информации о сексуальности. Примером может служить женщина, которая не знала что с ней происходит, когда у неё началась менструация.
Прикрытым злоупотреблением будет разговоры родителей о сексе перед детьми не подходящего для этого возраста. Сюда же входят сексуальные замечание на счёт частей тела ребёнка. Примеры: шутки матери о размере члена или шутки отцов о размере груди своих дочерей.

\item Нарушение границ - включает в себя свидетельство ребёнком сексуального поведения родителей. Также запрет на приватность самого ребёнка. Им не разрешают закрывать дверь (в туалете, ванне). Родители должны моделировать соответствующее представление о наготе, например, должны быть адекватно одеты после определённого возраста. В возрасте около 3-6 лет дети становятся одержимыми наготой и начинают замечать тела родителей.
Разгуливание голышом перед детьми без сексуальной стимуляции не является злоупотреблением. Однако это дисфункциональное поведение, т.к. оно не выстраивает адекватных сексуальных границ.


\end{enumerate}

\item Эмоциональное сексуальное насилие - является результатом скрытого принятия ребёнком роли супруга (enmeshment). В таком случае родитель использует ребёнка для удовлетворения своих нужд. Дочь может стать Отцовской Маленькой Принцессой, а сын - Маминым Маленьким Мужчиной. В обоих случаях потребности ребёнка забрасываются. Ребёнку нужен родитель, а не супруг.
Pia Mellody даёт такое объяснение этому феномену: <<когда отношения родителя с ребёнком для него важнее отношений с супругом, это и будет эмоциональным сексуальным злоупотреблением.>>
Подобная замена ролей может происходить и между родителем и ребёнком одного пола. Самым частым случаем является отношения мамы и дочки. Мамы часто испытывают сексуальный гнев в сторону мужчин и используют своих дочерей для удовлетворения своих эмоциональных потребностей, по ходу дела извращая представления дочери о мужчинах.
В общем сексуальное поведение в отношение детей того же возраста не считается злоупотреблением. Злоупотреблением это становится, когда возраст одного из детей различается на 3-4 года и больше.

\end{enumerate}


\ж{Физическое насилие}

Физическое насилие стоит на втором месте после сексуального по степени генерирования стыда. Хуже всего то, что физическое насилие аддиктивно. Насильники, подгоняемые токсическим стыдом, становятся зависимыми от физического насилия.
Образ родителя, применяющего физическое насилие: изолированный, с плохой самооценкой, с низкой чувствительностью к чувствам других, чаще всего сами страдали от физического насилия, недостатка материнской любви и неудовлетворёнными потребностями в любви и комфорте, отрицают проблемы и воздействие этих проблем, не видят никого к кому могли бы обратиться за советом, имеют нереалистичные ожидания от детей, ожидают удовлетворения своих потребностей от детей, а в случае отказа, отвечают злостью и разочарованием, общаются с детьми как будто они старше, чем есть на самом деле.
Такие родители уверены, что волю ребёнка нужно сломать и делают это через насилие.
Жертва физического насилия также привязывается к нему из-за стыда. В самом начале жертва связывается со стыдом из-за ужаса. Со временем, самооценка падает и жертва теряет способность выбирать. Они становятся детьми голодными до кусочков и крох любви.
Насилие чаще всего непредсказуемо и импульсивно, что ведёт к <<наученной беспомощности>>. Наученная беспомощность это вид спутанности сознания. Человек теряет способность думать и планировать, он становится пассивным принимающим насилия.
Физическое насилие очень часто в семейной жизни, т.к. это все ещё поощряется как способ воспитания детей.
Физическое насилие бывает разного вида: шлёпанье, угроза ремнем, удары, пощёчины, дёрганье, удушение, пинки, угроза отказом, угроза посадить в тюрьму или вызвать полицию, свидетельство насилия в сторону родителя или родственника.
Свидетель насилия становится жертвой насилия в семья, где муж бьёт жену.

\ж{Эмоциональное насилие}

Эмоциональное насилие универсально, все испытывали его. Ядовитая педагогика ясно говорит, что эмоции это слабость и мы должны быть рациональными и логичными и не позволять своим эмоциям управлять нами. Все эмоции должны быть под контролем, а гнев и сексуальные эмоции должны быть подавлены.

\ж{Связывание эмоций со стыдом}

Наши эмоции являются нашей базовой силой. Они выполняют две важные функции: контролируют наши основные потребности и сообщают нам о нужде, потере или насыщенности; дают нам <<топливо>> или энергию для действия.

Когда достижение основных потребностей нарушается, мы испытываем гнев, он помогает нам отвоевать то, что нам нужно.

Грусть - это энергия, которую мы выпускаем, что бы исцелиться. Когда мы выпускаем эту энергию по-поводу потерь, относящихся к нашим основным потребностям, мы можем интегрировать шок этих потерь и приспособиться к реальности. Грусть болезненна и мы пытаемся избежать её, но в действительности, выпускание этой эмоции также выпускает энергию, которая вызывает боль. Задерживая эту эмоцию, мы также задерживаем боль.

Страх - энергия, которая предупреждает нас об опасности нашим основным потребностям. Страх это энергия, которая ведёт нас к проницательности и мудрости.

Вина - говорит нам о том, что мы преступили наши ценности. Эта эмоция позволяет нам действовать и изменяться.

Стыд - предупреждает нас о том, что мы вышли за рамки человечности. Это сигнал о наших основных ограничениях.

Радость - оживляющая энергия, которая возникает, когда все наши потребности удовлетворены. Мы хотим петь, бегать и прыгать. Это сигнал о том, что все в порядке.

Когда наши эмоции не зеркалятся и не называются, мы теряем контакт с нашими жизненно важными силами. Родители, которые сами потеряли этот контакт, не могут моделировать эти эмоции. Они психически онемели. Они сами понятие не имеют, что чувствуют. Отсюда их действия по остановке эмоций детей.

Семейные традиции и религия одобряют ядовитую педагогику, которая говорит о недопустимости эмоций как важной части жизни. В особенности самой плохой эмоцией считается гнев - один из смертных грехов (однако в действительности грехом является действие в соответствии с генов - крики, драка, критиканство и т.д.).

\ж{Shame Parfaits (хуй знает как это перевести)}

При пристыжении гнева, происходит две вещи. Первое, гнев становится повязанным стыдом, т.е. каждый раз испытывая гнев, человек также испытывает стыд. Второе, пристыженный гнев подавляется. Подавление это основная защита ЭГО и работает подсознательно, после первоначальной <<настройки>>. Энергия гнева отправляется в подсознание, но стремится выразится. Со временем подавляется все больше гнева, гнев растёт в силе.
Вирджиния Сатир один раз сравнила этот процесс с заточением в подвале голодных собак. Чем голоднее они становятся, тем сильнее они пытаются вырваться. Чем сильнее они это делают, тем надёжнее мы должны их охранять. Подавленная энергия растёт и растёт, и в конце концов обретает свою собственную жизнь. В один день энергия вырывается. Человек, все это время подавлявший злость, находит себя в неконтролируемой ярости.
Подавленная, неразрешённая энергия гнева превращается в ярость.

Когда пристыживают печаль, она вырастает в безутешную скорбь и отчаяние. Иногда это становится основой суицидальности. В нашей культуре, детей пристыжают за плачь. Если не пристыжают, то пытаются остановить при помощи подкупа. Иногда детей бьют за плачь.

Тоже самое со страхом - пристыжение страха превращает его в полную паранойю и ужас. Чаще всего разрешение на переживание грусти и страха связывается с полом. Мальчики не должны плакать или боятся. Девочкам это чаще всего разрешено.

Пристыжают даже радость. Это приводит к ощущению стыда каждый раз, когда вы чувствуете себя счастливым.


\ж{Пристыжение сексуального желания}

Возможно, сексуальное желание является самым подавляемым из всех желаний человека. Сексуальность это ядро самости человека. Наш пол, это не что-то, что мы имеем или должны что-то делать по этому поводу, это то, кто мы есть. Сексуальность это базовый фактор всех существ. Отключив этот стимул, мы уничтожим всю человеческую расу за одно поколение. Наша сексуальная энергия (либидо) это наша уникальная инкарнация жизненной силы. Пристыжение этой силы приводит к пристыжению всего ядра личности.

Все дети от рождения заинтересованы сексуальностью. Это что-то удивительное, странное, поэтому дети натурально стремятся узнать об этом больше, исследуют собственный гениталии, а с определенного возраста участвуют в детских сексуальных играх.

Примером пристыжения сексуальности в детском возрасте может служить подобное поведение: ребёнок исследует мир и себя, называет части своего тела, а родители и другие взрослые в его окружении одобрительно улыбаются, смеются и подбадривают его в дальнейших поисках. Однажды ребёнок находит свои гениталии, выбегает к взрослым и радостно показывает, что же он нашёл и тут же получает волну негатива, в какой бы форме это не выражалось. Подобные случаи становятся основой для интернализации стыда в отношении сексуальности.

Подобное поведение родителей основано на их собственном интернализированном чувстве стыда по-поводу сексуальности. Когда ребёнок исследуют свою сексуальность, такой родитель реагирует неодобрением и даже отвращением: <<Это плохо!>>, <<Не трогай себя там!>>, <<Веди себя прилично, прикройся!>>. Это связывает сексуальность с чем-то плохим и отвратительным. Поэтому, от этой части себя нужно отречься. Стыд становится связанным с сексуальностью.
ребёнок, вырастая в такой семье, чувствует, что проявление сексуальности является постыдным.


В общем, большая часть нашей спонтанной инстинктивной жизни становится постыдной. Детей пристыжают за излишнюю раздражительность, за желание чего-либо и за громкий смех. Большая часть стыда образуется за обеденным столом, когда детей заставляют есть, даже если они не голодны или есть то, что не выглядит аппетитным.

Когда наша инстинктивная жизнь становится повязанной со стыдом, все ядро человека становится также повязанным. Это как если бы жёлудь проходил через мучительную агонию за вырастание дубом, или цветок, чувствующий стыд за цветение. Мы не можем подавить наши инстинкты, поэтому как только они будут повязаны стыдом, они становятся как голодные собаки, за которыми нужно постоянно следить.


\ж{Стыд - главная эмоция}

Мы называем стыд главной эмоцией, потому что когда он интернализирован, остальные эмоции также становятся повязанными на стыд. Эмоционально повязанные стыдом родители не могут позволить своим детям иметь эмоции, потому что эмоции ребёнка становятся триггером эмоций родителя. Подавленные эмоции часто ощущаются слишком большими, как будто они полностью поглотят нас, если мы их выразим. Также присутствует страх перед стыдом, который будет также запущен, если мы выразим свои эмоции.


\ж{Заброшенность: пренебрежение потребностями ребёнка (любовь, привязанность, безопасность, уход, питание, тепло)}

Дети зависят от других в вопросах удовлетворения своих нужд. Им нужды родители до 15 лет. Их потребности (пикрилейтед) могут быть удовлетворены только родителями (или другими опекунами). Детям нужен кто-то, кто будет их держать, трогать, зеркалить и подтверждать их чувства, нужды и желания. Детям нужна структура с границами; предсказуемость. Взаимные доверительные отношения. Детям требуется своё пространство и разрешение быть другими. Также безопасность, питание, одежда, жилье и медицинский уход, время и внимание родителей.

\ж{Связывание потребностей со стыдом}

Когда ребёнку отказано в этих потребностях, то он понимает это так, что его потребности не важны и он теряет чувство собственной ценности. Он не достоин чьего-то внимания. По мере того, как потребности ребёнка отклоняются, он перестаёт верить, что у него есть право зависеть от кого-то. Эти потребности зависимости зависят от межличностного моста и взаимной связи для их удовлетворения. Именно этот межличностный мост разрушается при заброшенности и пренебрежении. У нас теперь нет никого, от кого мы можем зависеть, мы приходим к выводу, что у нас нет прав для этого. Мы чувствуем стыд, когда нуждаемся в чем-то. Так как эти потребности являются базовыми, т.е. мы не можем быть полноценным человеком без них, мы вынуждены удовлетворять их нездоровыми путями.

Заброшенный ребёнок может научиться получать внимание попадая в неприятности или раздражая своих родителей. Бывают случаи, когда дети удовлетворяют нужду в прикосновениях через получение шлепков.

Заброшенность через пренебрежение потребностями в зависимости является главным фактором в становлении взрослого ребёнка. Мы вырастаем, выглядим как взрослые, говорим как взрослые, но под поверхностью живёт ребёнок, который чувствует пустоту и нужду, ребёнок, потребности которого ненасытны, т.к. у него присутствуют детские потребности во взрослом теле. Этот ненасытный ребёнок становится корнем всех компульсивных/аддиктивных поведений.

Когда кто-то становится повязан стыдом через заброшенность, боль от этого глубока и всеобъемлюща. Человек чувствует себя ничтожеством, мелким и раскрытым. Такое состояние может привести к побегу в сексуальное удовольствие и оргазм, который становится наслаждением, заменяющим все остальные потребности человека.
Вот что говорит Kaufman о процессе конвертации всех потребностей в сексуальные:
<<Мальчик, который научился не требовать ничего эмоционального от родителей, сталкивается с дилеммой в каждом случае, когда он чувствует себя нуждающимся или неуверенным в чем либо. Если мастурбация была его основным источников приятных чувств, он может начать находить утешение в мастурбации для восстановления самочувствия каждый раз, когда он будет переживать любую потребность, даже совершенно не связанную с сексуальной.>>

Защитные механизмы ЭГО преобразуют любые важные для развития потребности в потребности чего-то другого. Это может быть еда, деньги или чрезмерное внимание.

\ж{Заброшенность через исполнение ролей для удовлетворения открытых и скрытых потребностей семейной системы}

Ранее я уже описывал, что такое семейная система, каковы её компоненты, роли и законы динамического гомеостаза. Вы видели, как дисфункциональные семьи используют членов семьи для поддержания баланса. Чем более дисфункциональной является семья, тем более жёсткими и закрытыми будут роли её членов.
Все жёсткие роли, построенные дисфункциональными семьями, являются формами заброшенности. Быть <<Героем>> семьи, значит не показывать свою уязвимость, играть роль бесстрашного. Эти роли как сценарии, по которым нужно играть. Они предполагают, какие чувства вы можете и не можете испытывать.
Роли нужны для поддержания баланса всей системы, а значит, ребёнок, играющий какую-либо роль, должен отказаться от своей аутентичной личности для исполнения своей роли.
Каждая форма заброшенности нарушает межличностный мост и взаимную связь.

\ж{Взаимосвязь образов}

Третий пункт интернализации происходит через интернализацию образов. Эти внутренний образы могут быть в виде человека, который ранее пристыжал человека, места или самого переживаемого опыта. Также они могут быть в форме звуков. Услышав определённые слова мы можем запустить старое ощущение стыда. Наши индивидуальные переживания стыда сливаются вместе по средствам языка и образов. По мере того, как слова, образы и события, ассоциируемые со стыдом, сливаются вместе, смысл стыда проходит трансформацию. <<Я чувствую стыд>> превращается в <<Я постыден, неполноценен как человек в основном смысле это слова>>. Стыд перестаёт быть одним чувством среди многих и становится составной частью ядра личности. Это ядро дефективности создаёт фундамент, через который будут восприниматься все другие чувства о себе. Со временем, стыд становится базовой идентификацией человека.

\ж{Функциональная автономия}

После интернализации стыда, это чувство может быть активировано без внешнего стимула. Больше нет необходимости в событии, которое может вызвать стыд.

\ж{Спиральная динамика стыда}

Последним следствием интернализированного стыда будет то, что Kaufman называл <<внутренней спиралью стыда>>. Он описывал это так:
<<Происходит вызывающее событие. Возможно, попытка сблизиться с кем-либо и получения отказа. Или критическое замечание друга. Человек впадает в стыд, внимание переводиться внутрь и все переживание становится внутренним. Чувство стыда ходит по кругу, стимулируя все больше чувств. Первоначальное событие переживается снова и снова, заставляя углубиться чувству стыда, впитать в себя другие нейтральные переживания. В конце концов человек становится парализован стыдом.>>

Спиральная динамика это один из самых разрушительных аспектов дисфукнционального стыда. Когда спираль запущена, она может вызвать переживание других постыдных событий и таким образом все больше укрепляя стыд внутри личности.

После интернализации стыда, страх раскрытия усиливается во много раз. Раскрытие теперь значит раскрыть свою дефективность как человека другим на обозрение. Раскрыться, т.е. быть видимым непоправимо и неописуемо плохим. Человек должен найти способ защититься от подобного раскрытия. С развитием подобных механизмов защиты, интернализированный стыд становится все менее осознанным.

Интернализированный стыд имеет 4 главных следствия: формирование стыдливой идентичности; углубление и усиление стыда с последующей <<заморозкой>> состояния; активация автономной функциональности стыда; запуск механизма спиральной динамики стыда.


\ж{Школьная система}

\ж{Перфекционизм}

Перфекционизм, одно из правил семейных систем, является коренным виновником в формировании токсического стыда. Мы увидим это также в религиозных и культурных системах.
Перфекционизм отрицает здоровое чувство стыда. Он делает это через предположение, что мы можем быть идеальными. Такое предположение отвергает нашу ограниченность как человека. Перфекционизм отрицает то, что мы будем часто совершать ошибки и что это естественный процесс.
Перфекционизм всегда имеет место в том случае, когда мы берём какую-то негативную норму и возводим её в абсолют. После этого, норма становится мерой ко всему. Мы сравниваем и осуждаем на основе этой нормы.

В школе нас сравнивают относительно отличной оценки. Если мы делаем ошибки, нам дают оценку по убывающей шкале. Когда ребёнок часто получает низкие оценки, он начинает ассоциировать себя с неудачником. Плохие оценки видятся как оценка личности. И чаще всего, те дети, которым тяжело даётся школа, уже изначально имели токсический стыд и именно это явилось результатом их провала. Провал в школе ещё больше углубляет чувство стыда. Токсический стыд порождает токсический стыд.

Обратной стороной токсичного стыда является супер-успешные ученики, которые подобным образом прикрывают свой стыд.

Школьная система способствует постыдной классификации интеллекта человека. Это было бы только на половину неправильно, если бы подобное измерение действительно определяла бы способности интеллекта. Я не думаю, что интеллект это способность запоминать и воспроизводить знания из памяти. Интеллект это не то, что ты знаешь как делать, а то, что ты делаешь, когда не знаешь, что делать.

Перфекционизм также порождает разрушительную конкуренцию. Конечно, есть полезная форма конкуренции, которая позволяет нам становится лучше и расти. Но перфекционистская школьная система поощряет обман и создаёт сильный стресс. Оценки часто выставляются на показ и в этом ощущается постыдное раскрытие тех, кто получает низкие оценки.

\ж{Рационализм}

Наши школы делают большой уклон в сторону образования ума, нежели всего человека. Мы делаем упор на разумности, логике и математике, и почти не заботимся об эмоциях, интуиции и творчестве. Наши ученики становятся зубрилами и унылыми конформистами, нежели захватывающими и чувственными творцами.

Было проведено множество исследований в последнее время над работой правого полушария мозга. Эта часть мозга является источником <<чувственных мыслей>>. Чувственные мысли это ядро музыки и поэзии. Правое полушарие холистическое и интуитивное. Оно использует воображение, нежели память. Ученики, у которых от рождения есть склонность к этой части мозга, наказываются.

Наш рациональный уклон вызывает пристыжение и отвержение воображения и эмоций. Таким образом школы пристыжают одни из самых ярких и творческих аспектов человеческой психики.

\ж{Пристыжение в группе ровесников}

Период полового созревания это время, когда человек чувствует себя наиболее раскрытым и уязвимым. Какой бы токсический стыд он не нёс в себе, он будет проверен в этот период. Человек может стать объектов насмешек группы сверстников из-за любой детали, физической или нет, или же стать участников подобной группы и пристыжать другого козла отпущения. Обычно дети пристыжают других в такой же форме, в какой родители или родственники пристыжали их самих.

\ж{Религиозные системы}

Множество религиозных конфессий учат тому, что человек изначально грешен, запятнан так называемым первородным грехом.

\ж{Бог как каратель}

Вера в то, что Бог ведёт счёт всех поступков и обязательно покарает за все грехи становится сильнейшим усилителем токсического стыда. Учитывая первородный грех, человек изначально находится в проигравшей позиции.

\ж{Отрицание вторичной причинности}

Одним из самых коварных и токсичных религиозных искажений из множества является отрицание вторичной причинности. Это значит, что в соответствии с некоторыми доктринами, человеческая воля некомпетентна. Нет ничего ценного, что может сделать человек. Сам по себе человек это червь. Только когда Бог действует через него, человек возвращает своё достоинство. Подобная интерпретация показывает человека как изначально дефективного, т.е. по сути определяет токсический стыд.

\ж{Отрицание эмоций}

Религиозные системы в общем не считают эмоции чем-то важным. Существуют конфессии и секты, которые являются чрезмерно эмоциональными, но в большинстве своём выражение эмоций в религиозных системах не поощряется.

Я вижу два типа религиозных структур - Apollonian и Dionysian. Ни один из них не позволяет здорового выражения эмоций.

Apollonian тип религии очень жёсткий, стоический. Свободное выражение эмоций в них не принято.
Dionysian тип характеризуется харизматичным и культовым поклонением. Подобные виды поклонений включают в себя бурное выражение эмоций, но только определённых эмоций. Подобные виды религий используют эмоции как что-то внешнее, не имеющее внутренней причины, они не настоящие.
Ну и конечно религии запрещают выражение сексуальности и гнева.

\ж{Перфекционизм - религиозное писание}

Религия всегда была главным источником пристыжений через перфекционизм. Моральные правила и обязанности были наложены с помощью субъективной интерпретации религиозных откровений. Библия использовалась для оправдания всевозможных обвинений и осуждений. Религиозный перфекционизм учит какой-то поведенческой праведности. Существуют религиозные тексты, которые содержат стандарты святости и праведности. Эти стандарты диктуют как говорить, одеваться, ходить и вести себя практически в любой ситуации. Отклонение от этих стандартов считается греховным.

Перфекционисткая система создаёт правила поведений <<как делать правильно>>. В таких правилах обучают тому, как действовать с любовью и праведностью. На деле становится важнее действовать любящим и праведным, чем на самом деле любить и быть праведным. Подобное поведение отличным способ успокаивать свой токсический стыд, а также передавать его другим.

\ж{Религиозная зависимость}

Изменение настроение это ингредиент компульсивного/аддиктивного поведения. Зависимость иногда описывают как патологическое взаимодействие с любым изменяющим настроение опытом, которое имеет опасные для жизни последствия. Как было предложено ранее, токсический стыд является ядром и топливом всех зависимостей. Религиозная зависимость также берет свои корни в токсическом стыде и она также может быть изменяющей настроение через различные религиозные поведения. Человек может получать ощущения праведности через любые формы поклонений. Можно держать пост, молиться, медитировать, служить другим, проходить через ритуалы, цитировать Библию, проносить имя Иисуса. Любое из этих действий может изменять настроение. Если у человека присутствует токсический стыд, такие действия могут стать невероятно вознаграждающими.


\ж{Культурные системы}

В книге Bradshaw On.- The Family рассматривается все общество, построенное на больных семейных системах, основанных на правилах ядовитой педагогике. Эти правила отвергают эмоции. Подобное отношение вызывает психическое онемение, которое ведёт к зависимостям. Эти правила пришли со времён королей. Они не демократичны и основаны на неравенстве взаимоотношений владелец-раб. Они предполагают навязчивую порядочность и послушание. Они жёстки и отрицают саму жизнь. Хорошие дети определяются как кроткие, внимательные, бескорыстные и совершенно законопослушные. Такие правила не оставляют места для витальности, спонтанности, внутренней свободе, внутренней независимости и критическому суждению. Такие правила заставляют родителей, даже имеющих благие намерения, покинуть своих детей. Подобная заброшенность создаёт токсический стыд.

\ж{Общество - компульсивное и зависимое}

Наше общество невероятно аддиктивно. Миллионы людей серьёзно страдают от алкоголя, никотина и других наркотиков. Миллионы семей испытывают насилие внутри семьи. Почти половина людей имеют проблемы с пищевой зависимостью. Нет точных данных о трудоголизме и сексуальных зависимостях. Если токсический стыд является топливом всех зависимостей, то это значит, что существуют серьёзные проблемы со стыдом в нашем обществе.

Другим индикатором безнадёжности, которая берет своё начало и является результатом стыда это наша лихорадочная суета и компульсивный стиль жизни. Эрих Фромм отлично описал это в своей книге <<Революция надежды>>. Он увидел нашу суету как сигнал беспокойства и недостатка внутреннего мира с самим собой, который выходит из нашего стыда. Токсический стыд не позволяет нам идти во внутрь, т.к. это слишком болезненно и безнадёжно.

\ж{Миф успеха}

Кто-то однажды сказал: <<Успех различается в разные периоды жизни - начиная с того, что бы не намочить кровать в младенчестве, быть любимым в детстве, иметь секс в молодости, зарабатывать деньги и иметь авторитет во взрослой жизни, иметь секс в среднем возрасте, быть любимым в старости и не намочить в штаны в глубокой старости.>> Что особенно интересно в этом описании, это акцент на зарабатывании денег, авторитете и любви.

Возможно, одной из великих современных американский трагедий была пьеса The Death Of A Salesman - Arthur Miller. Миллеру вдалось создать аристотелевского трагического героя из обычного человека. Willy Loman стал символом американского мифа успеха. Он живёт жизнью основанной на убеждении, что успех это быть любимым и зарабатывать деньги. Он умирает в одиночестве и в нищете, кончая жизнь самоубийством, что бы получить страховую выплату, которая докажет то, что он был успешным. В своей <<Поэтике>> Аристотель утверждает, что сила трагического героя заключается в комбинации его благородства вместе с каким-либо трагическим изъяном. Вилли был благородным. Он готов умереть за свою веру. Его вера и стала трагическим изъяном. Он действительно верить, что если человек зарабатывает деньги и любим всеми, он будет успешным.

Миф успеха также продвигает идею прочного человека. Человек должен достичь всего сам. В этом мифе деньги и их символы являются мерилом успешности. Человек в свои 50 лет с низким доходом должен чувствовать стыдливые уколы от своих собственных убеждений. Даже если человек пытается протестовать подобному положению вещей, деньги и слава, которая следует за ними, все равно имеет огромную силу в нашей жизни.

\ж{Жёсткие роли полов}

Жёсткие роли полов, навешиваемые общество, все ещё являются символом идеала. Существуют настоящие мужчины и женщины. До нашего рождения уже существует проект того, как быть настоящим мужчиной и женщиной.

Настоящие мужчины жёсткие, они больше делают, чем говорят. Они мало говорят. Настоящий мужчина никогда не показывает свои слабости, эмоции и уязвимости. Настоящие мужчины победители.

Настоящая женщина это помощница настоящего мужчины. Они должны заботиться о домашнем очаге. Они эмоциональны, уязвимы и хрупки. Они играют роль миротворцев. В обмен они хотят получить вечную романтическую любовь. Они ищут принца, который придёт и наградит их за все, от чего они отказались. Наградой же будет забота со стороны настоящего мужчины до конца жизни.

Можно подумать, что эти роли уже в прошлом, но это не так. Достаточно последить за тем, как родители воспитывают своих детей, как воспитание различается в отношении мальчиков и девочек. Присмотритесь к тому, как их одевают, какие игрушки покупают, каким голосом говорят, за что наказывают и что поощряют. Это разное отношение к полам приводит к формированию тех ролей, которые, казалось бы, остались в прошлом.

Эти сексуальные роли жёстки и вносят раскол. Они также постыдны по своей сути, т.к. являются карикатурами мужественности и женственности. Они превозносят одни части нашей личности, при этом не позволяют быть целостными. Каждый из нас носит в себе как женские так и мужские гормоны. Каждый пол определяется большинством гормонов, но каждому полу нужно интегрировать свою противоположность, что бы стать целостной личностью. Жёсткие сексуальные роли устанавливают стандарты, которые отрицают целостность и завершённость. Такие стандарты стыдят нашу противоположную часть. Мужчину стыдят за попытку принять свои уязвимости, а женщин за настойчивость и другие мужские аспекты.

\ж{Миф идеальной десятки}

Наша культура представляет собой перфекционисткую систему, которая жестоко стыдит физически обделённых. Идеальный мужчина и идеальная женщина это <<10>>.

Идеальная <<10>> имеет определённые атрибуты, которые усиливают сексуальное пристыжение. Идеальная женщина <<10>> имеет идеальные округлые груди, размер 38D, с соответствующими бёдрами и ягодицами. Идеальные мужчина <<10>> это мускулистый, загорелый и пропорционально идеально сложенный. Конечно, с пенисом в 20 см.

Эти физические идеалы стали причиной страданий и стыда у огромного числа людей. У меня было множество случаев, мужчин и женщине, которые испытывали сильнейший стыд из-за размеров своих гениталий. Множество женщин с небольшой грудью и болезненными школьными воспоминаниями были в моем офисе за последние 20 лет.
Сравнение себя с идеальной десяткой стало огромным источников сексуального стыда в нашем обществе.

\ж{Отрицание эмоций}

Наша культура не очень справляется с эмоциями. Мы любим, когда люди счастливы и в порядке. Мы учимся ритуалам счастливого притворства с ранних лет. Я помню множество случаев, когда я говорил людям <<Все отлично!>>, в это же время чувствую как весь мир обрушивается на меня. Настоящие эмоции не одобряются на рабочем месте и мы с охотой примем счастливую актёрскую игру, чем настоящие эмоции, отличные от позитивных.

\ж{Миф о добрых парнях и прелестных девчатах}

Миф о добрых парнях и прелестных девчатах это что-то вроде мифа социального соответствия. Они создают парадокс, когда сталкиваются с мифом успеха, а конкретно с частью о жёсткой личности. Как я могут быть жёстким, быть сам себе господином, и в то же время быть конформным? Конформность значит <<Не поднимать волну>>, <<Не качать лодку>>.

<<Нас учат быть приятным и вежливым. Нас учат тому, что это лучше чем говорить правду. Наши церкви, школы и политики учат нас быть нечестными. Мы улыбаемся, когда нам грустно; нервно смеёмся переживая горе; смеёмся над шутками, которые не считаем смешными; говорим людям вежливые вещи, хотя совершенно так не считаем.>> Bradshaw On: The Family

Актёрство это форма вранья. Если человек действует так, как он действительно чувствует и это раскачивает лодку, он подвергается остракизму. Мы поощряем притворство и вранье как часть культурной жизни. Подобная жизнь вызывает внутреннее разделение. Это учит нас скрывать и прикрывать наш токсический стыд. Это отсылает нас все глубже в изоляцию и одиночество.


\ж{Где прячется токсический стыд}

\ж{Ощущение токсического стыда}

Токсический стыд возникает из неожиданного раскрытия уязвимого аспекта ребёнка. Это раскрытие происходит до до того, как у ребёнка появляется ЭГО границы, которые могли защитить его. Раннее постыдное событие происходит в контексте, в котором у него нет способности выбирать. Ощущение стыда это ощущение раскрытия, когда человек не готов к тому, что бы его видели. Токсический стыд часто проявляется во снах, когда человек видит себя нагим в неподходящем месте или чувствует неподготовленность к какому-то неожиданному событию, например, к экзамену.

Неожиданное качество стыдливого события создает у ребёнка недостаток доверия к себе. С развитием стыда, ребёнок перестаёт верить своим глазам, суждениям, чувствам и желаниям. Эти способности составляют нашу базовую силу. Недоверие в наши базовые силы приводит к чувству беспомощности. Когда уязвимые аспекты человека пристыжаются, они отвергаются и отделяются от нашего ощущения самости. Это самоотделение приводит к разделённой личности. Мы становимся объектами для самих себя. Когда я становлюсь объектом, я перестаю быть мной. Я чувствую пустоту и раскрытие. У меня нет границ, а значит, я не защищён. Я должен бежать и прятаться, но не существует места, где бы я мог спрятаться, т.к. я полностью раскрыт. Они следуют за мной и поймают меня врасплох. Не существует момента, когда я могу расслабиться. Я должен постоянно стоят на страже. Я один в самом полном смысле этого слова.

Агония этой хронической стадии не может долго продолжаться. На глубоком уровне токсический стыд запускает наши базовые механизмы автоматической защиты. Фрейд называл их первичными защитными механизмами. Как только эти механизмы будут запущены, они продолжают работать сами по себе в подсознании, отправляя нашу аутентичную личность скрываться. Мы развиваем ложную идентичность из этого базового ядра. Мы становимся мастерами имитаторами. Мы избегаем нашу корневую агонию и боль, а о временем, мы начинаем избегать избегание.

Пикрилейтед показывает различные слои защиты, которые мы разрабатываем для защиты от боли. Каждый слой более осознан. Самый глубокий слой это наши защита ЭГО, роли в семейных системах - автономны и бессознательны. Каждый слой компульсивен и каждый характерный элемент нашего щита секретности, кажется, имеет свою отдельную жизнь.

\ж{Первичные защитные механизмы}

Фрейд был первым, кто чётко определил автоматические процессы используемые для самосохранения, которые активируются перед лицом серьёзной угрозы. Всем нам нужно время от времени использовать эти механизмы. Они изначально были созданы как ситуационная защита, нежели хроническая.

Дети беззащитны и бессильны. Каждому ребёнку требуется сформировать границы и силу ЭГО. Детям нужны защитные механизмы даже больше чем взрослым. Они нужны им до того, как они смогут развить устойчивые границы. Для этого, им нужны родители с сильными ЭГО границами, но у людей с токсическим стыдом их просто нет. Без границ и защиты, ребёнок просто не может выжить. Природа обеспечила детей первичными защитными механизмами, которые выполняют роль границы. Каждый механизм защиты позволяет ребёнку пережить ситуации, которые на самом деле невыносимы.

\ж{Отрицание и <<фантазия связи>>}

Возможно, самый элементарный защитный механизм это отрицание. Перед лицом угрозы, люди отрицают то, что происходит, или отрицают боль, или воздействие события на их жизнь.
Robert Firestone развил идею Фрейда назвал этот механизм <<фантазией связи>>. Фантазия связи это иллюзия связанности, которую создает ребёнок в отношении родителя, который пристыжает его.
Парадоксально, но чем больше ребёнок страдает от действий родителя, тем больше становится эта фантазия связи.
Насилие обычно непредсказуемо, как случайный шок. Насилие снижает самоуважение и уменьшает возможности выбора альтернативы.
В конце-концов, человеку не остаётся никакого выбора и часто он выбирает прицепиться к насильнику. Фантазия связи это иллюзия того, что кто-то находиться рядом с ними, любит и защищает их.

\ж{Подавление}

Любое невыносимое событие сигнализирует сильными эмоциями. Эмоции это форма энергии в движении. Они сигнализируют о потере, угрозе или насыщенности. В каждом случае, когда ребёнок страдает от любой формы заброшенности, возникают чувства гнева, боли и грусти. Так как родители, повязанные стыдом, не могут переносить свои собственные эмоции, они также не переносят проявлений эмоций ребёнком, поэтому они стыдят его за его эмоции. Подавление это способ заглушения эмоций, что бы больше не чувствовать их. Не совсем ясно, как именно работает механизм подавление, но определённо то, что это как-то связано с напряжением мышц, изменением дыхания и фантазиями.

\ж{Стирание субъективного опыта}

Кауфман говорит о прямой связи связанности эмоций со стыдом и защитным механизмом подавления. Он предполагает, что после определенного периода времени пристыжения эмоций человека, он входит в состояние, которое Кауфман называет <<опытом стирания>>. Этот опыт стирания эквивалентен подавлению. Эмоции переживаются внутренне, перед тем как выражаются в открытую. Кауфман пишет:

<<Само переживание определенного чувства может быть подавлено, если связывающий эффект стыда распространяется на внутреннее осознанное регистрирование эффекта связывания стыдом. В момент, когда сознание неожиданно чувствует раскрытие, даже если только себе, понимание содержания этого осознания может быть стёрто.>>

Это стирание опыта является механизмом защиты. Постепенно, мы учимся даже не быть в курсе переживания чувства, которое вызвало стыд. Мы учимся ничего не чувствовать.

\ж{Диссоциация}

Диссоциация это защитный механизм ЭГО, который сопровождает самые агрессивные формы пристыжения - сексуальное и физическое насилие. Травма такая большая, а страх настолько ужасен, что человеку нужно мгновенное облегчение. Диссоциация это форма мгновенного онемения. Оно включает в себя механизмы отрицания и регрессии, также элементы отвлекающего воображения.

Жертва инцеста просто уплывает в мечтания во время акта насилия. Тоже самое правдиво для физического насилия. Боль и унижение невыносимо - жертва <<покидает>> своё тело.

Это причина, по которой эти формы виктимизации так сложно лечить. Воспоминания экранированы, в то время как чувства остаются. Жертва часто чувствует себя как сумасшедший, как будто он живёт в вымышленном мире. Также часты случаи, когда у жертва происходит расщепление личности. Из-за того, что связь между насилием и реакцией на насилие была разорвана, жертва думает, что её сумасшествие и стыд говорят о ней, нежели о том, что случилось с ней.

Диссоциация не ограничивается сексуальным и физическим насилием. Эмоциональное насилие, серьёзные травмы, хронический стресс также могут стать факторами, приводящими к диссоциации. Диссоциация может продолжаться всю жизнь.

\ж{Вытеснение}

Вытеснение тесно связано с диссоциацией.
Один из моих клиентов, чей отец часто приходил к ней в комнату после закрытия баров и насиловал её, часто просыпалась около 3:00 часов ночи и видела тёмную фигуру в своей комнате. У неё также были повторяющиеся кошмары с фигурой чёрного монстра, который тыкал в неё и бил своим большим чёрным пальцем. Когда она пришла на терапию, она понятие не имела, что пережила инцест.

Ей было 26 лет, когда мы начали терапию. Когда мы провели с ней работу с историей семьи и находясь в гипнотическом состоянии возрастной регрессии, она начала вспоминать о своей спальне и отце. По-началу, воспоминания были нечёткими, но со временем, углубляясь в детали, она начала рассказывать о том, как её отец заставил её заняться с ним сексом. За месяца работы она реконструировала два с половиной травматических года, начиная с 4 с половиной лет. Она была любимицей отца. Он также угрожал ей, если она кому-нибудь об этом расскажет.

Как только она нашла связь между эмоциями и самими событиями, её тёмное вытеснение и чёрный монстр пропали из её снов.

\ж{Деперсонализация}

Тесно связанная со смещением.
Деперсонализация это поведенческое проявление переживания насилия. Это чаще всего происходит в отношении со второй половинкой, с которой человек перестаёт ощущать свою субъективную личность. Человек ощущает себя как объект. Результатом этого становится потеря осознания внутреннего переживания. С продолжением насилия, человек перестаёт воспринимать реальность своего переживания и окружающей среды.

\ж{Идентификация}

В случаях виктимизации, жертва часто идентифицирует себя со своим притеснителем. Делая это, жертва перестаёт чувствовать свою беспомощность и стыд из-за унижения. Идентифицируясь с притеснителем, человек перестаёт чувствовать стыд.

\ж{Конверсия}

\ж{Подмена}

Я уже говорил о том, как в процессе интернализации стыда мы отрекаемся от наших жизненно важных частей личности. Отщеплённые части нашего внутреннего переживания (чувства, потребности и желания) требует выражения. Они как голодные собаки, закрытые в нашем подвале. Мы должны найти какой-то способ утихомирить их. Один из способов, это конверсия чувств. В этой конверсии, мы конвертируем запрещённые и стыдливые чувства в более приемлемые.

Гнев может конвертироваться в более приемлемые чувства боли и вины.

Трёхлетний Петя в ярости, т.к. его мама обещала отвести его в Баскин-Робинс, но теперь отказывается это делать. Петя кричит, что ненавидит свою мать. Мама, повязанная стыдом, в ужасе от гнева, своего или чужого. Гнев Пети запускает её собственный гнев в сторону её родителей. Так как её гнев повязан стыдом и виной, она не допускает собственное чувство стыда через пристыжение и обвинение Пети.
Она говорит ему, как её обижает, когда он злится на неё. Она начинает плакать, т.к. она научилась ещё маленькой девочкой конвертировать гнев в грусть.
Сцена из детства матери: её отец был в гневе, т.к. она не хотела идти спать и продолжала играть. Когда отец выражал свой гнев, она начала рыдать. Отец почувствовал себя плохо, поднял её на руки и начал гладить. Он дал ей стакан сока и начал укачивать её. Как ребёнок, мать Пети научилась тому, что грусть приемлема и даёт ей силу. Гнев не работал в её семье. Когда Петя говорит, что ненавидит её, она плачет и говорит ему, что возможно в один день её не будет дома, когда она будет нужна ему. Или она может даже умереть.
Бедный Петя опустошён. Его заброшенность, страх и тревога разделённости активировались. Он бежит к своей маме с чувством сильной вины. Его осознание своего гнева полностью потеряно. Его гнев был конвертирован в вину.

Иногда родители реагируют гневом на выражение детьми своего гнева, страха или грусти. Когда такое происходит, чувства ребёнка повязываются стыдом или конвертируются в страх и ужас.


\ж{Соматическая конверсия}

Ещё одна форма конверсии предполагает конверсию потребностей и чувств в какое-либо телесное или соматическое выражение. Потребности и чувства могут быть трансформированы в физическую болезнь.

Когда человек болен, обычно за ним ухаживают. Когда человек болен, он также может чувствовать себя также плохо, насколько он реально болен. Такая динамика особенно преобладает в семьях, в которых болезнь привлекает внимание и вознаграждается.

Конверсия чувств в болезнь это основа психосоматических болезней.

\ж{Проекция}

Проекция это один из самых примитивных защитных механизмов. Самым драматичным проявлением этого механизма является психопатический бред и галлюцинации. Когда мы повязаны стыдом, проекция неизбежна. Отщеплённые чувства, потребности и желания будут искать пути выражения.
Один из путей это приписывание их другим. Если я отрицаю свой гнев, я могу проецировать их на других и спросить, почему они злятся.
Проекция используется, когда провалилась попытка подавления. Это главный источник конфликтов и вражды в человеческих отношениях.

\ж{Вторичный защитные механизмы}

Фрейд описал также вторичные защитные механизмы. Они вступают в дело, когда перестают справляться первичные.

\ж{Торможение}

Один из моих клиентов был парализован при попытке танцевать в клубе. Во-время нашей работы, он вспомнил случай с танцем, когда ему было 12 лет.
Однажды его мать пришла домой пьяной (она была алкоголичкой). Она включила музыку и пригласила своего сына потанцевать с ней. Он был неуклюж и мать его за это пристыдила. Во время танца, у клиента случилась эрекция. Его мать заметила это и начала дразнить его.
Его неспособность танцевать это пример защитного механизма. С помощью торможения мышц он защищает себя от возможности заново пережить стыд.

\ж{Реактивное образование}

Реактивное образование используется для страховки того, что подавленные чувства, которые вызовут стыд, будут держаться подальше от сознания. Реактивное образование включается в действие, когда ослабевает подавление. Доброта часто развивается как противоположность импульсивной жестокости. Жестокость вызовет стыд, поэтому его противоположность - доброта - страхует человека от переживания стыда.

\ж{Отыгрывание}

Отыгрывание это волшебное поведение, нацеленное на отмену чувств, мыслей или действий, которые могут привести к стыду.
Один студент, которого я консультировал, тратил огромное количество времени на учебные ритуалы. Он тратил часы перед тем, как начать учиться, выкладывая свои книги, карандаши, ручки, тетрадки до достижения определенного сложного шаблона. Каждый предмет должен был быть выложен так, что бы он не касался никакого другого предмета.
Подобное поведение оказалось волшебным способом отыгрывания назад его желания трогать свой пенис и мастурбировать.

\ж{Изоляция аффекта}

Изоляция аффекта это способ конвертирования вызывающих стыд чувств или импульсов в мысли. Таким образом, человек может отречься от любой ответственности за эти чувства или импульсы.

\ж{Поворот против себя}

Поворот против себя или аутоагрессия это защитный механизм, в котором человек отклоняет агрессию от других людей и направляет её в свою сторону. Это часто происходит в случаях насилия, т.к. ребёнку отчаянно нужны родители для выживания, он направит свою агрессию с них на себя. Суицидальность - экстремальная форма этой защиты. В этом случае, человек настолько идентифицируется с насильником, что он решает убить его убивая себя.
Распространённые формы включают в себя откусывание ногтей, постукивания головой и членовредительство. В дальнейшей жизни люди могут вредить себе финансово или социально. Во всех случаях ярость в сторону насильника настолько страшна и постыдна, что направляется против себя.

В дальнейшей жизни, первичные защитные механизмы усложняются. Вторичные защитные механизмы также включают в себя рационализацию, минимизацию, сублимацию, компенсацию и другие типы поведения, которые позволяют переместить стыд на других.

Находясь на первой линии вашего самосохранения, эти механизмы защиты будут последними в излечении токсического стыда. Их сила и мощь лежит в их автоматичности и неосознанности. Они были лучшим решением для выживания в своё время и они помогли вам остаться в живых. Они буквально спасли вашу жизнь. Эти самые защитные механизмы, однажды бывшие жизненно важными, в данный момент стали хранителями нашего токсического стыда.

\ж{Ложное <<Я>>}

Я писал ранее о разрыве личности как о самом глубоком аспекте интернализированного стыда. Так как мы видим себя, как дефективных и ущербных, мы не можем смотреть на себя без боли. Поэтому, мы должны создать ложное <<Я>>. Ложной <<Я>> это второй слой защиты, созданный для облегчения чувства стыда.


Все главные школы терапии говорят о ложном <<Я>>. Юнгианцы называют её персоной (маска). В трансакционном анализе - адаптированный ребёнок.
Bob Subby говорит о публичном <<Я>> и приватном <<Я>>. Он использует пикрилейтед для иллюстрирования своей точки зрения. Маленькая фигурка, которая становится все меньше, это пристыженное аутентичное <<Я>>. Большая фигура - ложное <<Я>>.
Я разделяю ложное <<Я>> на три категории: Культурное ложное <<Я>>, Сценарий жизни и Роли семейных систем.

\ж{Культурное ложное <<Я>>}

В предыдущих главах я изложил наши культурные сексуальные роли, отмечая, как эти роли создают перфекционисткую систему мер. Так как каждый из нас является уникальной снежинкой, не существует способов сравнивания и измерения. Поэтому, жёсткие сексуальные роли пристыжают по своей сути.
Важно увидеть динамику того, как образуются сексуальные роли. Социологи описывают этот процесс как <<социальное конструирование реальности>>. Когда мы поймём этот процесс, то будет легче увидеть, как мы легко соглашаемся с этими ролями и делаем их нашим ложным <<Я>>.

\ж{Социальное конструирование реальности}

Каждый из нас рождён в обществе, пришедшем к консенсус по-поводу социальной реальности.
Мы, люди, действуем через повторения, необходимых для выживания и эти повторения становятся привычными. Это привычное поведение очень скоро становится социально приемлемым поведением. Со временем, социально приемлемые привычные действия становятся тем, что социологи называют <<узаконенными>>. После этого, узаконенное поведение становится бессознательным. Эта бессознательное узаконивание со временем превращается в законы реальности. Мы больше не подвергаем их сомнению. Мы принимаем их, они предсказуемы. Они обеспечивают нашу безопасность. Если кто-то попытается изменить их, мы сильно расстроимся.

Фактически, это не законы реальности. Разные культуры имеют свой жизненный уклад, свои правила реальности. На самом деле, узаконенные привычные действия это лишь реальность консенсуса.
Заметными частями каждой культурной реальности консенсуса являются понятие того, кто такой мужчина и женщина, что такое брак и семья.
Подобные стереотипные роли идеальны для пристыжения. Когда я играю роль настоящего мужчины, я получаю культурные вознаграждения. Эти стереотипные роли часто стыдят те части нашей аутентичной личности, которые не подходят под идеальную роль. В настоящее время, наши культурные сексуальные роли были узаконены во времена западной экспансии. Мужчины были охотниками и воинами, а женщины были с детьми.
Эти роли не только стыдят нас, но и становятся нашим убежищем для сокрытия. Притворяясь настоящим мужчиной или женщиной, мы можем скрыть факт того, что мы не знаем кто мы на самом деле. Мы можем изменять своё настроение играя роли, таким образом избегая болезненного чувства стыда.

\ж{Сценарий жизни}

Eric Berne, основатель и создатель Трансакционного анализа, развил понятие сценария жизни. Он наблюдал такой факт, что часть населения проводит довольно трагическую жизнь. Их жизнь трагично, потому что кажется, что у них просто не выбора. Они как актёры, играющие роли соответственно их сценарию. Берне чувствовал, что большинство людей живёт банальной и мелодраматической жизнью. Мелодраматический сценарий был описан Thoreau, когда он сказал, что большая часть населения живёт в молчаливом отчаянии. Берне думал, что немногие живут действительно аутентичной жизнью.

Сценарии жизни работают также как сценарии фильмов или пьес. Они описывают определённые типы характеров. Они предопределяют, что чувствовать и что не чувствовать; как себя вести или не вести. Трагические сценарии обычно кончаются убийством кого-либо или самоубийством; или жизнью медленного самоубийства; или сумасшествием.

Claude Steiner, терапевт ТА, говорит о трёх базовых сценариях: сценарий без мозгов (сумасшествие), без любви (убей себя или кого-то), без чувств (зависимость). Эти трагические сценарии созданы через пристыжение наших базовых стремлений: знать, любить и чувствовать.
Формирование сценариев сложное дело. Корневой механизм процесса возникает по средствам выбора, результирующего из запретов и предписаний, моделирования сценарием и жизненного опыта.

\ж{Запреты}

Запреты идут от пристыженного ребёнка в родителях. Запреты обычно невербальны. Они принимают форму сообщений: не будь, не будь мальчиком, не будь девочкой, не будь важным или успешным. Все токсические сценарии имеют запрет <<не будь собой>>.


\ж{Предписания}

Предписания более осознаны и обычно идут в вербальной форме. Сообщения в виде <<Как ты можешь быть таким тупым?>> или <<Для чего тебе мозги?>> создают сценарий <<без мозгов>>. Сообщения вида <<Ты ведь любишь своего брата, не так ли?>> или <<Мой сын не был бы таким гневным>> создают сценарий, в котором человек не знает своего собственного чувства любви. Сообщения вида <<Я знаю, что ты на самом деле не злишься>> или <<Нечего плакать об этом>> отвергают собственные чувства человека и приводят к замешательству.


Предписания могут прийти также, когда мама говорит по телефону со своими подругами и упоминает: <<Он мой послушный ребёнок>> или <<Она мой маленький озорник>>. Другие типы родительский предписаний: <<У тебя всегда будут проблемы с твоей учёбой, весом, гневом и т.п.>>, <<Ты всегда был эгоистом. Не повезёт твоей жене/мужу>>, <<Каждый член этой семьи был адвокатом>>, <<Ни одна женщина в нашей семье никогда не разводилась>>. Сценарные сообщения говорят нам, кто мы такие и какую роль должны играть в жизни. Они стыдят нашу аутентичность и создают разрушение личности.

\ж{Модели}

Я уже описывал динамику родительских моделей в интернализации стыда. Сценарные модели не ограничены родителями. Они могут прийти из сказок, фильмов, ТВ или других культурных или семейных моделей.
Женщины часто верят в волшебную сказку, что если они будут ждать достаточно долго, то к ним придёт принц.
Прячась за фантазиями вида ,<<в один день>>, <<если только>> и <<когда>> человек может прожить всю жизнь в ожидании. Фантазия связи создает волшебные переносы на другие фантазийные связи.


\ж{Жизненный опыт}

о, что происходит в семье, становится главным фактором формирования сценария. Если мама алкоголичка, ребёнок может взять на себя роль спасателя. Когда ребёнок получает внимание и похвалу за эту роль, она становится сильным фактором в формировании сценария спасателя. Другой ребёнок может получать много внимания во время болезни. Это может привести к сценарию с пожизненной болезненностью.

\ж{Роли семейных систем}

Все семьи имеют роли. Отец и мать играют роли моделей того, что значит быть мужчиной и женщиной, отцом и матерью. Родители также моделируют интимность, Эго границы, умение справляться с проблемами, разрешать споры и т.п. Роль детей быть любознательными и учиться. Члены здоровой семьи имеют гибкие роли. Мать может быть героиней, потому что испекла особый торт. Дочь может взять на себя эту роль, когда возьмёт на себя обязательство вымыть посуду. Сын становиться героем, когда замечает дым выходящий их печи и предотвращает пожар. Отец герой, когда берет всю семью на отдых.
В нашем центре для восстановления семей в Хьюстоне, мы обнаружили большое число ролей в дополнение к тем, о которых я писал ранее. Некоторые их них: Родитель родителя, Друг мамы или папы, Советник семьи, Отцовская звёздочка, Материнская звёздочка, Идеальный, Святой, Мошенник, Симпатичный, Атлет, Семейный миротворец, Семейный судья, Семейное жертвоприношение, Религиозный, Победитель, Неудачник, Мученик, Супер-мама, Супер-супруг, Клоун, Супер-папа, Гений, Мамин или папин козёл отпущения.

Мы предлагаем людям реально почувствовать свою роль дав ей имя. Вы можете понять, что играли несколько ролей. Каждая роль имеет свою чувственную подпись, и эта подпись останется с вами, даже если вы перестали исполнять эту роль.
Важно понимать, что когда мы играем какую-то роль, мы отрекаемся от нашей аутентичной настоящей личности. Роль это ложное <<Я>>. В дисфункциональных семья роль становятся потребностью семейной системы для баланса в ответ на стресс. Стрессом может быть алкоголизм отца, зависимость от таблеток или еды матери, насилие отца, инцест, религиозная зависимость матери и т.п. Каждая роль это попытка семейной системы справится со стыдом и стрессом. Каждая роль помогает каждому члену семьи почувствовать какой-то контроль. Чем чаще исполняется роль, тем жесте она становится. Стыд, который способствовал появлению роли, усиливается из-за этой роли.

Главная проблема в том, что роли не работают. Сила этих ролей для стыдливого человека в том, что они предсказуемы. Исполнение роли даёт ощущение идентичности. Поэтому так сложно отказаться от этих ролей.

Роли в дисфункциональных семьях приводят к потере связи с нашей реальностью. Со временем роли становятся бессознательными. Мы верим что роль и есть наша личность. Мы верим, что чувства, вызываемые этой ролью, действительно наши чувства.

\ж{Характерологические стили бесстыдства}

Третий слой защиты против чувства токсического стыда это бесстыдство. Это распространённый шаблон поведения для стыдливых родителей, учителей, политиков и проповедников. Бесстыдство включает в себя несколько видов поведений, которые служат для изменения чувства стыда и для передачи собственного токсического стыда другому человеку. Теоретики ТА называют это передачей горячей картошки. Эти поведения являются стратегиями защиты от боли токсического стыда. Они изменяют настроение и вызывают привыкание. Некоторые из поведений: перфекционизм, стремление к власти и контролю, ярость, высокомерие, критицизм, патронирование, помощь, зависть, угодничество. Каждое поведение фокусируется на другом человеке и ослабляет давление на самого себя.

\ж{Перфекционизм}

Перфекционизм произрастает из ядра токсического стыда человека без здоровых Эго границ. У перфекциониста нет чувства здорового стыда; у него нет внутреннего ощущения своих ограничений. Перфекционист никогда не знает предела хорошей работы.
Перфекционизму обучаются, когда человека ценят только за его действия. Когда родительская любовь и принятие зависит от исполнительности. Исполнительность всегда опирается на внешнее. ребёнок учиться всегда стремится во-внешнее. Ему не остаётся времени для отдыха и внутренней радости и удовлетворённости.
Перфекционизм всегда создает супер-мерку, по-которой человек сравнивает себя. И как бы тяжело он не трудился, он никогда не сможет соответствовать идеальной отметке. Не соответствие переводиться в сравнение хорошо \& плохо. <<Хорошо \& плохо>> ведёт к морализаторству и осуждениям.
Кауфман пишет: <<Когда перфекционизм принимает первостепенное значение, то сравнение себя с другими неизбежно кончается ощущением себя хуже других.>>
Сравнение себя с другими это один из главных способов, которым человек продолжает себя стыдить уже внутренне. Человек делает внутри себя то, то с ним было сделано снаружи. Осуждения и сравнения ведут к деструктивной конкуренции. Человек стремится обойти других и почувствовать себя лучше других, нежели просто быть лучшей версией себя. Стремление быть лучше других изменяет настроение и вызывает привыкание.

\ж{Стремление к власти и контролю}

Контроль это грандиозное расстройство воли, которое мы уже обсуждали. Тот, кто хочет контролировать все, боится быть уязвимым. Быть уязвимым значит быть открытым к пристыжению.
Всю свою жизнь я использовал свою энергию что бы быть всегда начеку. Это была пустая трата энергии. Страх был в том, что меня раскроют, а когда это случиться, все увидят мою дефективную и ущербную личность.
Контроль это способ удостовериться, что больше никто и никогда нас не пристыдит. Это включает в себя контроль наших мыслей, выражений, чувств и действий. Также это включает контроль других людей, их мыслей, чувств и действий. Контроль является главным злодеев уничтожающим интимность. Мы не можем свободно делиться чем-то, если мы не равны. Контроль другого человека разрушает равенство.
Стремление к власти идёт из стремления к контролю. Когда у человека больше власти, чем у других, он становится менее уязвим для стыда. Стремление к власти часто становится главной целью жизни. В своей самой невротической форме это становится зависимостью. Человек тратить всю свою энергию на планирование, игры и маневрирование ради получения лучшей позиции и подъёма по лестнице успеха.
Те, кто играет в эту игру, стремятся максимизировать свою власть. Они часто стремятся к властным должностям и закрепляют свою позицию нанимая в своё подчинение более слабых людей. Разделение власти для таких людей исключено.
Властные стратегии часто включают в себя использование власти для мести. Стыдливые родители делают со своими детьми то же самое, что сделали с ними их. Они повторяют свою виктимизацию на своих детях, в это раз в роли властной фигуры.

\ж{Ярость}

Ярость возможно самое естественно происходящее прикрытие для стыда. Она подходит достаточно близко к тому, что бы назвать её первичным защитным механизмом. Так и было бы, только не все дети выражают ярость. Некоторые дети выражают ярость в ответ на пристыжение; другие подавят её и иногда направят её в свою сторону.
Ярость защищает двумя путями: держит других подальше; передаёт стыд на других. Человек, придерживающий свою ярость, часто становится саркастичным. С таким человеком не очень приятно находиться рядом.
Несмотря на то, что выражение враждебности изначально было задумано как защита против дальнейшего переживания стыда, она стала также интернализирована. Ярость становится состоянием бытия, нежели чувством.
Интернализированная ярость разжигает глубокую горечь внутри себя. Горечь разрушает личность в негативном стремлении. Ярость часто усиливается до ненависти. Если человек с интернализированной яростью получает власть, это может привести к насилию, мести и криминалу.


\ж{Высокомерие}

Высокомерие определяется как оскорбительное преувеличение собственной важности. Высокомерный человек изменяет своё настроение через своё преувеличение. Жертвами высокомерия становятся те, кто ниже его по знаниям, опыту или власти.
Высокомерие это способ прикрыть свой стыд. После многих лет высокомерия, человек теряет контакт с самим собой.

\ж{Критицизм}

Критицизм и обвинения это самый распространённый способ передачи стыда. Если я чувствую себя подавлено и унижено, я могу ослабить это чувство критикуя и обвиняя кого-нибудь другого. Погружаясь в детали его ошибки, я выберусь из своего чувства стыда (изменю настроение).
Критицизм и обвинения также являются стратегиями защиты против переживания стыда. Они эффективно изменяют настроение поэтому со временем становятся аддиктивными. Дети, подвергшиеся критике и обвинениями, становятся повязанными стыдом. Дети не могут разобраться в защитном поведении своих родителей. Критика мамы воспринимается как утверждение что <<я плохой>>. Мама освобождается от своего стыда и сбрасывает его на ребёнка.

\ж{Морализаторство}

Морализаторство и склонность к осуждению это ответвления перфекционизма. Это попытка выиграть в духовном соперничестве. Осуждая других как плохих и греховных это способ почувствовать себя праведным. Это способно довольно сильно изменить настроение, поэтому часто становится аддиктивным поведением.
Когда человек использует перфекционизм и морализаторство для изменения своего настроения что бы избежать своего стыда, такое поведение называется бесстыдным. Дети, которые станут жертвами подобного поведения, должны будут перенять стыд бесстыдного родителя. Это не только эмоциональное насилие и убийственно для души, это также духовное насилие.


\ж{Презрение}

Презрение это интенсивное сознательное рассмотрение другого человека как отвратительного. В презрении личность другого человека полностью отвергается.
Родители, учителя и проповедники часто действуют бесстыдно в сторону детей, студентов и учеников. Когда важный опекун или учитель презирает другого человека, находящегося под его опекой, этот человек чувствует себя отвергнутым.
ребёнок учиться осуждать самого себя через интроекцию голоса опекуна и идентифицируясь с осуждающим. У ребёнка просто нет никаких способов защиты, поэтому идентификация позволяет ему почувствовать защищённость. ребёнок начинает осуждать других, также как осуждали его.

\ж{Патронирование}

Патронировать значит поддерживать, защищать и бороться за кого-то, кто неравен в правах, знаниях или власти; но при этом не получив согласия на поддержку и защиту. Это способ чувствовать себя выше кого-либо. Патронирование передаёт стыд на патронируемого. На поверхности кажется, что человек помогает другому человеку через поддержку и одобрение, но в реальности такая помощь ни чем не помогает. Человек чувствует себя пристыженным. Патронирование это прикрытие для стыда, и чаще всего скрывает осуждение и пассивно агрессивный гнев.


\ж{Помощь}

Странное дело, но помощь и уход за другим человеком часто усиливает его стыд. Помощник это распространённая роль в семейных системах. Помощники чаще всего ни чем не помогают другим, но они всегда помогают себе.
Человек, чувствующий себя ущербным и дефективным, также чувствует себя бессильным и беспомощным. Такой человек может изменить своё настроение и чувство беспомощности помогая другим. Когда он помогает другим, он чувствует себя лучше. Цель помощника это сам акт помощи, а не реальная польза для другого человека. Помощник пытается активно отвлечь себя от своих чувств.
Помощь и уход это защитная стратегия против токсического стыда которая ведёт к спасению и позволению. Опекающая супруга алкоголика на самом деле позволяет (разрешает(enabling)) алкоголизм, таким образом усиливая его токсический стыд. Родители часто позволяют или спасают своего ребёнка, делая для них то, что они могли бы сделать для себя. ребёнок приходит к тому, что чувствует себя недостаточным и дефективным. Спасение и позволение это воровство, оно ворует у человека его чувство достижения и силы, таким образом усиливая его токсический стыд.

\ж{Жополизы и <<хорошие>> парни и девчонки}

Жополизы и <<хорошие>> парни и девчонки тоже ведут себя бесстыдно и передают стыд другим. В своей книге Creative Aggression, George Bach и Herb Goldberg детально описали поведение жополизов и <<хороших>> людей.

Быть <<хорошим>> это официальное культурное прикрытие токсического стыда. <<Хороший>> человек скрывается за фасадом дружелюбного приятного человека.
Цель <<хорошего>> человека это его собственный образ, а не интересы другого человека. Подобное поведение в первую очередь способ манипулирования людьми и ситуациями, т.к. это ограждает человека от реальных эмоций и интимности. Избегая интимности человек защищает себя от возможности раскрытия своей сути - ущербного и дефективного человека.

Bach и Goldberg подводят итог подобного поведения. Это приводит к саморазрушению и непрямому пристыжению других.

<<Хороший>> парень:
\begin{enumerate}
\item Стремится создать атмосферу, где никто не может дать честную обратную связь. Это блокирует его эмоциональное развитие.
\item Подавляет рост других, т.к. он сам не даёт честную обратную связь. Это лишает других людей реального человека с которым можно взаимодействовать. Другие чувствуют вину и стыд за то, что чувствуют злость на этого <<хорошего>> парня. Они перенаправляют свою агрессию на себя, генерируя стыд.
\item <<Хорошее>> поведение нереально; это ставит ограничения на любые взаимоотношения.
\end{enumerate}


\ж{Зависть}

Самым распространённым определением зависти будет <<дискомфорт из-за превосходства и удачи других>>. Такой дискомфорт часто сопровождается вербальными выражениями принижения. Однако, выражение зависти может варьироваться от бушующего унижение до тонких инсинуаций. Последнее делает зависть такой загадочной.

Благодаря такой способности маскировки, зависть может принимать формы, которые невозможно сразу определить. Завистливый человек может скрывать свою зависть не только от других, но и от себя.

Я помню случай, когда я слушал публичное выступление одного человек, с которым сравнивали меня. Я был поражён силой и энергетикой его сообщения. Позднее, когда я упоминал о нем в разговоре с другими, я услышал свои слова <<мне очень понравилась его энергетика и сила... однако, я должен признать, меня удивило то, как часто он смотрел в свои записи>>. Если бы вы попросили меня под присягой признаться, завидовал ли я ему или нет, я бы поклялся, что нет, не завидовал.
Но фактически, я завидовал, и моё небольшое замечание было путём забрать свои позитивные слова о нем назад. Позднее, когда я честно подумал об этом спикере, я пришёл к выводу, что его манера речи была слишком драматичной и эгоцентричной. То, что мне не понравилось, была его самоуверенность. Это распространённый акцент, когда зависть приходит в форме умаление и принижения. Почти всегда, когда зависть пренебрежительна, это проекция нашей собственной самоуверенности.

Часто, после моих выступлений, ко мне подходили и говорили: <<Это была отличная лекция, но разве вы не взяли ваши основные идеи из такого-то места?>> Такая похвала на самом деле утверждение знаний другого человека. Такая самоуверенность также способна спровоцировать зависть в том, кому завидуют. Но завистливые отрицают самоуверенность, также как и саму зависть.

Кроме самоутверждения зависть может надевать маску восхищения или жадности. Кстати, один раз я поймал себя на восхищении человеком, которому я на самом деле завидовал. Анализируя это, я понял, что говорил противоположные вещи тому, что я на самом деле думал. Это высшая степень маскировки зависти, представлять себя в виде полной противоположности.

Leslie Farber: <<Истинное восхищение, которое свободно от сознательной воли, всегда молчаливо. Зависть в форме восхищения требует публики. Чем сильнее зависть, тем сильнее завистливый драматизирует себя как поклонника, чья страсть... стыдит других, более сдержанных.>>

Самая детская форма зависти это жадность. Когда я завидую кому-либо, я завидую чему-то, что у него есть: мудрость, смелость, харизма и т.п. Завистливый верит, что если у него будут эти качества, он будет в порядке. Зависть в форме жадности эксплуатируется современной индустрией рекламы, которая предлагает, что мы те, чем мы владеем.

В конечном счёте, самоутверждение, восхищение и жадность являются маскировкой зависти, которые она использует для прикрытия токсического стыда. Опасение превосходящих сил другого человека критично в оценке самого себя. Токсический стыд характеризуется в чувстве собственной неполноценности, боль от которой зависть избегает с помощью самоуверенного унижения. Восхищение может быть более пристыжающим, нежели критицизм.

Как говорит Farber: <<Это может вызвать зависть восхваляемого человека к образу, которым мы восхищаемся, т.к. в его сознании он видит огромную разницу между этим образом и его собственным представлением о себе.>>

Зависть как восхищение и самоуверенное принижение самоуверенности другого человека это способ межличностной передачи токсического стыда. Зависть как жадность основана на стыдливом убеждении, что я могу быть нормальным, только за счёт внешних вещей.

\ж{Компульсивное/аддиктивное поведение и повторение}

В Bradshaw On: The Family я представил спектр компульсивных/аддиктивных поведений, который предполагает, что существует намного больше зависимых, чем думают большинство людей. Мы часто ограничиваем эту область излишне концентрируясь на алкоголе и других наркотиках. Pia Mellody определяет зависимость как <<любой процесс, используемый для избегания невыносимой реальности>>. Так как это избавляет от невыносимой боли, это становится нашим главным приоритетом. Это настолько нам помогает, что начинает забирать время и энергию у других аспектов нашей жизни. Поэтому это имеет разрушающие жизнь последствия.

Быть повязанным стыдом значит быть в невыносимой боли. Физическая боль ужасна, но существует облегчение, есть надежда на излечение. Внутренний разрыв стыдом и <<скорбь>> по нашей аутентичной личности - хроническая. Она никогда не уходит. Не существует надежды на излечение, т.к. вы дефективны. Вы такой, какой есть. У вас нет взаимоотношений с самим собой или с другими. Вы полностью один. Вы в одиночной камере хронической скорби.

Вам нужно облегчение от этой невыносимой боли. Нужно что-то снаружи, что заберёт с собой это ужасное чувство о себе. Вам нужен изменяющий настроение опыт.

Существует неисчислимое количество различный способов изменения настроения. Любой из них потенциально аддиктивен. Если он каким-то образом забирает ваш ноющий дискомфорт, он станет вашим высшим приоритетом и самым важным взаимоотношением. Какой бы способ вы не выбрали, он станет приоритетным над всем остальным в вашей жизни. Также как и с физической болью, вы сделаете все, что бы остановить её.

Когда-нибудь была пульсирующая зубная боль? Вы не можете думать ни о чем другом. Вы становитесь центрированы на зубе. Если доктор выпишет рецепт на лекарство против боли, оно станет для вас важнее супруга, работы и семьи.

Процесс, убирающий боль, станет хроническим, как и сама боль. Хроническое состояние станет вредным для жизни и патологичным. Вы сделаете все, что бы сохранить этот способ снятия боли. Если кто-то попытается забрать у вас этот способ, вы пуститесь в любые уговоры и доказательства, что вам это нужно, что никакого вреда это не приносит. Вы будете верить, что это все хорошо, несмотря на факты опасности для жизни.

В такой форме, любой процесс снятие токсического стыда, станет аддиктивным. Если вы повязаны стыдом, вы станете зависимым. Зависимость формирует внешний слой защиты против токсического стыда.

Зависимость прячет стыд и усиливает его, но стыд ещё и питает зависимость. Более того, зависимость всегда семейное заболевание.


\ж{Зависимости с приёмом во внутрь}

Некоторые изменяющие настроение феномены изначально более аддиктивны чем другие, поэтому химические вещества и еда были основным фокусом компульсивных/аддиктивных поведений.

\ж{Алкоголь и другие наркотики}

Некоторые химикаты имеют изначальные аддиктивные свойства. Наркотик, такой как алкоголь, который действует на электрическую активность в лимбической системе (часть мозга контролирующая эмоциональную реакцию), имеет мощное аддиктивное свойство. Алкоголь также является поведенческим стимулятором, т.к. снижает подавление. Алкоголь это изменяющий сознание химическое вещество. Также он влияет на химию тела и питание при продолжительном приёме. Были выявлены чёткие и прогрессивные стадии зависимости и в данный момент они принимаются многими исследователями в этой области. В чем также согласны многие, это в том, что дети алкоголиков имеют в 5-9 раз больший шанс стать алкоголиком, чем дети не алкоголиков.

В моем случае, мой отец и его мать оба были алкоголиками. Я также верю, что сам являюсь генетически алкоголиком и у меня были проблемы с ним с первой выпивки. Моя первая потеря памяти была в 15 лет. Воспоминания о событиях стираются при достижении определенного порога толерантности. Потеря памяти это мощный сигнал о генетической предрасположенности к алкоголизму.

Кажется, будто генетические данные должны опровергнуть идею, что токсический стыд является ядром зависимости. Хотя я и не хочу сказать, что случаев чисто генетического алкоголизма не существует, я все же ни разу не видел подобного случая, хотя и нахожусь в сообществе излечения алкоголизма уже 22 года. Я консультировал около 500 алкоголиков и 4 года вёл Palmer Drug Abuse Program в Лос Анджелесе, был консультантом этой программы 10 лет до этого. За все эти года, я не видел ни одного человека, у которого бы не было проблем с заброшенностью и интернализированным стыдом. Думаю, это также правда со всеми другими депрессантами, такими как транквилизаторы и снотворное, также и для стимуляторов, галлюциногенов, никотина и кофеина.

\ж{Пищевые расстройства}

Пищевые расстройства и пищевые зависимости также являются комбинацией генетических факторов с проблемами в переживании эмоций. Врачи обычно делят пищевые расстройства на 4 категории: ожирение, анорексия, булимия и также то, что называется расстройство толстый/тонкий.

\ж{Пищевая зависимость: ожирение}

Fossum и Mason определяют ожирения как избыток веса в 7 кг. Чаще всего, используется мощная рационализация для оправдания ожирения и отрицания её опасных для жизни последствий: расстройство желез, наследственность, старение, образ жизни, беременность и большие кости. Нет сомнений, что есть определённая генетическая предрасположенность к ожирению, но никто не знает на сколько большую роль это играет. Моё обсуждение будет ограничено эмоциональным компонентом этой проблемы.

Jane Middelton-Moz, блестящий врач из Сиэтла, описывает истоки возможного пищевого расстройства сидя в аэропорту. Мать и отец ругаются. Их 18-ти месячный ребёнок лежит на сиденье рядом с ними. Они не обращали на него внимания. Каждый раз, когда ребёнок произносил что-то, мать направляла бутылку сока в его рот. Кто-то сел рядом с ребёнком и он вздрогнул и начал плакать. Мать посмотрела в свою сумку и нашла там другую бутылку с молоком и снова направила её в рот ребёнка. Оба, мать и отец, с избыточным весом в более чем 9 кг.

Этот ребёнок будет наблюдать модель переедания перед собой. Также он учится подавлять выражение своих эмоций и забивать чувства едой.

По крайней мере одна из динамик ожирения это результат само-потакания и злоупотребления шаблонами поведения для выживания, выученных в дисфункциональных семьях. Тучные люди были пристыжены из-за своего гнева или грусти. Они чувствуют себя пустыми и одинокими, поэтому едят, что быть полными и заполненными. Гнев проявляется в животе, еда и заполненность убирает чувство гнева обманывая человека, заставляя его поверить, что их тугой живот говорит о наполненности, нежели о гневе, который нужно выразить. Тучные люди часто ведут себя весело и счастливо, что бы прикрыть свой страх потенциального стыда, если они выразят свою глубокую скорбь или гнев.

По большей части диеты это величайшие обман когда либо проворачиваемый на страданиях группы людей. Большинство людей сидящих на диете набирают вес обратно в течении 5 лет. Диета подчёркивают один из самых парадоксальных аспектов токсического стыда. Диеты и похудение дают человеку чувство контроля и решения проблемы. Как вы видели ранее, контроль это одна из главных стратегий прикрытия стыда. Все слои прикрытия это попытки контроля внешнего, что бы внутреннее не было раскрыто.

Одним из моих ориентиров в определении это демона, которого я называю токсическим стыдом, была схема динамики контроля/освобождения в книге Facing Shame, Fossum и Mason.

Картина 3-3 это адаптация их работы. Контроль и освобождение это натуральные полярности в человеческой деятельности. Вы должны научить в детстве как держать и отпуска вещи развивая свои мышцы. В дальнейшем это принимает более сложные формы.

Когда стыд интернализирован, он становится токсичным и нарушает все границы и баланс. Вы становитесь грандиозным: или все или ничего. Вы или экстраординарны или вы червь. Или вы держите все под полным контролем (компульсивность) или у вас нет никакого контроля (зависимость). Они взаимосвязаны и дополняют друг-друга.

Fossum и Mason писали:
<<Когда стыд лежит в основе контроля и освобождения он усиливает обе стороны. Стыд делает динамику контроля более требовательной и неумолимой, а освобождение более динамичной и саморазрушающей. Чем интенсивнее человек контролирует, тем больше ему нужен баланс с освобождением и чем более саморазрушаемо человек убирает контроль, тем сильнее ему необходим контроль.>>

Диеты следуют этому циклы контроля и освобождения. Зависимость есть зависимость. Это слово буквально значит предать себя. Быть зависимым значит одержимо сдаться чему-то. Решением зависимому будет не пытаться контролировать зависимость. Решение это осознать свою беспомощность и неуправляемость и сдаться. Сдаться значит стать перед лицом факта, что вы не можете контролировать это. Поэтому это и есть зависимость.

\ж{Расстройство толстый/тонкий}

Многие пищевые зависимости визуально не видимы. Расстройство толстый/тонкий характеризуется постоянными мыслями о еде. Мысленная одержимость изменяет настроение. Таким образом человек, постоянно думая о том, что есть, а что нет, отвлекает себя от своих чувств.

Я лично ощущал это расстройство много лет. Я проходил через циклы тренировок, правильного питания, исключения сахара и потом, обычно после месяцев контроля, я съедал пончик и кусок пирога. Я делал это в разъездах, именно тогда, когда моё одиночество и уязвимости проявлялись ярче всего. Я вознаграждал себя за всю тяжёлую работу, что я сделал.

Как только я съем что-то сладкое, включается фаза освобождения. Я становлюсь одержимым и набиваю свою живот сладостями.

Это продолжает до того, как у меня образуется грудь. Теперь я снова сажусь на диету, тренируюсь и исключаю сахар. Это и есть компульсивное/аддиктивное поведение, никакого баланса, все или ничего.

\ж{Анорексия}

Анорексия определённо самое парадоксальное и опасное из всех пищевых расстройств. Более всего оно распространено в богатых семьях с дочерьми в возрасте от 13 до 25 лет. Это практически эпидемия в некоторых приватных школах.

Анорексия чаще всего идёт из богатых семей в которых доминирует перфекционизм. Богатые семьи часто фокусируются на актуализации своей самооценки. Респектабельность и высокий класс требуют поддержки определенного образа. Доминирует определённые шаблоны поведения: перфекционизм, не выражение эмоций, жёсткие и контролирующий отец и мать, полностью потерявшая связь со своими грустью и гневом, псевдо-интимный брак, сильный страх потерять контроль, исполнение детьми отцовских и материнских ролей. Эти факторы проявляются в различных комбинациях.

Анорексик берет на себя контроль над семьёй через своё голодание и потерю веса. Она метафора того, что не так с её семьёй. её жёстко контролируют, она отрицает все чувства, супер-достигатор. Она становится Прислугой или Козлом отпущения семейной системы. Мать и отец сближаются друг с другом в ответ на угрозу жизни дочери.

Зависимость обычно начинается с циклов объедания и голодания и сильной тягой к сладостям. Часто сопровождается чрезмерными тренировками и депрессией. Использование слабительных и принудительной рвоты обычно сопровождается голоданием. Все это позволяет мощно изменять настроение и сознание.

Анорексики драматически подчёркивают отказ быть человеком, что и лежит в сердце токсического стыда. Это включает в себя отвращение и отрицание собственного тела. Это отвращение становится физическое манифестацией их отказа от инстинктивной и эмоциональной жизни. Анорексики отказываются от своей сексуальности буквально отказываясь от развития женских половых признаков (менструация и развитие груди). Они отказываются от эмоция через отказ от еды. Для анорексика еда означает чувства. Так как все их чувства повязаны стыдом, отказ от еды это способ избегания чувства токсического стыда.

Также присутствует исполнение роли супруга и замешательство в границах между дочерью и матерью. Дочь часто держит в себе подавленный матерью гнев и грусть на счёт её отца. Эти чувства переполняю, т.к. они глубоко подавлены (вспомним голодных собак в подвале). Поэтому, голодание и избегание еды это защита против переживания этих подавляющих эмоций.

Анорексия это сложное заболевания и точно не утверждаю, что вышеописанное полностью создает клиническую картину. Моё желание в том, что бы вы увидели корни этой зависимости в изменении семейного стыда. Вера в то, что вы можете жить, в то же время отказывая от питательности пищи это высшее отрицание человечности. Это попытка быть более чем человеком.

\ж{Булимия}

Анорексики часто решают проблему своего голодания через цикл булимии объедания/чистки. Булимия также может развиться и без предшествующей анорексии. Булимия не ограничивается женщинами, т.к. многие мужчины становятся булимиками из-за зависимости от физических занятий. Для поддержания своего подтянутого тела многие мужчины начинаю вызывать у себя рвоту.

Kaufman видит булимию и анорексию как синдромы повязанности стыдом. Он видит токсический стыд в обоих циклах - объедания и чистки.

<<Когда человек чувствует себя пустым внутри, голодным до чувства принадлежности, отчаянно нуждающимся в близости и желании быть желанным, но эти нужды были повязаны стыдом, он направляет свой взор на еду.>>

Еда не может удовлетворить тоску, поэтому когда тоска превращается в стыд, человек ест ещё больше, для облегчения. Мета стыд, стыд за обжорство в секрете, это смещение эффекта, трансформирующего стыд о себе в стыд о еде. Такая же динамика присутствует в ожирении.

В цикле булимии, обжорство усиливает стыд, что запускает цикл рвоты, что добавляет отвращение и презрение к себе. Рвота это реакция отвращения. Отвратительная эмоциональная ситуация часто вызывает чувство тошноты. Сознательно рвота может быть действием для сброса веса, но подсознательно булимик обращается к рвоте как к способу очистки себя от стыдливого количества еды, которую они только что употребили.

Tomkins описывает булимическую рвоту как эффект увеличение. Увеличение чего-то приведёт к пиковой интенсивности, что позволит разрядиться. Рвота увеличивает унижение и отвращение к себе. Это подводит токсические стыд к пику и происходит эффект взрыва, в котором человек чувствует очищение.

Многие повязанные стыдом люди, кажется, имеют связь со своими эмоциями, т.к. довольно интенсивно их выражают. Но эмоциональный взрыв это лишь способ пережить чувства. Это мазохистская стратегия сокращения через увеличение. Увеличение может привести к двум результатам: взрыву или онемению.

\ж{Чувственные зависимости}

Существуют и другие пути изменения настроения без использования химических веществ или еды. Я уже описывал заменители чувств, которыми заменяются нежелательные чувства. Любая эмоция может быть аддиктивной. Самой распространённой аддикцией к эмоциям является форма усиленного гнева - ярость.

Ярость это единственная эмоция, которую не может контролировать стыд. На самом деле, ярость это лишь гнев, который был пристыжен. Гнев, как и сексуальность, это эмоциональная энергия сохранения. Гнев это чувство самосохранения. Наш гнев это энергия, с помощью которой мы защищаем себя. Наш гнев это наша сила. Когда гнев пристыжают, он становится той самой голодной собакой в подвале, которую невозможно сдержать. Пристыженный гнев становится яростью, которая вырывается из подвала для нашей защиты.


\ж{Зависимость к ярости}

Когда мы в ярости мы чувствуем себя объединённым внутри себя - больше нет никакой разделённости. Мы чувствуем мощь. Все съёживаются в нашем присутствии. Мы больше не чувствуем неадекватность и ущербность. Если нам за это ничего не будет, ярость становится нашим выбором в виде процесса, изменяющего настроения. Мы становимся зависимыми от ярости.

Я видел множество семей, разрушенных яростью. в одной из них мать была зависимой от ярости. Она тиранизировала свою семью, использую ярость для манипуляций. Отец нашёл союзника в виде старшей дочери (Суррогатная супруга). Дочь, травмированная этим эмоциональным инцестом, никогда не имела сексуальных взаимоотношений. Ей было 29 лет.

\ж{Зависимость от грусти, страха, волнения, религиозной праведности и радости}

Любая эмоция может быть аддиктивной. Большинство людей может представить человека, зависимого от меланхолии, тревожности или страха. Я часто сталкивают с зависимыми от радости. Они носят застывшую улыбку на лице. Они никогда не злятся. Смеются в неподходящие моменты и говорят только о счастливых и радостных вещах.

Чувство праведности одно из корневых среди зависимых от религии. Религиозную аддикцию очень сложно побороть, т.к. человек не может понять, что плохо в любви к Богу и служении человеческому роду.

Я вспоминаю дочь одного священника. Она была повязана стыдом до основания. Она думала о себе как о Вавилонской шлюхе. Она была покинута своим, также повязанным стыдом, самодовольным отцом. Он был занят спасением душ и исполнением роли мистера Великолепия, у него не было времени для неё. Я видел его на конференции через пару лет и он был все такой же напыщенный и пассивно агрессивный. Такие люди опасны. Они скрывают свой стыд покровительственной праведностью, что передаёт стыд их детям и ученикам.

Если вы боитесь своего гнева, тогда вы можете быть постоянно испытывать грусть. Грусть помогает изменить вашу ярость.

Также вы можете поверить в то, что жизнь это череда неожиданных возбуждений. Тогда вы научитесь постоянно искать новые и неожиданные способы достижения возбуждения (зависимость).

\ж{Зависимость к стыду}

Повязанные стыдом люди всегда зависимы от стыда. Это источник всех их мыслей и поведения. Все организовано вокруг идеи избегания стыда. Вы никогда не можете снять маску и защитные механизмы. Токсический стыд намного хуже чем голодные собаки в подвале. Это как стая акул-людоедов - вы не можете убрать контроль даже на секунду.

\ж{Зависимость от вины}

Вы также можете быть зависимы от токсической вины. Токсическая вина говорит о том, что у вас нет прав быть уникальным. Оставаться под контролем токсической вины значит постоянно анализировать себя. Жизнь это проблема, которую нужно решить, нежели тайна, которую нужно прожить. Токсическая вина заставляет вас постоянно работать над собой и анализировать каждое событие или взаимодействие. Нет времени для отдыха, потому что всегда полно дел. Вина также позволяет чувствовать себя сильным, когда вы на самом деле бессильны. <<Я разозлил свою мать>> или <<Я виноват в её болезни>> - утверждения грандиозности.

\ж{Мыслительные зависимости}

Мысли и умственная деятельность тоже потенциально аддиктивны. Мысленные процессы это часть любой зависимости. Умственная одержимость, прокручивания одних и тех же мыслей - это все часть аддиктивного цикла и это также аддиктивно само по себе. Я упоминал ранее защитный механизм под названием <<изоляция эффекта>>. Фокусируясь на повторяющейся мысли вы можете избежать болезненного чувства. Вы также можете избежать чувств размышлениями, прокруткой мыслей снова и снова. Вы можете быть зависимыми от абстрактного мышления.

Одна из моих степеней это философия. Я провёл годы моей жизни за изучением великих философов. Само по себе это не вредно. Для меня, чтение и преподавание философии было способом уйти от моих чувств. Когда я читал Summa Theologiae - Thomas Aquinas или Critique of Pure Reason - Emmanuel Kant, я мог полностью изменить свой токсический стыд.

Интеллектуализация это часто способ избегания внутренних состояний, которые повязаны стыдом. Универсализация и обобщение создает такие широкие и абстрактные рамки, что пропадает всякий контакт с конкретной сенсорной реальностью. Абстрактное обобщение это изумительные способ изменения настроения.

\ж{Детализация}

Детализация это ещё один мысленный процесс, который изменяет настроение.
На ум приходит пример моего клиента, которую считали скучной. Я попросил рассказать мне об этом. Она сказала:
<<Ну, когда я начала ходить на эти встречи, я хотела надеть моё синее шёлковое платье, но я забыла, что отнесла его в чистку. Я разочарована в наших хим-чистках. Раньше они хорошо справлялись с работой и цены были приятные. В действительности, они испортили две куртки моего сна Боби. Хотя, довольно сложно заставить Боби следить за своей одеждой. Он весь в отца. Они оба любят жить в комфорте, но при это оба неряшливы...>>

И вот таких деталей на 15 минут. Я остановил её и мягко указал ей на её одержимость деталями.

По мере нашего общения, я узнал, что её отец был совершенно сумасшедшим. Он держал её дома под дулом пистолета до 32 лет. Она выросла в маленьком городке западного Техаса, где её отец был шерифом. Он мог делать все что угодно. Он был жестоким и вербально оскорблял её. Его классической репликой была <<Женщины должны держать рот на замке, а ноги раздвинутыми.>>

Мой клиент была жертвой физического, эмоционального и сексуального насилия. Когда она убежала от него, она не переставал говорить. её болтовня стала способом избежать мучительного стыда и одиночества.

Умственная одержимость является распространённым элементом во всех созависимых взаимоотношениях.

\ж{Зависимость от активностей}

Другая форма изменения настроения это через поведение или деятельность. Я уже описывал ритуальное и магическое поведение, которые заключают в себе защитный механизм. Определённое обсессивно/компульсивное ритуализированное поведение имеет цель увести страх человека от определённых стыдливых желаний, чувств и импульсов.

Более распространённые формы деятельности, которые изменяют настроение это работа, покупки, накопительство, чтение, азартные игры, тренировки, наблюдение за спортом, смотрение ТВ, уход за животными. Ни одна из этих деятельностей не является зависимостью, если не имеет угрожающих жизни последствий. Но все они могут стать полноправными и угрожающими жизни зависимостями. Каждый из них это способ настолько вовлечься в дело, что у человека изменяется настроение.

Трудоголизм это серьёзная зависимость. Трудоголики, тратящие тысячи часов на работу, могут избегать болезненных чувств одиночества и депрессии. Я знаю об экспериментальном ретрите, который был проведён вместе с 10 руководителями компаний. Этих людей попросили избегать в течении 4-х дней всего, что может увести их от своих чувств: не читать, не пить, не курить, не смотреть ТВ, не разговаривать о делах, не использовать телефон, не тренироваться и т.п. На третий день, эту группа динамичных супер-достигаторов была вся в депрессии. Они были в контакте со своей пустотой и одиночеством. В большинстве случаев, у их детей были серьёзные проблемы с наркотиками и законом. Их дети часто тоже переживали одиночество и боль.

Такая же динамика правдива и для других занятий. Все они способы прикрытия одиночества и боли, выходящих из токсического стыда.

\ж{Зависимость воли}

Воля человека теряет своё взаимоотношение с интеллектом из-за <<загрязнения>>, которое стало результатом связывания эмоций стыдом.
Интеллектуальные операции восприятия, суждения и рассуждения имеют решающее значение для способности воли выбирать. Восприятие, суждение и рассуждения позволяют воле видеть альтернативы при выборе. При повязке стыдом эмоций, энергия замораживается, а интеллект становится предвзят и нарушен. Воля теряет свою способность виденья альтернатив; воля теряет свои глаза. Без глаз, воля слепа. Она больше не имеет данных, содержания вне себя. Содержанием становится сама воля, что приводит к желанию воли самой себя.

В моменты своеволия, человек перестаёт чувствовать себя расщеплённым. В эти моменты, человек становится един, силен и целен. Воля ради воли ведёт к угрожающему жизни своеволию. Своеволие ведёт к эгоцентризму, безумному контролю, драматическим крайностям и воле к нереальному. Своеволие не имеет границ.

Такая своевольность это корень всех зависимостей. Все зависимые в конечном счёте зависимы от своей воли. В АА это выражается как <<Я хочу что я хочу когда я хочу.>>

Зависимость от своей воли это способ, которым токсический стыд вызывает духовное банкротство. Поэтому так важно духовное излечение, когда дело касается излечения синдромов токсического стыда.

После такого изложения, вы можете подумать, не все ли мы страдаем зависимостью? Stanton Peele называл зависимость <<Опытом, объединяющим все человечество.>> Но если все зависимы/компульсивны и всё аддиктивно, тогда слово зависимость теряет всякий смысл. Мой ответ такой, что мы должны увидеть токсический стыд как топливо всех аддиктивных/компульсивных поведений. Не все используют вещества, чувства, мыслительные процессы, активности, потому что не все повязаны стыдом. Однако, когда я смотрю на нашу культуру, школы, религию и семейные системы, я убеждаюсь, что большое число людей действительно имеют токсический стыд.

Может быть, такая одержимость этой деталью тоже способ избегания стыда. В книге Bradshaw On: The Family я цитировал Satir и Wegscheider Cruse, которые предположили, что 96% семей дисфункциональны. Такая статистика не предназначена быть точной. Она призвана привлечь наше внимание; что бы удивить и напугать нас. Однако, статистика по зависимостям является довольно точной. Эта статистика, будь она о полиомиелите или оспе, была бы названа эпидемией.

\ж{Отыгрывание}

Ещё один способ прикрытия токсического стыда это через явление, называемое отыгрыванием. Примеры подобного поведения: повторное вхождение в разрушительные или пристыжающие взаимоотношения, которые повторяют раннюю травму; некоторые криминальные активности; делать с детьми тоже самое, что было сделано с вами, а также панические атаки.
Для понимания этих видов поведения, важно понять природу человеческих эмоций. Эмоции это энергия в движении. Они энергия, которая движет нами - наше топливо. Наши эмоции также являются красным датчиком уровня масла, который сигнализирует о потребности, потере или насыщенности. Наш гнев это наша сила; страх - проницательность; грусть - исцеляющее чувство; вина - формирует совесть; стыд сигнализирует о наших ограничениях, а также это источник духовности.

Когда эмоции пристыжены - они подавляются. Подавление включает в себя напряжение мышц и поверхностное дыхание. Группа мышц активизируется для блокировки энергии эмоций, которых мы стыдимся. Например грусть часто преобразуется в фальшивую улыбку.

Я часто улыбался, когда мне было грустно. Когда энергия блокирована, мы больше не чувствуем её. Однако, это все же форма энергии. Она динамична. Я уже приводил пример того, как усиливается подавленный гнев. В примере Вирджини Сатир гнев как голодные собаки в подвале, которые со временем вырываются, когда их уже невозможно сдерживать. Отыгрывание это выход эмоциональной энергии. Поведение, которое вызвало стыдливое событие, запускает суррогаты, которые отыгрывают оригинальную пристыжающую сцену.

\ж{Отыгрывание виктимизации}

Жертвы инцеста часто продолжают отыгрывать раннее сексуальное насилие в последующих взаимоотношениях. Быть осквернённым значит быть использованным и брошенным. Насилие часто трансформируется в понимание, что секс это единственный способ почувствовать себя желанной и стоящей. Я должна быть сексуальной, иначе я ничего не стою.

Линде было 30 лет. Она были бизнесвумен, владела агентством. её муж хотел, что бы она ушла с работы и они завели детей. Он вербально пристыжал её и угрожал разрушить её бизнес. Она ненавидела мужа, который требовал от неё оральный секс четыре раза в неделю. У Линды был роман на стороне. Когда я поговорил с её мужем, я быстро определил, что он был насильником. Он говорил о ней как об объекте. Он говорил, что не имеет значения чем она занимается, пока удовлетворяет его сексуальные потребности 4 раза в неделю и родит ему ребёнка.

Роман у Линды был с известным всем бабником. За время курса терапии она родила ребёнка, бросила свою карьеру и завела ещё несколько романов. В конце-концов она развелась и продолжила заводить романы один за другим. Каждый раз она выбирала богатых мужчин с историей распутства. Каждый раз её одаривали подарками, использовали как сексуальный объект и бросали. Каждый раз она отыгрывала сексуальную травму, пережитую в детстве от своего отца-алкоголика.

С 5 до 10 лет она регулярно делала минет своему отцу. Она была его любимой; он одаривал её подарками. Он был единственным, от кого она получала хоть какую-нибудь любовь. Линда была очень соблазнительна. Она надевала одежду, делающую её сексуально привлекательной для мужчин. Она пришла к реализации, что её соблазнительное поведение было способом отыграть свою стыд. Это был способ получить контроль над ситуацией, в которой ранее она была бессильна. Это был способ, разрешить её неразрешённую скорбь.

В каждом случае, когда мужчина бросал её, она могла чувствовать то, что ей было запрещено чувствовать в оригинальной ситуации. Она боялась заброшенности; она бы плакала; она была бы в ярости на него. Каждое отыгрывание для неё было несостоявшейся попыткой отыгрывание чувств, от которых она была диссоциирована. В её нормальном диссоциированном состоянии, она была не в контакте с чувствами. Она часто чувствовала себя сумасшедшей. Отыгрывая чувства она чувствовала себя более-менее в своём уме.

Навязчивое повторение, стремление к повторению, упоминается Alice Miller как <<логика абсурда>>. Это случается в менее жестокой форме со многими.

Я знаю мужчину, который рос с эмоционально недоступной матерью. Он был женат 4 раза, каждый раз на эмоционально недоступной женщине.

\ж{Преступное поведение}

В своей книге о преступности Alice Miller представила дело Jurgen Bartsch, который был серийным убийцей. Он убил четырёх мальчиков в период 1962-1966 годов. С некоторыми незначительными отклонениями его методы работы были одинаков. После того, как он заманивал мальчика в бывшее бомбоубежище недалеко от его дома, он избивал ребёнка, связывал его бечёвкой, мастурбировал, трогая его гениталии, далее убивал ребёнка при помощи удушения или ударов, разрезал тело, опустошал живот и хоронил останки. Bartsch засвидетельствовал, что достигал оргазма, когда резал труп.

Вот что написала Alice Miller: <<Каждое преступление содержит скрытую историю, которая может быть расшифрована опираясь на способы совершения преступления и конкретные детали.>>

Bartsch был сиротой и был усыновлён родителями, которые долго и аккуратно искали идеального ребёнка. Jurgen тратил часы в ритуальном поиске подходящего мальчика для убийства. Jurgen регулярно избивали в детстве. Часто его находили в синяках и ссадинах. Его избивали в той же комнате, в которой его отец разделывал туши. Позднее, его закрывали в старом погребе. Так продолжалось 6 лет. Ему запрещалось играть с другими детьми. Он пережил сексуальное насилие со стороны матери, т.к. она купала его до 12 лет, трогая его гениталии. В возрасте 8 лет его соблазнил его 13-ти летний
двоюродный брат, а позднее, в 13 лет, учитель. Его преступления содержат в себе импринты каждой детали его жизни. Он отыгрывал свою неудовлетворённую ненависть на маленьких мальчиках. Он разделывал их разделочным ножом, как делал его отец, пока его избивала и хлестала мать. Она часто целовала его в губы после избиения. Jurgen тоже целовал своих жертв.

Jurgen был жертвой, ставший преступником. Он пробуждает наше возмущение и ужас. <<Но>>, пишет Alice Miller, <<ужас должен быть направлен в сторону первого убийства, которое было совершено в секрете и прошло безнаказанным.>>

Когда над ребёнком совершается насилие, его нормальная реакция это кричать в гневе и боли. Гнев запрещён, потому что принесёт ещё больше наказаний. Выражение боли также запрещено. ребёнок подавил это чувства, идентифицировался с агрессором и подавил воспоминания о травме. Позднее, отделённый от оригинальной причины и чувства гнева, беспомощности, замешательства и боли, он отыгрывает эти мощные чувства против других в преступном поведении, или против себя в наркотической зависимости, проституции, психическом расстройстве или суициде.

Alice Miller пишет <<Тому, кому было не разрешено осознавать то, что с ним произошло, не имеет другого способы рассказать об этом, кроме как снова повторить это.>>

Все родители, не проработавшие свои травмы, будут отыгрывать их на своих детях.

\ж{Панические атаки}

Компульсия повторения ясно показывает, что поведение людей имеет смысл, каким бы странным и бесчеловечным оно ни было. Каждая не оплаканная и не завершённая деталь поведения будет отыграна снова. Это можно также увидеть в панических атаках.

Jane Middelton-Moz называет панические атаки <<окном в мир напуганного ребёнка>>. Она описывает два примера панических атак как отыгрывания ранних детских травм.

В одном случае женщина впадает в панику, когда её сердцебиение повышается до определенного значения во время пробежки. Она пробовала множество способов контроля этих атак. Jane Middelton-Moz смогла помочь ей найти неразрешённую детскую травму, которая и была источником панических атак.

ребёнком она жила в бедном районе. Однажды она играла со своим братом на улице. Это было рискованно, т.к. по району бродили различные банды. Одна банда погналась за ними. Она вспомнила, как смогла убежать от них, но её брата они догнали и убили.

Прошлое ужасное и неразрешённое горе запускалось и отыгрывалось каждый раз, когда она бегала и достигала такого же состояния как в том трагическом происшествии.

Jane Middelton-Moz рассказывает о другой женщине, с которой развелись все три мужа из-за её безумной ревности. Обычно она была не ревнивым человеком. её атаки происходили в случаях разделения. Если один из её мужей должен был идти на ночную смену или уехать в командировке, у неё случалась паническая атака. Когда её муж возвращался домой, она впадала в истерическую тираду. На следующий день она раскаивалась, понимая, что реагировала совершенно неадекватно. Jane Middelton-Moz помогла ей найти связь панического чувства с эпизодом из прошлого.

Она жила в городе на среднем западе. её отец работал на противоположном от их дома берегу реки. Один раз, когда он ушёл на работу, случился шторм, а далее паводок. Отец целую неделю не мог вернуться домой. её мать была алкоголичкой, а отец был её поставщиком. В шестилетнем возрасте эта девочка и её 4-х летний брат в течении недели переживали делирий их матери. Каждый раз, когда один из её мужей собирался уходить, её неразрешённая травма активировалась.

В этой части вы могли увидеть, какими угрожающими жизни последствиями может обернуться интернализированный стыд. Его сила лежит в тьме и секретности. Обличая этого демона, мы можем начать выстраивать образовательный и терапевтический подход для предотвращения интернализации стыда в семьях.

\ж{Часть 2 - Решение}

\ж{Процесс экстернализации}


Для излечения токсического стыда мы должны выйти из укрытия. До тех пор, пока наш стыд скрыт, мы ничего с ним сделать не сможем. Для изменения токсического стыда мы в первую очередь должны сначала принять его. Старая терапевтическая пословица гласит, что <<единственный выход - пройти через это.>>

Принятие нашего стыда включает в себя ощущение боли. Боль это то, что мы пытаемся избежать. На самом деле, большинство наших невротических поведений это избегание боли. Мы пытаемся найти лёгкий путь и это вполне разумно. Однако, как говорит Scott Peck, <<Склонность к избеганию эмоционального страдания... это основа всех психических заболеваний человека.>>

В случае стыда, чем больше мы избегаем его, тем сильнее он становится. Мы не можем изменить наш <<интернализированный>> стыд до тех пор, пока мы не <<экстернализируем>> его. Работа по ослаблению стыда простая, но тяжёлая.

Методы экстернализации:
\begin{enumerate}
\item Выход из укрытия через социальный контакт, что значит честно делиться своими чувствами с близкими людьми.
\item Увидеть наше отражение в <<глазах>> хотя бы одного не пристыжающего человека, который является частью нашей новой семьи. Это восстанавливает межличностную связь.
\item Работа в программе 12 ступеней.
\item Делать работу над легитимизацией нашей травмы заброшенности. Мы делаем это через письмо и рассказ об этом (Разбор). Письмо в особенности помогает в экстернализации прошлых стыдливых переживаний. Таким образом мы экстернализируем наши чувства об этой травме. Мы можем выражать их, прояснять и воссоединится с ними.
\item Экстернализировать нашего внутреннего ребёнка. Мы делаем это через сознательный контакт с нашей уязвимой детской частью. 
\item Научиться распознавать различные отщеплённые части личности. Сознательное распознавание этих частей (их экстернализация) позволяет интегрировать их.
\item Принять решение в безусловном позитивном принятии всех наших частей. Научиться говорить <<Я люблю себя за...>> Научиться экстернализировать наши потребности и желания став более самоуверенным.
\item Экстернализировать старые бессознательные воспоминания из прошлого, который формируют коллаж стыдливых сцен и научиться излечивать эти воспоминания.
\item Делать упражнения для экстернализации нашей самооценки и её изменения.
\item Экстернализировать голоса в нашей голове. Эти голоса держат нашу спираль стыда в активном состоянии. Делать упражнения, которые останавливают стыдливые голоса и научиться заменять их новыми позитивными голосами.
\item Быть в курсе межличностных ситуаций, которые могут запустить стыдливую спираль. 
\item Научиться справляться с критичными и пристыжающими людьми через практику техник настойчивости и создание якорей экстернализации стыда.
\item Научиться справляться с нашими ошибками, и найти смелость, что бы быть несовершенным.
\item И наконец, через молитвы или медитации, создать пространство тишины и спокойствия внутри себя, где мы центрированы и заземлены на нашей личной высшей силе.
\end{enumerate}

Все эти методы экстернализации были адаптированы из основных школ терапии. Большинство терапий пытается сделать то, что скрыто и неосознанно, открытым и осознанным.

Эти техники могут быть освоены только через практику. Вы должны их практиковать, снова и снова. Они будут работать, если вы будете работать.

\ж{Притча: узник в тёмной пещере}

Жил когда-то человек, которого приговорили к смерти. Ему связали глаза и бросили в кромешно тёмную пещеру. Пещера была 100 на 100 метров. Ему сказали, что существует выход из пещеры, и если он сможет его найти - он свободен.

После закрытия входа в пещеру огромным камнем, ему разрешили снять повязку. Его будут кормить только водой и хлебом в первые 30 дней, а после он ничего не получит. Хлеб и воду спускали через небольшую дыру в крыше на южной стороне пещеры. Крыша была в 6 метрах над землёй. Дыра - примерно 30 сантиметров в диаметре. Заключённый мог видеть слабый свет над своей головой, но в саму пещеру свет не попадал.

Заключённый бродил и ползал по пещере и наткнулся на камни. Некоторые были довольно большими. Он подумал, что если он сможет построить холм из камней достаточной высоты, то сможет дотянуться до дыры в крыше и сбежать. Так как его рост был 1,8 метра и он мог подпрыгнуть ещё на пол метра, то холм должен быть высотой близкой к 4 метрам.

Так, часы бодрствования заключённый проводил в подъёме камней и копании грязи. По истечению двух недель, он построил холм высотой в 2 метра. Он подумал, что если сможет удвоить это достижение в следующие две недели, то сможет сбежать до того, как у него закончится еда. Но т.к. он уже использовал большинство камней в пещере, ему пришлось копать все больше и больше. Копал он голыми руками. прошёл ещё один месяц и холм достиг высоты в 3 метра и он уже почти мог дотянуться до дыры в крыше. Он был изнеможён и слаб.

В один день, когда он думал, что может уже достать до проёма, он упал. Он просто был слишком слаб, даже что бы подняться на ноги. Через два дня он умер. Его тюремщики пришли забрать тело. Они откатили огромный камень, закрывающий вход. Свет, попавший в пещеру, осветил отверстие в стене, примерно метр в окружности.

Отверстие оказалось туннелем, который вёл на другую сторону горы. Это и был выход на свободу, о котором говорили заключённому. Проем был на южной стене пещеры, прямо под дырой в крыше. Все, что должен был сделать заключённый, это проползти 70 метров и он оказался бы на свободе. Он был так сосредоточен на светлом проёме в крыше, что ему совершенно не пришло в голову, искать выход в тьме пещеры. Освобождение все это время находилось рядом с холмом, который он строил, но оно было во тьме.

\ж{Выход из укрытия и изоляции}

Мучительное одиночество, которому способствует токсический стыд, унижает человеческое достоинство. Все большая изоляция человека приводит к потере преимуществ человеческой обратной связи. Человек теряет зеркалирующие глаза других людей. Erik Erikson ясно продемонстрировал, что формирование идентичность это всегда социальный процесс. Он определяет идентичность как <<внутренней чувство сходства и преемственности, которое согласовано зеркалированием хотя бы одного близкого человека.>> Вспомните, неудовлетворительное зеркалирование наших близких стало причиной токсического стыда.

Что бы излечиться, мы должны выйти из укрытия и изоляции. Это значит, найти группу близких людей, которым мы готовы доверять. Это невероятно сложная задача для стыдливых людей.

Я помню, как отчаянно искал гипнотизёра, когда мне посоветовали вступить в 12-ступенчатую программу. Я был в ужасе от идеи того, что бы раскрыться перед другими людьми.

Стыд становится токсическим из-за преждевременного раскрытия. Мы раскрыты или неожиданно, или перед тем, как были готовы к раскрытию. Мы чувствуем себя беспомощными и бессильными. Не удивительно, что мы боимся пристального взгляда других людей. Однако, единственный способ выхода из токсического стыда это его принятие... мы должны выйти из укрытия.

\ж{Поиск социального круга}

Лучший способ выхода из укрытия это найти не пристыжающую интимную группу людей. Главное слово тут интимная. Мы должны выйти на корневой уровень, потому что стыд находиться именно там. Токсический стыд маскирует наши самые глубокие секреты; он олицетворяет убеждение, что мы по своей сути дефективны. Это такое ужасное чувство, что мы не смеем посмотреть на самих себя, не говоря уже о том, что бы рассказать об этом кому-либо ещё. Единственный способ узнать, что мы были не правы в осуждении самих себя, это рискнуть раскрытием перед пристальным взглядом других людей. Когда мы доверяем кому-либо и чувствуем их любовь и принятие, мы можем начать менять свои убеждения о самих себе. Мы учимся тому, что мы не плохие; тому, что любимы и нас принимают.

Истинная любовь лечит и влияет на духовный рост. Если мы не растём благодаря чьей-то любви, то скорее всего это из-за поддельной формы любви. Истинная любовь безусловна в своём позитивном отношении. Безусловное позитивное отношение позволяет нам быть цельными и принять все наши части личности. Быть цельным значит воссоединить все пристыженные и отщеплённые части.

Вирджиния Сатир говорила о 5 свободах, которые проявляются, когда человек испытывает безусловную любовь. Эти свободы включают в себя наши основные способности. Это способность воспринимать; любить (выбирать и хотеть); выражать эмоции; думать и выражать себя и способность воображать.

Когда мы цельны и полностью принимаем себя, мы имеем свободу видеть и слышать то, что мы видим и слышим, нежели то, что мы должны или не должны видеть и слышать; свободны думать и выражать то, что мы думаем, нежели то, что мы должны или не должны думать и выражать; свободу любить (выбирать или хотеть) то, что мы хотим, нежели то, что мы должны или не должны любить; свободу воображать то, что воображаем, нежели то, что должны или не должны воображать. Когда мы испытываем безусловную любовь, например, когда нас принимают таким, какой мы есть, тогда мы сами можем принимать себя таким, каким мы являемся.

Самопринятие преодолевает саморазрушение токсического стыда. Самопринятие эквивалентно личной силе. Самопринятие значит, что мы едины; вся наша энергия центрирована и течёт наружу. Не принятие себя создает разрушение и внутреннюю борьбу. Когда мы не принимаем себя мы становимся в стороне от себя и используем свою энергию на сокрытие самого себя от себя. Как результат, у нас недостаёт энергии для взаимодействия с внешним миром. Самопринятие делает нас максимально функциональным.

Так как личные взаимоотношения стали источников токсического стыда, нам нужны эти взаимоотношения и для лечения стыда. Это имеет решающее значение. Мы обязаны рискнуть и обратиться к не пристыжающим взаимоотношениям для излечения стыда. Другого пути нет. Когда мы вступим в диалог и будем находиться в сообществе, нам предстоит дальнейшая работа. Но мы не можем начать эту работу до установления этих отношений.

12-ступенчатая программа имела большой успех в лечении стыдливых людей. Помните, что токсический стыд является корнем всех зависимостей. 12-ступенчатая программа была рождена двумя смелыми людьми, которые рискнули выйти из укрытия. Один алкоголик (Bill W.) обратился к другому алкоголику (Dr. Bob) и они рассказали друг-другу, как плохо они о себе думают. Эта программа всегда работает в контексте группы.

Так как мы по своей сути социальные создания, мы не можем жить счастливо и полноценно без социального контекста. Другими словами, нам, людям, требует любить и быть любимыми, нуждаться и быть нужными. Это основа. Мы не можем быть полноценным человеком без удовлетворения этих потребностей.

Для излечения нашего стыда, мы должны рискнуть и войти в группу. Мы должны быть готовы раскрыть нашу сущность. АА учат алкоголиков представить себя словами - <<Меня зовут Х. Я алкоголик.>> Идентификация корневой проблемы имеет важнейшее значение для восстановления. Вербальное заявление становится принятием беспомощности и неуправляемости. Это является индикатором того, что человек принял свой стыд и сдался ему.

Эта капитуляция является ядром духовного парадокса, который говорит о том, что бы можем выиграть только когда сдадимся. Что бы снова обрести жизнь, мы должны потерять её. Мы должны отказаться от иллюзорных ложных <<Я>> и защитных механизмов, что бы найти наше драгоценное ядро личности. В нашем невротическом стыде лежит уязвимое и чувствительное <<Я>>. Мы должны принять тьму, что бы найти свет. Скрытым, в резервуаре токсического стыда, живёт наше истинное <<Я>>.

Нет жизни без смерти, звука без тишины и света без тьмы. Притча в начале этой части является адаптацией истории, рассказываемой в терапевтических кругах. История типичного человеческого стремления. Мы всегда ищем решение в очевидных местах. Нам не приходит в голову, что есть другой путь, через тьму. И это единственный путь в вопросе токсического стыда.

\ж{Руководство по выбору группы}

Определённо, существуют и другие интимные группы. Их можно найти в церкви/синагоге или в психотерапевтических группах.

Решающее значение в поиске группы имеют несколько вещей:

\begin{itemize}
\item Группа должна быть беспристрастной и не пристыжающей. Знайте, что вы всегда можете покинуть группу, если чувствуете излишнее раскрытие.
\item Группа должна быть демократичной и свободной. Каждый человек в ней может быть самим собой. Это то, что никогда не испытывал пристыженный человек.
\item Лидер группы должен быть моделью здорового стыда. Это значит, что он не должен действовать бесстыдно (контроль, перфекционизм, жёсткость). Лидер будет человеком, который прошёл дальше всех в своём лечении и может подсказать группе, что ждёт их дальше.
\item Большинству пристыженных людей нужна группа, в которой практикуют объятия и прикосновения в уважительной манере. Это значит, что никто не обнимает вас без вашего на то разрешения. Вы также можете отказаться от объятий.
\item Большинство из нас были пристыжены в довербальной стадии развития, когда нам недоставало прикосновений и объятий. До развития вербальной коммуникации, межличностная связь устанавливается по средству поглаживаний и объятий.
\item Также группа должна позволять полное выражение эмоций. Члены группы должны иметь возможность открыто и свободно выражать свои эмоции. Стыд <<замораживает>> все эмоции, и свободное их выражение становится процессом оттаивания.
\end{itemize}

Мой совет это медленно учиться определять и выражать свои эмоции. Это то, чему мы не научились из-за ядовитых правил и плохого моделирования в дисфункциональных семьях. По-началу эмоции могут напугать вас. Мы боимся быть переполненными эмоциями. В начале пути, просто ощущение своих эмоций ослабляет токсический стыд. Обмен эмоциями с другими значит быть уязвимым. Это процесс экстернализации и выхода из укрытия.

Robert Firestone в своей книге The Fantasy Bond заявил, что мы не функционируем полностью как человеческие существа без дружбы и жизни в сообществе. Противоположно этому жизнь в выдуманной иллюзорной связи. Все зависимости это выдуманные связи, который создают жизнь внутренней отстранённости и потакания собственным слабостям. Такая жизнь бесчеловечна. Только живя в диалоге с сообществом мы по-настоящему живём и растём над собой.

\ж{12 шагов для трансформации токсического стыда в здоровый}

Я обязан жизнью 12-ступенчатой программе, поэтому, я не могу быть беспристрастным относительно способности этой программы в излечении токсического стыда. Никто не спорит с тем, что 12-ступенчатая программа имеет довольно успешный опыт в работе с зависимостями. Я опишу то, как я понимаю работу шагов относительно излечения токсического стыда. 

\ж{Шаг 1} говорит: <<Мы признаем, что были бессильны перед (называние зависимости) и наша жизнь стала неуправляемой.>> Этот шаг подтверждает самый мощный аспект любого синдрома стыда - его функциональную автономию.

Перед лицом токсического стыда мы бессильны. Все излечившиеся подходили к поворотной точке в своей жизни, к которой их привела боль, вызываемая их зависимостью.

Боль заставила меня осознать мою беспомощность и неуправляемость. Единственный выходом из этой боли было раскрыться - я должен был сдаться. Я должен был принять мой стыд и боль. В моем случае, боль стала такой мучительной, что я был готов идти до конца. Принятие моей боли привело к раскрытию боли, печали, одиночества и стыда. Это то, чего я так долго боялся сделать. Когда я признался в том, как на самом деле я себя чувствовал, я увидел принятие и любовь в глазах других. Когда они приняли меня, я начал чувствовать что я имею значение. Я начал принимать самого себя. Межличностная связь была восстановлена.

\ж{Второй шаг} просит нас обратиться к чему-то большему, чем мы сами. Он гласит: <<Мы пришли к убеждению, что только Сила, более могущественная, чем мы сами, может вернуть нам здравомыслие.>>

Ранее я говорил о Падении в Бытие. В Бытие говориться, что четыре вида взаимоотношений были разрушены токсическим стыдом Адама: отношения с Богом; отношения с собой; отношения с братом и ближним; отношения с миром (природой). 12 ступеней восстанавливают эти взаимоотношения. Второй шаг начинается с принятия чего-то высшего. 

\ж{Третья ступень} говорит <<Мы приняли решение препоручить нашу волю и наши жизни заботе Бога, как мы Его понимали.>> Хотя здесь Бог упоминается как Высшая Сила, каждый человек сам решает, что он понимает под этим понятием. Программа не навязывает какое-либо понимание Бога своим участникам.

Восстановление взаимной связи с Богом имеет огромную силу в лечении токсического стыда. Токсический стыд это расстройство воли, которая становится грандиозной. Завязшие в своём стыде люди начинают скрывать свои ошибки через перфекционизм, контроль, обвинения, презрение и т.п. Они действуют бесстыдно, они играют роль Бога. Эта грандиозная игра в Бога является духовной катастрофой. Шаги 2 и 3 воссоединяют необходимую связь с Высшей Силой.

Стыдливые люди также не верят, что они имеют право зависеть от кого-либо. Это последствие нарушенной потребности в зависимости через травму заброшенности. Отдать свою волю и жизнь Богу значит восстановить право на зависимые отношения. Пойти на встречу и довериться другим людям значит рискнуть зависимостью от них.

Здоровый стыд это разрешение быть человеком - иметь ограниченную силу, совершать ошибки. Стыд говорит нам о том, что мы не Боги и нам нужна помощь.

Первые три шага восстанавливают правильные взаимоотношения между нами и источником жизни. Признание беспомощности и неуправляемости, вера в высшую силу, которая восстановит наше здравомыслие, принятие решения отдать контроль Богу, так как мы его понимаем, восстанавливает наш здоровый стыд. Мы сдаёмся в своём бесстыдстве, грандиозном контроле и игре в Бога.

С помощью первых трёх шагов программы, мы снова вступаем в ряды человеческой расы; мы принимаем нашу потребность в сообществе и ограничения нашей человеческой реальности. Scott Peck однажды определил эмоциональную болезнь как избегание реальность всеми возможными способами, а здоровье как полное принятие реальности. Первые шаги восстанавливают здоровое принятие реальности.


\ж{Шаг 4} гласит <<Мы сделали тщательную и бесстрашную моральную инвентаризацию самих себя.>>
В этом шаге мы восстанавливаем взаимоотношение с самим собой и своим ближним. Наши стыдливые защитные механизмы скрывали нас от других. Ещё более трагично то, что эти механизмы скрывали нас самих от себя.

Я не писал свою инвентаризацию четвёртого шага в течении двух лет после вступления в программу. Это не хорошо и не плохо. Процесс продвижения по шагам индивидуален.

Большинство 12-ступенчатых программ настоятельно советуют, что бы новые члены нашли спонсора. Спонсор это человек, который имеет какую-то степень здоровых эмоций и продолжает работать над собой. Спонсор служит моделью и предлагает свою помощь в работе над шагами.

В моем собственном случае инвентаризация должна была подождать, т.к. я застрял на первых трёх шагах. Моя интеллектуализация была сильна. Я был профессором и имел степени в психологии, философии и теологии. Я учил всем этим вещам на университетском уровне. Я сопротивлялся простоте 12-ступенчатой программы. Часть моего фасада была игрой в чувствительного человека, который понимает всю сложность человеческого страдания. Я пил, потому что носил на своих плечах осознание всего человеческого страдания. Как я и говорил ранее, все это было ерундой. Это был тонкий способ поддержания иллюзии и отрицания.

Один из самых важных уроков в моей жизни преподал Abraham Low, основатель Recovery, Inc. Он сказал, что интеллектуализация о своих проблемах это сложно, но легко, в то время как сделать что-то на счёт них тяжело, но просто. Стыдливые интеллектуалы любят дискутировать и усложнять.

Когда я все же выполнил четвёртый шаг, я понял, что большинство моих проступков были результатом моего питья и страха. Я увидел, что корнем всех моих проблем было чувство недостаточности. В то время я ещё не понимал, что такое стыд, но именно он был корнем всех проблем.

Я понял во-время своей инвентаризации, что корневая проблема была моральной, нежели аморальной. На самом деле, моей первой попыткой инвентаризации был длинный список всех моих аморальных поступков. Мой спонсор помог увидеть, что я участвовал в повторяющихся моральных провалах. В моей грандиозности я был или суперменом, или недочеловеком. Я пытался быть более чем человеком (бесстыдным) или менее чем человеком (полным стыда).


Для того, что бы быть моральным, человек в первую очередь должен иметь здоровую волю. Моральные действия требуют суждения, разума и возможности выбирать. Я думаю, стыдливые люди находятся в доморальной зоне из-за нарушенной воли. Это не значит, что реальные вред не был принесён другим людям. В 4 шаге я взял на себя ответственность за все свои проступки и прикоснулся к корню реальной проблемы. Через несколько лет я снова прошёл этот шаг, но уже с пониманием токсического стыда. Тогда я ясно увидел, что 95% всего стыда было результатом травмы заброшенности. Когда я увидел это, я был готов что-то с этим сделать. Для стыдливого человека увидеть то, что его стыд большей частью пришёл из вне довольно обнадеживающе. Это даёт надежду на излечение.

Четвёртый шаг помогает человеку увидеть свои проступки в таком свете, что бы открыть дорогу к излечению. В 4 шаге начинается процесс трансформации токсического стыда в здоровый, который является основанием для здоровой вины.

\ж{Шаги 5, 6 и 7}: <<Мы признали перед Богом, собой и каким-либо другим человеком истинную суть наших ошибок.>>, <<Мы были полностью готовы к тому, чтобы Бог избавил нас от всех этих дефектов характера.>>, <<Мы смиренно просили Его устранить наши недостатки.>> Я сгруппировал эти шаги вместе, потому что каждый из них это часть процесса отказа от контролирующей и грандиозной воли. Каждый шаг это шаг взятие на себя ответственности за свою жизнь и отказ от контроля. Каждый шаг это акт надежды.

В 5 шаге мы выходим из укрытия. Мы говорим о нашем стыде. Мы говорим Богу и другим людям о нашем стыда. По моему мнению, этот шаг не только помогает сфокусироваться на своих проступках и ошибках, но также помогает увидеть эти действия вытекающими из дефектов характера, которые были защитными механизмами стыда. Рассказывая другим людям об этом, мы принимаем боль нашего стыда, и раскрываем себя перед глазами других. Мы позволяем другим увидеть то, как плохо мы себя на самом деле чувствовали. Больше никакого притворства и прикрытия.

Шаг 6 это акт веры и надежды. Мы видим себя в достаточно хорошем свете для того что бы поверить, что Бог удалит эти дефекты характера. По крайней мере верим, что у нас есть право зависеть от кого-то или чего-то большего, чем мы сами. Грандиозный контроль и игра в Бога окончена. Нам нужна помощь, мы знаем это и мы просим об этом. Предположение под прошением удаления этих дефектов это вера в то, что человек достоин их удаления.

В 7 шаге мы смиренно просим Бога убрать наши недостатки. Просить смиренно значит восстановить свой здоровый стыд. Мы знаем, что ошиблись. Мы люди и совершаем ошибки, но мы также верим, что нам можно помочь. Мы можем измениться. Мы можем извлечь уроки из нашей боли и несчастья.

Этим шагом я восстановил свой здоровый стыд. Из этого здорового стыда выросла здоровая вина. Вина это эмоция, которая формирует совесть. Быть бесстыдным значит не иметь совести. Наша совесть говорит нам, что мы ошиблись, переступили собственные ценности. Эмоция вины побуждает нас измениться. Виноватый человек боится наказания и хочет загладить вину. Стыдливый человек хочет быть наказанным. Когда я воссоединился со своей виной и совестью, я перешёл к заглаживанию вины.

Шаги 5, 6 и 7 возвращают нас к самим себе. Мы принимаем себя настолько, что бы быть готовым говорить о наших ошибках. У нас достаточно надежды о самих себе, что бы просить у нашей высшей силы о помощи. Мы готовы быть ответственными, что бы исправить ошибки, двигаться вперёд и расти.

\ж{Шаги 8 и 9} это шаги исправления. Они гласят: <<Мы составили список всех людей, кому мы причинили вред, и обрели готовность возместить им всем причинённый ущерб.>>, <<Мы напрямую возмещали причинённый этим людям ущерб, где только возможно, кроме тех случаев, когда это могло повредить им или кому-либо другому.>>

Теперь мы обращаемся к третьему разрушенном взаимоотношению, указанному в Падении. Взаимоотношению с людьми. Возможно самая большая рана, которую несёт стыдливый человек, это неспособность быть интимным во взаимоотношениях. Эта неспособность произрастает напрямую из фундаментальной нечестности, корнем которая является токсический стыд. Ношение масок, прятки и бесконечные секреты исключают любую возможность честности во взаимоотношениях. И как я подчеркнул ранее, стыдливые люди всегда ищут взаимоотношений с точно такими же стыдливыми партнёрами.

Секретность, нечестность и постоянные игры были, определённо, важной чертой моей истории взаимоотношений. Я делал больно женщинам, относясь к ним как игрушкам.

Я помню как мне однажды сказали в программе каким жестоким я был к женщинам. Они сказали, что мне не следует встречаться женщинами, находящимися в программе. Это одно из правил в большинстве программ.


Я немедленно начал встречаться с женщиной из программы. И как полностью дисфункциональный взрослый ребёнок и алкоголик, я всегда выбирал женщин, которые были сильно больны. Обычно я жалел их и чувствовал себя мощным и сильным в их присутствии. Со временем бремя отчаянной зависимости становилось все невыносимей и я их бросал. Это было также нечестно и жестоко. Когда я столкнулся с моей жестокостью и пассивно-агрессивной злостью в сторону женщин, я был в полной ярости! Заняло годы, что бы понять, что я воспроизводил мою злость в сторону собственной матери довольно разрушительными пассивно-агрессивными способами. Моя роль героя, хорошего парня и бывшего семинариста вводили меня в замешательство. После стольких лет исполнения этих ролей, я легко бы прошёл детектор лжи в их отношении.

Также были жертвы моего патронирования. Были студенты, которые несли на себе мой стыд, который я прятал под чёрной мантией перфекционизма, святости и неприкосновенности. В особенности пострадал мой брат от моей патронированной святости. Я был жёстким семинаристом во время моего обучения в священники. В ранние дни моего обучения бывало что я проводил по 6 часов на коленях, стараясь не пошевелить ни одной мышцей. Внутренне я чрезвычайно осуждал себя, что и передалось моим ближним и семье.

Во-время моих пьяных эпизодов я был яростен и жесток. Я разрушал чужое имущество и нарушал личные границы и права людей. Цель вины сподвигнуть человека устранить причинённый ущерб. Вина помогает возместить ущерб и двигаться дальше.

\ж{Шаги 10, 11 и 12} это шаги по поддержанию этих восстановленных отношений. Шаг 10 гласит: <<Мы продолжали проводить личную инвентаризацию и, когда были неправы, сразу признавали это.>> Шаг 11: <<Мы искали путём молитвы и размышления улучшения нашего сознательного контакта с Богом, как мы понимали Его, молясь лишь о знании Его воли для нас и о даровании силы её исполнить.>> Шаг 12: <<Духовно пробудившись в результате этих шагов, мы пытались нести эту весть другим и применять эти принципы во всех наших делах.>>

Шаг 11 продолжает и углубляет нашу взаимную связь с Богом. Он способствует взаимоотношению осознанного контакта. Это истинное взаимоотношение. Мы прошли полный круг, начиная с разрушенного и заброшенного источника взаимоотношений, который привёл нас к интернализации стыда, и заканчивая дружбой с Богом, так как мы его понимаем.

Шаг 10 поддерживает наши взаимоотношения с собой. Он держит нас на связи со здоровым стыдом, эмоцией, говорящей о том, что мы можем и будем делать ошибки. Находясь в этой связи, мы можем принять себя. Признавая свои ошибки значит принять и выразить наши уязвимости и ограничения. Подобная осознанность останавливает наше стремление стать грандиозным и бесстыдным.

Шаг 12 провозглашает, что духовное пробуждение это цель и продукт 12-ступенчатой программы. Он подчёркивает факт того, что токсический стыд и все его прикрытия ведут к духовному банкротству. Токсический стыд это убийство души. Из-за него, мы становимся недолюдьми, без жизни внутри себя и внутреннего мира. Стыдливые люди жаждут внутренней тишины и уединения.

Тишина и уединение это знаки духовной зрелости. Они ведут к миру и блаженству. Духовная жизнь это внутренняя жизнь. Она не может быть достигнута снаружи. Духовная жизнь сама по себе является наградой и не стремиться ни к чему другому за её гранью. Когда мы достигаем внутреннего мира и осознанного контакта - нас переполняет. Это знак правды и любви призывающий двигаться в сторону добра и трансцендентности.

Древние философы называли доброту, правду, красоту и любовь трансцендентными качествами существования.

Шаг 12 движет нами, что бы передать сообщение нашим братьям и сёстрам, которые все ещё скрываются за масками токсического стыда. Этот шаг призывает нас практиковать духовные принципы строгой честности и служение другим во всех делах. Он призывает нас привлекать других через моделирование жизни с любовью и уважением. Подавая пример восстановленных взаимоотношений с Богом, собой, ближним и миром, мы можем показать другим, что выход есть. Надежда существует.

\ж{Освобождение внутреннего ребёнка}

В книге Bradshaw On: The Family описано три особые фазы моего собственного ослабления стыда и процесса экстернализации. Рисунок 6.1 визуально показывает эти фазы.

Первая фаза это фаза восстановления. Через группу поддержки и зеркалирующие глаза, я восстановил своё чувство достоинства. Увидев себя в не пристыжающих глазах других, я почувствовал себя лучше. Я воссоединился с самим собой. Я больше не был совершенно один. Группа и ближние люди восстановили чувство межличностной связи.

Процесс восстановления это процесс первого порядка. Это значит, что я изменил одно поведение на другое. Я бросил пить и перестал изолировать себя. Я поделился своими переживания и надеждами. Я начал делиться своими чувствами. Я стал чувствовать свои чувства. Я сместил свою зависимость на новую семью, которую я открыл для себя. Зависимость все ещё существовала, так же как и пристыженный ребёнок внутри меня, который сделал группу моими новыми родителями, оберегающими меня.

Мой стыд ослаб, но все ещё был активен. Это было доказано тем фактом, что я все ещё был компульсивен и имел проблемы с интимностью. Я выбирал женщин, которым я был нужен, путая любовь с жалостью. Я начал работать по 12 часов в день, включая субботу. Я стал больше курить и есть сахара. Действительно, я остановил угрожающую жизни болезнь - алкоголизм; я ослабил мой стыд; я видел себя в лучшем свете, но все ещё был компульсивен. Я не был ещё свободен.

Для того, что бы стать свободным, я должен был провести работу с семейным источником проблемы. Мне все ещё нужно было вырасти и по-настоящему покинуть родной дом.

Фритц Перлз однажды сказал: <<Цель жизни в движении от поддержки окружающей средой к самоподдержке.>> Цель жизни это достижение независимости. Независимость произрастает из здорового чувства стыда. Мы сами отвечаем за собственные жизни.

Наши первоначальные взаимоотношения были связаны с неудачным моделированием и заброшенностью. Это создало стыдливую идентичность. Из-за того, что мы не имеем аутентичного <<Я>>, мы цепляемся к нашим родителям в вымышленной связи или строим стены вокруг себя, что бы никто не мог причинить нам боль. Эти ранние импринты окрасили все наши последующие взаимоотношения.

Однажды я услышал слова Werner Erhard: <<До тех пор пока мы не разрешим наши первоначальные взаимоотношения, мы не можем вступить ни в какие другие отношения.>> Покинуть родной дом значит разорвать первоначальные отношения, и т.к. наш стыд стал результатом этих отношений, уход из дома становится мощным способом ослабления стыда.

Уход из дома
Что включает в себя уход из дома? Как нам это сделать?
Уход из дома это вторая фаза на пути к целостности. В этой фазе мы идём на контакт с раненым и одиноким внутренним ребёнком, который был заброшен много лет назад. Этот ребёнок - часть нас, которая хранит в себе заблокированную эмоциональную энергию. Эта энергия особенно заблокирована, если мы пережили серьезное насилие. Для воссоединения с раненым ребёнком мы должны вернуться назад и пережить те эмоции, которые были заблокированы.

Когда мы формируем эмоциональные блоки, они сильно влияют на нашу способность думать и рассуждать. Наш разум сужен в своём диапазоне виденья. Наше суждение, восприятие и умение рассуждать относительно событий в нашей жизни сильно искажено.

Когда практичное суждение отключается, воля, которая является исполнителем, теряет свою способность видеть альтернативы и больше не основывается на реальной ситуации. Эмоционально закрытые люди буквально наполнены волей, они становятся своенравными. Своенравность характеризуется грандиозностью и необузданными попытками контроля. Своенравие это игра в Бога.

Единственный способ снять наш мозг с крючка и вылечить компульсивность это вернуться и пережить эмоции. Заблокированные эмоции должны быть пережиты как они впервые появились. Неудовлетворенные и неразрешенные потребности должны быть удовлетворены с новыми уроками и правильным опытом.

Наше потерянное детство должно быть оплакано. Компульсивность это результат тех старых заблокированных чувств (неразрешенного горя), которые проигрываются снова и снова. Или мы проработаем эти чувства через перепроживание их, или мы будем компульсивно действовать согласно им. Мы также можем направить эти заблокированные эмоции на самих себя через депрессию и суицид, или на других, передавая таким образом стыд.

Мы должны покинуть родной дом и стать хозяином самому себе для излечения компульсивности. Мы должны проработать первоначальную боль.

\ж{Работа с первоначальной болью}

Любой стыдливый человек жил в травматической семье. Дети с травмой переживали слишком много стимулов за короткий период времени с которыми они не смогли справиться. Все формы травмы заброшенности стимулируют в детях эмоцию горя и тут же блокирует его выражение.

Однажды, я наблюдал за мужчиной и его дочкой в аэропорту. Они сидели недалеко от меня. Отец непрерывно ругал ребёнка, и в какой-то момент злостно сказал ей, что она приносит много неприятностей, так же как её мать. Я предположил, что они были в разводе. Когда он уходил, он дал ей пару шлепков. Когда ребёнок заплакал, он ударил её снова. Затем он потянул её в отдел с мороженым, что бы заткнуть её. Этот ребёнок с ранних лет учиться, что она не желанна, что это все её вина, что её чувства не имеют значения и что она в ответственности за чувства других людей. Я не могу представить где бы она могла найти союзника, который бы посидел с ней, подтвердил её грусть и разрешил погоревать.

В здоровой семье чувства ребёнка имеют значение. Травмы в детстве неизбежны.

Как говорит Alice Miller: <<Не травмы, пережитые в детстве являются источником страданий, а невозможность их выразить.>>

Когда ребёнка забрасывают или подвергают насилию, то присутствует обида и боль. Детям требуется, что бы их боль подтвердили. Им нужно показать, как выразить свои чувства. Им нужно время для этого и помощь. Каждый заброшенный ребёнок не стал бы стыдливым, если бы в это время кто-то помог им пережить их боль и дал время на разрешения горя.

Я вспоминаю здоровую семью, в которой отец был серьёзно ранен в домашнем происшествии. Шестилетний сын играл снаружи, когда услышал взрыв. Он был шокирован, когда нашел отца, в крови и искалеченным. Отец приказал ему позвонить в скорую. Сын провел остаток дня с соседом, пока с работы не приехала мама. Мать отвела его к детскому терапевту. Он боялся спускаться в подвал (там произошел взрыв). Он злился на мать, из-за того что она не была дома и на отца, за то что покинул его (когда его увезли в больницу).

За следующие месяцы мальчик проработал свои чувства в контексте символической игры. Его мать и отец были рады тому, что он смог выразить свой гнев в их сторону. Они поддерживали его в его работе со страхов. Они делились с ним своими чувствами.

\ж{Подтверждение}

Для разрешения горя должны произойти несколько вещей. Первое это подтверждение. Наша травма заброшенности должна быть подтверждена как реальная проблема. Возможно, одно из самых разрушительных последствий токсического стыда это то, что мы на самом деле и не знаем насколько мы депрессивны и злы. Мы, в действительности, не чувствуем не разрешенного горя. Наше ложное <<Я>> и защитные механизмы держат нас в стороне от этого переживания. Парадоксально, но те самые защитные механизмы позволившие нам пережить детские травмы, теперь стали преградой. Мы должны раскрыть наше застывшее горе.

Я помню как моя бабушка высмеивала мою истерику, когда мой отец ушел напиваться. Он только что поссорился с моей матерью и ушел из дома в гневе, поклявшись напиться. Я начал плакать и очень скоро был истерике. Бабушка высмеяла меня, назвала маменькиным сынком и попросила держать себя в руках. Я не смог забыть это переживание. Годы спустя я все ещё нес в себе неразрешенное горе.


\ж{Поддержка}

Величайшая трагедия в том, что мы знаем как горе естественно лечит, если у нас есть поддержка.
Причина, по которой люди откладывают скорбь по своему горю, заключается в том, что у них нет никого рядом для подтверждения и поддержки. Вы не можете скорбеть в одиночестве. Миллионы таких как мы пробовали. Мы ложились спать в слезах и закрывались в ванных комнатах.

Отложенная скорбь это корень того, что называется посттравматическим стрессовым синдромом. Солдаты, вернувшиеся с войны, переживают похожие симптомы: паника, физическое онемение, пугливость, деперсонализация, нужда в контроле, кошмары и расстройства сна. Те же симптомы распространены у детей из дисфункциональных семей. Это симптомы неразрешенного горя.


\ж{Работа с горем}

После подтверждения и поддержки человек должен пережить чувства, которые ранее были заблокированы. Это должно происходить в безопасной не пристыжающей среде. Чувства, связанные с работой над горем это злость, сожаления, обида, депрессия, грусть и одиночество. Время работы варьируется в зависимости от интенсивности травмы. Каждому человеку нужно время для завершения этой работы, т.к. в дисфункциональных семьях этого времени не было.

В нашем центре для семей в Хьюстоне мы проводим эту работу в течении 4 с половиной дней. Мы используем роли семейных систем как способ, который помогает увидеть людям как именно они потеряли свою аутентичную личность и застряли в ложной. Увидев этот процесс убийства души, человек начинает горевать. Часто терапевт помогает принять эти чувства, т.к. они все ещё повязаны стыдом. Когда человек воссоединяется со своими истинными чувствами, стыд ослабевает. Эта работа продолжается и после воркшопа. Она может идти в течении нескольких лет.

Существует множество методов подобной работы, но она должна быть проделана, если мы хотим прекратить компульсивное поведение.

\ж{Корректирующий опыт}

Работа с неразрешенным горем это процесс перепроживания, освобождения и интеграции с вашим потерянным внутренним ребёнком.
Воссоединение очень важно, т.к. пренебрежение нашими потребностями в зависимости были крупным источников токсического стыда. Каждая стадия развития была уникальной в свои потребностях и динамике. В младенчестве нам нужно зеркалирование безусловной любви. Нам нужно слышать такие слова (невербально) как <<Я рад, что ты здесь. Добро пожаловать в нашу семью. Я рад, что ты мальчик/девочка. Я хочу слышать тебя, держать и любить тебя. Твои потребности важны для меня. Я уделю тебе все своё внимание и время для их удовлетворения.>>

Я люблю организовывать небольшие группы (6-8 человек) и посадить одного человека в центр. Человек в центре указывает на то, как близко он хочет, что бы находились другие участники группы. Некоторые хотят, что бы их держали и покачивали. Другие хотят только прикосновений. Каждый человек сам устанавливает границы дозволенного.

После того, как группа определена мы включаем колыбельную, а каждый человек в группе говорит одну из вербальных аффирмаций, в это время касаясь, поглаживая или просто сидя рядом с человек в центре группы.

Люди, чьими потребностями пренебрегали, начнут всхлипывать. Если человек был Потерянным ребёнком, то он может начать рыдать. Эти слова касаются дыре в его душе.

После аффирмаций группа обсуждает их опыт. Аффирмации, прикосновения, поддержка и последующее обсуждение становится корректирующим опытом.

Я также предлагаю другие способы удовлетворения младенческих потребностей. Обычно человеку потребуется друг, который предоставит свою физическую поддержку (обилие прикосновений), и будет кормить их (водить куда-нибудь есть). Им нужно множество приятных кожных ощущений. Возможно им потребуется горячая ванна или теплое покрывало. Или они могут попробовать телесный массаж.

Далее мы подходим к потребностям ползунков. Ползункам требуется чувство разделения, поэтому все садятся близко к человеку в центре, но раздельно от него. Обычно я делаю медитацию с возрастной регрессией, в которой я прошу человека в центре почувствовать себя ползунком. Я даю такие аффирмации как <<Ты можешь бродить и исследовать. Ты можешь отделиться от меня. Я не покину тебя. Ты можешь проверить свои границы. Ты можешь злиться, говорить нет. Ты можешь это делать любым способом, каким хочешь. Я буду рядом. Ты не обязан спешить. У тебя есть все время которое тебе нужно. Ты можешь держаться и отпускать. Я не покину тебя.>>

И снова группа делиться своим переживанием после того, как каждый услышит эти аффирмации. Часто, во-время обсуждения, люди выражают глубокие эмоции. Часто они вспоминают эпизоды заброшенности, о которых давно забыли.

Мы проходим через все стадии развития до подросткового возраста. Этот возраст также важен, потому что многие люди проходили через болезненную заброшенность и пристыжающие инциденты в этот период.

Обычно я прошу каждого человека написать письмо своим родителям о том, в чем они нуждались, но не получили от них.

Wayne Kritsberg советует писать такие письма не доминирующей рукой, т.к. это позволяет почувствовать себя ребёнком. При чтение подобных писем перед группой происходит разрядка сильных эмоций. После прочтений, я прошу группу дать аффирмации человеку, на основе тех потребностей, которые были не удовлетворены в его письме.

К концу воркшопа я прошу всех участников встретиться со своим Потерянным ребёнком. Сложно описать словами мощь этого упражнения. Сделать это можно через медитацию. Вы можете записать её аудио и в дальнейшем воспроизводить.


\ж{Медитация: встреча с вашим Потерянным ребёнком}

Сядьте в вертикальном положении. Расслабьтесь и сфокусируйтесь на своём дыхании. Проведи некоторое время следя за дыханием. Следите за входящим и выходящим воздухом. Заметьте разницу в воздухе, когда он входит и когда выходит. Сфокусируйтесь на этой разнице (одну минуту).

Теперь представьте, что вы спускаетесь по длинной лестнице вниз. Спускайтесь медленно, пока я считаю до 10. Десять... (10 секунд) Девять... (10 секунд) Восемь ... (10 секунд) и т.д.

Когда вы достигнете низа, поверните налево и пройдите в длинный коридор с дверьми по обе его стороны. На каждой двери нарисован цветной символ (1 минуту). Когда вы посмотрите в конец коридора, вы увидите силовое световое поле. Пройдите через него и вернитесь во времени на улицу, на которой вы жили в возрасте до 7 лет. Пройдите по этой улице к дому в котором вы жили. Посмотрите на этот дом. Отметьте его цвет, окна, двери.

Посмотрите на ребёнка, выходящего из входной двери дома. Во что он одет? Какого цвета его обувь? Подойдите к нему. Скажите ему, что вы пришли из его будущего. Скажите, что вы знаете как никто другой, через что он прошёл. Его страдания, заброшенность, его стыд. Скажите ему, что из всех людей, кого он только будет встречать, вы единственный, кого он никогда не потеряет. Теперь спросите его, пойдет ли он с вами домой? Если нет, скажите, что навестите его завтра. Если он согласен, возьмите его за руку и начинайте уходить. Когда вы будете уходить от дома, на его крыльцо выйдут ваши родители. Помахайте им. Смотрите через ваше плечо на них, пока вы уходите. Они становятся все меньше в размере, пока совсем не исчезают.

Поверните за угол и посмотрите на вашу Высшую Силу и ваших самых заветных друзей, ждущих вас. Обнимите ваших друзей и позвольте вашей Высшей Силе войти в ваше сердце. Теперь уходите и пообещайте вашему ребенку, что вы будете встречаться с ним каждый день на пять минут. Выберите точное время. Возьмите ребёнка в свои руки и позвольте ему уменьшиться до размеров ваших рук. Теперь поместите его в ваше сердце.

Теперь идите в какое-нибудь красивое природное место. Встаньте в центре этого места и поразмыслите над опытом, который вы только что пережили. Почувствуйте ощущение общности с собой, со своей Высшей Силой и вообще со всем сущим. Теперь посмотрите на небо. Взгляните на фиолетовое облако в форме цифры 5. 5 превращается в 4... почувствуйте ваши ноги... 4 трансформируется в 3... почувствуйте жизнь в вашем животе и руках... 3 превращается в 2... почувствуйте свои руки, лицо и все тело. Знайте, что вы сейчас полностью придёте в сознание, взгляните как 2 превращается в 1 и вы полностью просыпаетесь и помните ваше переживание.

Я призываю вас найти ваши ранние фотографии. Желательно до 7-летнего возраста. Положите её в ваш кошелек, сумку или на рабочий стол, что бы вспоминать об этом ребёнке, живущем внутри вас.

Этот ребёнок самая живая и спонтанная наша часть и её нужно полностью интегрировать в нашу жизнь.


\ж{Удовлетворение детских потребностей во взрослом возрасте}

Мы проигрываем наши потребности в области развития всю нашу жизнь. Каждый раз, когда мы начинаем что-то новое, мы запускаем наши младенческие потребности. Когда мы уверяемся в безопасности и начинаем доверять окружению, наша ползунковая часть начинает исследовать и экспериментировать. Наши собственные дети запускают наши потребности, когда они сами проходят различные потребности в области развития. Как взрослые, у нас есть возможность позаботиться о самих себе в каждой из этих стадий.

Во взрослой жизни мы может создавать такой контекст, что бы удовлетворять эти потребности. Мой отец пренебрегал мной. Я создал группу мужчин, которая служит как группа поддержки и даёт мне обратную связь. Я понял, что могу сделать так, что бы то, что я получаю от других, удовлетворяло мои потребности. Детям всегда чего-то недостает. Взрослые учатся удовлетворять себя из того что у них есть. Так я могу взять мероприятие обмена опыта в группе и сделать из него опыт отцовства. Также я могу научиться получать то, что мне действительно нужно. Я могу быть добр к себе и относиться к себе с уважением и добротой.

\ж{Универсальный квест Внутреннего ребёнка}

Важно понять, что потребность в поиске Внутреннего ребёнка это часть пути к целостности каждого человека. Ни у кого не было идеального детства. Каждый несет в себе неразрешенные подсознательные проблемы своей семейной истории.

Путь Внутреннего ребёнка это геройский путь. Стать полностью функциональным человеком это героическая задача. На пути встретятся испытания и невзгоды. В греческой мифологии Эдип убивает отца, Орест убивает мать. Уход от родителей это одно из препятствий, которое человек должен преодолеть на своей геройской пути. Убийство своих родителей символизирует уход из дома и взросление.

Поиск Внутреннего ребёнка это первый прыжок через пропасть печали, которая угрожает нам всем. Нахождение Внутреннего ребёнка это лишь начало. Из-за своё изоляции, пренебрежения и нуждаемости, этот ребёнок эгоцентричен, слаб и испуган. Он должен быть дисциплинирован, что бы он смог выпустить свою огромную духовную силу.


\ж{Интеграция отвергнутых частей}

<<Курс нашей жизни определен множеством личностей, который живут внутри каждого из нас. Эти личности постоянно взываю к нам - в наших снах и фантазиях, в нашем настроении, недуге и в множестве непредсказуемых и необъяснимых реакциях на мир вокруг нас.>> Hal Stone и Sidra Winkelman.

<<Эти трансформации не могут пройти без следа. Они требуют моего понимания о неоднозначности знания о том, что кто я есть - я также есть его противоположность. Я не могу избавиться от моих демонов, не рискуя, что мои ангелы улетят вместе с ними.>> Sheldon Kopp.

Как ранее стыдливый человек, мне пришлось тяжело работать над полным принятием себя. Часть работы над самопринятием заключается в интеграции повязанных стыдом чувств, потребностей и желаний. Большинство стыдливых людей чувствуют стыд когда им требуется помощь; когда они чувствуют злость, грусть, страх или радость; также когда действуют сексуально или напористо. Эти важные части были отщеплены от нас.

Мы пытаемся действовать, как будто мы ни в чем не нуждаемся. Мы притворяемся, что не чувствуем то, что чувствуем. Я думаю о всех тех случаях, когда говорил, что я порядке, когда мне было грустно или обидно. Мы либо заглушаем нашу сексуальность и действуем по пуритански, либо используем сексуальность, что избегать все другие чувства и потребности. Во всех случаях мы отрезаны от наших жизненно важных частей. Эти отрезанные части обычно проявляются в наших снах и проекциях. В особенности это правдиво в случае сексуальности и природных инстинктов.

Вот пара примеров:

Мужчине сниться, что он находиться в классной комнате. Группа женщин раздевает его. Одна из них сексуально трогает его и он возбуждается.
Сон о диком животном, которое преследует меня.
Сон о грабителе, который пытается проникнуть в мой дом.
Женщине сниться как она проезжает мимо жадно смотрящего на неё черного мусульманина.

Каждый сон представляет инстинктивную энергию, от которой мы отреклись. Сны это великолепный способ соприкоснуться к этими отщепленными частями. Эти части пытаются привлечь наше внимание. Отщепленные части нашей личности это энергия - эмоция или желание или потребность, которую пристыжали каждый раз, когда она проявлялась. Эти энергетические шаблоны подавлены, но не уничтожены. Они живут в нашем подсознании.

Юнг называл эти отщепленные аспекты нашей личности нашей тенью. Без интеграции нашей тени, мы не можем стать цельными личностями.


\ж{Диалог с голосами по Hal Stone и Sidra Winkelman}

В книге Embracing Our Selves, Hal Stone и Sidra Winkelman разработали мощный подход к преодолению самоотчуждения, которое стало результатом токсического стыда. Их работа основана на предпосылке, что наша личность состоит из множества личностей, который живут внутри каждого из нас. Эти личности стали результатом само-расщепления, которое происходит естественным путём во-время взросления. Так как наши опекуны неидеальны, никто из них не мог принять нас с идеальной безусловной любовью. Каждый по своему накладывает свои ценности на нас и оценивает нас по ним. Делая так, они естественно отвергают те части, которые не соответствуют их представлению о мире. Эти части отщепляются и за долгое детство в какой-то степени приобретают автономию.

Каждая отщепленная часть становится маленькой личностью. Эти личности постоянно взывают к нам <<в наших снах и фантазиях, в нашем настроении, недуге и в множестве непредсказуемых и необъяснимых реакциях на мир вокруг нас.>> Эти внутренние личности переживаются как внутренний голос. Чем более осознанными мы станет в отношении этих голосов, тем больше будет наша свобода. У всех есть эти голоса, но стыдливые люди находятся с ними на ножах, поэтому им больше чем кому-либо требуется интегрировать их множество личностей.

Как форма энергии, отщепленные части оказывают на нас значительное влияние. Стыдливые люди как правило измучены. Они тратят много энергии держась за свои ложные личности-маски и пряча свои отщепленные части. Я сравнил это с держанием пляжного мяча под водой. Вирджиния Сатир с голодными собаками в подвале. Подавленные части оказывают огромное давление, заставляя нас продолжать защищать их противоположности.

В то время как мы склонны отвергать отщеплённые части, они также несут для нас определённый интерес. Основная предпосылка работы Stone и Winkelman в том, что все наши части в порядке. Ничто не может быть более обнадеживающим и менее пристыжающим. Каждый аспект каждого человек имеет решающее значение для его цельности и полноты. Не существует закона, который бы говорил, что одна часть лучше другой. Наше сознание с множеством личностей должно оперировать на принципах социальной справедливости и демократии.

Работа с голосами требует большой приверженности и практики. Я могу дать вам лишь голый набросок её богатой структуры.

Голосовой диалог постулирует сознание как процесс, нежели сущность. Сознание это не что-то, чего мы стремимся достичь; это процесс, который должен быть прожит. Это эволюционный процесс непрерывного изменения, колеблющийся от момента к моменту.

Существует три отдельных уровня сознания: (1) Осознание, (2) Переживание субличностей и внутренних голосов, (3) Эго. Уровень осознания это способность наблюдения, которая не осуждает то, что оно наблюдает. Субличность это опыт переживания части, которая манифестирует себя в виде энергетических шаблонов: физических, эмоциональных, ментальных и духовных.

Гневающийся мужчина, на пример, может переживать повязанный стыдом гнев, который он все эти годы подавлял. Гнев так переполняет его, что он начинает идентифицироваться с гневом. Здесь не осознания. Когда он осознает свою ярость, он может переживать её. Далее он, используя Эго, может стать более осознающим своё переживание. Эго это исполнитель психики - делающий выбор. Эго получает информацию от уровня осознания и из переживания различных энергетических шаблонов. По словам Stone и Winkelman: <<По мере того, как человек движется по пути сознательности, Эго становится более осознающим Эго. Как более осознающее Эго оно становится в силах делать реальных выбор.>>

Это нужное нам направление по излечению стыда. Грандиозная и отключённая воля, о которых я говорил ранее, погрязла в повязанных стыдом и отщепленных эмоциях. С развитием наших ложных перфекционистских, контролирующих, угождающих людям частей, наше Эго теряет свою аутентичную исполняющую силу и начинает идентифицироваться с тем, что Hal и Sidra назвали Защитником/Контролером.


Защитник/Контролер часто проявляется как субличности Перфекциониста, Внутреннего Критика или Льстеца. Как только Эго идентифицируется с одной из этих субличностей, оно теряет свою способность делать реальный выбор.

Идентификация Эго с Перфекционистом, Защитником/Контролером, Внутренним Критиком или Льстецом я называю Ложным <<Я>>. Главное это отличать каждую из этих субличностей от Эго, которое является подлинным исполнителем личности.


Как только мы идентифицируемся с одной из субличностей, мы теряем возможность выбирать. Цель в том, что бы соприкоснуться с энергией, которую мы переживаем и увидеть её как одну из многих энергетических шаблонов, которую нужно интегрировать, что бы повлиять на сознательность выбора, который ведет к интегрированным действиям. Все субличности это наши части - мы должны ценить и принимать их. Нам лишь нужно осознать, что голос который мы слышим, просто голос. Пробуждение процесса сознания и расширение являются целью. 

Stone и Winkelman подводят итог голосового диалога:
\begin{enumerate}
\item Исследование субличностей и энергетических систем

Субличности также упоминаются как Голоса. Диалог с голосами напрямую вовлекает субличности в диалог, без вмешательства критичного, смущённого или репрессивного Защитника/Контролера. К каждой субличности обращаются с полным признанием её индивидуальной важности, так и её роли, как части всей личности. Каждая субличность переживает жизнь по-разному.

Эти субличности также можно рассмотреть снаружи. Как энергетически шаблоны их можно увидеть через телесные проявления. Я видел, как люди руководящих должностей на моем воркшопе по Внутреннему Ребёнку трансформировались на моих глазах: бороздчатый лоб, тугие щеки и челюсти, сжатые губы могут широко раскрыться, появится расслабленная детская ухмылка.

\item Прояснение Эго

Голосовой диалог отделяет Эго от Защитника/Контролера и окружения из субличностей, которые работают с этим доминирующим энергетическим шаблоном. Субличность Защитника/Контролера развилась у всех людей. Она развилась, что бы заботиться об уязвимом ребёнке. В стыдливых людях Защитник/Контролер жестко идентифицирован с Эго. Энергетический шаблон Защитника/Контролера становится доминирующим. Другие субличности могут существовать отдельно от него, так и быть его частью.

Когда любая из суб-личностей начинает перенимать функцию Эго, ведущий будет указывать на это. Он попросит субъект переместиться в другое место и вовлечёт эту субличность в прямой диалог. Таким способом Эго становится все более дифференцированным, т.к. становится более осознанным Эго.

Это значит, что мы начинаем осознавать все места, в которых прячется стыд. Защитник/Контролер, Критик, Перфекционист это все маски ложного <<Я>>. Они ложные в том смысле, что Эго идентифицировалось с ними. Увидеть, что эти части или субличности были заморожены, что бы пережить стыд, который мы интернализировали, значит понять, что эти части это не то, кем мы на самом деле являемся. Мы не оцениваем эти части как плохие. Они просто части.

\item Усиление осознания

Наиболее важной частью диалога с голосами является осознание. Контекст диалога с голосами предоставляет физическое пространство для каждой субличности, для Эго, которое координирует и исполняет для осознания. Физическое пространство осознания отделено от двух других. Эта точка, через которую человек может наблюдать и обозревать все, что происходит без осуждения. Осознание это не о решениях и действиях. Ничего не нужно менять. Осознание это точка, где все отмечается и принимается. В осознании человек ясно видит драму, разыгрываемую субличностями в отношении к Эго.

Диалог с голосами невозможно делать одному. В начале нужен ведущий. На самом деле, это взаимный процесс и совместное предприятие. Ведущий и субъект кооперируются в поиске субличностей в попытке понять их функции. Со временем ведущий становится более чувствительным к ведению и изменению энергетических шаблонов. Это значит, что его осознание находиться в постоянном расширении.

Вы сами можете сделать следующее упражнение. Я адаптировал его из работ Hal Stone и Sidra Winkelman. Я называю его <<Примирение со всеми вашими жителями>>

\end{enumerate}


\ж{Примирение со всеми вашими жителями}

\begin{enumerate}
\item Подумайте о людях, которые вам не нравятся. Проранжируйте их в соответствии с интенсивностью эмоций, первым в списке должен оказаться самый достойный вашего презрения. Напишите пару строк о каждом человеке, отмечая самые отвратительные черты характера.

\item Прочитайте каждое имя в вашем списке. Остановитесь и задумайтесь о самых важных аспектах этих людей. Осознайте ваши собственные чувства. Какой признак вызывает самые сильные чувства праведности?

\item Сократите описание людей до их самых предосудительных черт. 

На пример:
\begin{enumerate}
\item Джо - грандиозный эгоманьяк
\item Гвенела - агрессивная и грубая
\item Макс - лицемер
\item Фаркуар - использует христианство как фасад
\item Ротгар - слабак.
\end{enumerate}

\item Каждая из этих черт характера представляет собой вашу отщепленную часть - энергетический шаблон, который вы не хотите интегрировать в свою жизнь. Вы экстернализировали черту личности, от которой отреклись.

\item Каждая отщепленная часть имеет противоположный энергетически шаблон, с которым идентифицируется Защитник/Контролер. Вы тратите много энергии, что бы поддерживать эти отщепленные части. Это объясняет то, почему мы чувствуем такую сильную энергию в отношении своих врагов. Важно интегрировать эту энергию и творчески подойти к её использованию.

Спросите себя о каждом человеке из вашего списка. Каким учителем он может стать для меня? Чему я могу научиться слушая его? Человек, к которому вы чувствуете отвращение, может помочь вам увидеть те части, с которыми вы чересчур себя идентифицируете.

В моем списке Джо помогает мне увидеть, что я чересчур идентифицирую себя со скромностью. В моем случае я больше стараюсь выглядеть скромным. Гвенела помогает увидеть мою идентификацию с угодничеством. Макс показывает моё желание помогать всем, не получая ничего в замен. Такая помощь бесчеловечна. Это продукт токсического стыда. Фаркуар - указывает на мою идентификацию с идеальным христианином, а Ротгар - указывает на идентификацию с силой и напористостью. Будучи сильным я пытаюсь быть более чем человеком - отказываясь от нормальных человеческих ограничений. Это отказ от здорового стыда.

\item Проходя по своему списку поговорите с каждой отщепленной частью. Спросите её, что она думает. Спросите её, как измениться ваша жизнь, если вы примете её. Послушайте эту часть. Посмотрите на мир через её точку зрения. Почувствуйте любую новую энергию, какую она принесет. Это может стать источников новых идей или решений ваших старых проблем. Точка зрения этих частей ранее не была вам доступна.

Вас может удивить эта новая энергия, которую вы получите во время этого упражнения. Вы вскрываете ваши части, которые были ранее закрыты и секретны. Вы превращаете вашу тень в свет. Вам не нужно становится вашими отщепленными частями. Это было бы то же самое, что вы делали до этого - идентифицировались с одной частью, исключаю другую. В этом упражнении вы учитесь разговаривать и слушать ранее пристыженные и отщепленные части. Делая это, вы освобождаете энергию, которая была связана стыдом.
\end{enumerate}

\ж{Тусовка частей}

\ж{Работа Вирджинии Сатир}

Пионером работы с частями в лечении токсического стыда была Вирджиния Сатир. Я помню как однажды увидел её работу с семьёй. её принятию и воспитанию стоит поучиться. Работа по зеркалированию позволяла каждому человеку принять себя. Когда это происходило, вся семья сближалась друг с другом. Работа была на столько красивой, что я заплакал наблюдая за ней.

В своей книге Your Many Faces Вирджиния представляет ядро техники, которой пользуются множество терапевтов по всей стране. Общеизвестное название - Тусовка частей. Я видел адаптацию этого упражнения различными школами терапии. Далее я представляю свою вариацию этого мощного упражнения. Я убеждён, что эту работу лучше проделывать в группе. Так оно имеет большое воздействие.

\ж{Медитация: Тусовка частей}

Упражнение - запишите инструкцию на аудио.
Закройте глаза... Позвольте сознанию сфокусироваться на дыхании. Проведите 2-3 минуты наблюдая за дыханием. Представьте цифру 7 на экране. Это может быть черная цифра на белом экране или наоборот. Сфокусируйтесь на цифре 7. Если не можете четко её увидеть, представьте, что вы рисуете её пальцем или слышите голос, произносящий её или делайте все вместе, если можете. Затем увидьте/нарисуйте/услышьте цифру 6; затем 5; 4 и т.д. до единицы.


Фокусируясь на цифре 1 позвольте ей медленно превратиться в дверь перед сценой. Пройдя через эту дверь вы видите красивый небольшой театр. Посмотрите на его стены и сцену...(пауза). Посмотрите на закрытые занавески (пауза). Сядьте на первый ряд и почувствуйте ткань кресла. Пусть это будет ваша любимая ткань. Устройтесь поудобнее (пауза).

Посмотрите по сторонам и сделайте этот театр таким, каким хотите. Затем, посмотрите на открывающиеся занавески. Почувствуйте возбуждение перед представлением. Когда занавески полностью откроются вы увидите вывеску: <<Обзор частей (впишите ваше имя)>>. Подумайте о ваших самых любимых частях и представьте их в виде людей, которых вы хорошо знаете и которые представляют эти части. Я люблю свой юмор и вижу как Джонни Карсона выходит на сцену. Слышу раздающиеся аплодисменты. Затем подумайте о другой любимой части и повторите процесс. Я люблю свою харизматическую речь и честность и вижу, как Ф. Кенеди выходит на сцену. Повторяйте процесс, пока на сцене, по правую от вас руку, не выстроится пять человек.

Теперь подумайте о свои частях, которые вам не приятны и пусть из персонифицированный образ тоже выйдет на сцену. Я не люблю свою неряшливость и неорганизованность и вижу, как мой неопрятный друг выходит на сцену. Я слышу его освистывание аудиторией. Следующая нелюбимая часть это трусость и боязливость. Я вижу эту часть как выходящего на сцену Иуду. Наконец, после пяти нелюбимых и отрицаемых частей, стоящих по левую от вас руку, представьте вашу мудрость в виде образа, выходящую на центр сцены. Это может быть старый мужчина с бородой, молодой, светящийся как Иисус или приятная мягкая женщина, или кто-то другой. Просто позвольте проявится этому образу. Затем, позвольте ему сойти со сцены и подойти к вам. Когда он будет подходить к вам, заметьте, что приходит к вам на ум на его счёт. Услышьте голос образа, приглашающий вас на сцену для обзора ваших частей.

Подойдите к каждому образу, представляющему ваши части; посмотрите ему в глаза. Как эта часть помогает вам? Как эти части ограничивают вас, особенно ваши нежелательные части? Чему вы можете научиться у ваших нежелательных частей? Чему они могут научить вас?

Теперь представьте, что все они разговаривают друг с другом. Представьте их за столом, обсуждающих проблему. Подумайте о конкретной проблему в вашей жизни. Что говорит о ней ваш юмор? Как это помогает вам? Как ограничивает? Чем помогает ваша неорганизованность?

Что бы случилось, если бы вы не имели этой части? Что бы вы потеряли? Как вы хотите изменить ту часть, которую отрицаете? Измените её таким способом, что бы она больше помогала вам. Как вы себя чувствуете изменяя эту часть? Далее повторите эту процедуру с каждой частью. Модифицируйте их так, что бы они устраивали вас.

Затем, подойдите к каждой части и представьте, что она сливается с вами. Делайте это, пока не окажетесь на сцене одни вместе с вашим образом мудрости. Услышьте слова мудрости о том, что это ваш театр жизни. Это место, в которое вы можете возвращаться и обозревать множество ваших личностей. Представьте, как он говорит, что все эти части принадлежат вам, что каждая часть имеет свою взаимодополняющую часть в вашем психическом балансе. Примите решение принимать все ваши части; любить и учиться у них. Посмотрите на уходящего мудреца. Поблагодарите его за урок. Знайте, что вы всегда можете снова встретиться с ним.

Спуститесь со сцены. Осознайте себя сидящим в кресле, смотрящим на сцену театра, на которой проигрывается ваша жизнь. Позвольте вашему сознанию увидеть каждую из ваших модифицированных частей, проплывающих мимо и почувствуйте себя одним большим организмом, с множеством аспектов и взаимозависимых частей. Произнесите и услышьте эти слова: <<Я люблю и принимаю всего меня.>> Произнесите снова.

Встаньте и идите к выходу из театра. Пройдите через дверь, обернитесь и представьте цифру 1. Черную единицу на белой занавеске или белую единицу на черной занавеске. Нарисуйте пальцем или услышьте. Затем представьте цифру 2, 3, почувствуйте жизненную силу в своих пальцах ног. Позвольте ей подняться по ногам. Представьте цифру 4 и почувствуйте, как все ваше тело оживает. Затем представьте цифру 5 и знайте, что вы возвращаетесь в свою нормальную сознательную жизнь. Представьте цифру 6 и скажите, что вы приходите в полное сознание. Почувствуйте место, в котором вы находитесь и когда вы увидите цифру 7, полностью осознайте ваше бодрствующее сознание.

\ж{Работа со снами}

Другой способ работы с пристыженными и неприемлемыми частями это научиться интегрировать и интерпретировать ваши сны. Полное объяснение работы со снами займет слишком много место в этой главе или даже в книге. Я лишь хочу, что бы вы были в курсе этого мощного инструмента для само-интеграции.

Нам сняться сны каждую ночь. Современные исследования говорят, что нам сняться сны от полутора до четырех часов за ночь. Каждую ночь наши сны как работники в банке, которые подводят баланс по всем счетам. Те части, которые мы отрицаем, во сне поднимают шум, что бы мы их заметили. Наши сны говорят нам об этих частях. Они пытаются привлечь наше внимание, что бы мы интегрировали их. Наши сны также могут говорить о тех частях, которые требуют актуализации. Иногда сны о смерти или умирании говорят нам о чем-то или о каком-то поведении, которое мы забрасываем. Такие сны могут сигнализировать о новом начале и новом творческом этапе в нашей жизни.

Все могут помнит свои сны. Большинству из нас не говорили об этом. Возьмите небольшую карточку и напишите на ней <<Я запомню свой сегодняшний сон.>> Положите её в карман. В течении дня взгляните на неё несколько раз, а главное, перед сном.

Заведите дневник рядом с вашей кроватью. Когда вы проснетесь и воспоминания ещё свежи, быстро запищите ваш сон. Запишите каждую деталь, т.к. все детали во сне важны. Язык сна это язык образов, а не язык логических мыслей. Сон всегда пытается сказать нам о том, чего наше сознание не знает.

Талмуд говорит: <<Сон это нераскрытое письмо к самому себе.>>
Работа со сном это серьезная работа. Большая ошибка думать, что её можно провести за небольшой промежуток времени. Образы из снов имеют свои ассоциации. Эти ассоциации являются нашими частями. Они должны быть интегрированы. Одна из ошибок работы со сном это чрезмерная интерпретация. Например, вера в то, что каждый раз, когда вам сниться оружие, это значит мужской сексуальный орган. Существуют целые книги о символах во сне. Они могут быть полезны, но не должны восприниматься слишком жестко. Также ошибкой будет только интеграция сна. В таком случае человек предполагает, что во сне не было символических ассоциаций и что все части сна это отщепленные и нереализованные части личности. Хорошая работа со сном включает в себя как интерпретацию, так и интеграцию.

Нужно действительно быть преданным этой работе и найти руководство. Мои любимые это руководство юнгианца Robert A. Johnson. Книга называется Inner Work, подзаголовок Using Dreams and Active Imagination For Personal Growth.

\ж{Первый шаг: Ассоциации}

После того, как вы аккуратно записали весь свой сон, включая все детали, вы готовы к выполнению первого шага. Помните, что язык сна отличается от языка нашего логического сознания. Мне приснился сон о полете на самолете. Я записал:
<<Я за штурвалом одномоторного самолета, в аэропорту Hobby. Я взлетаю, но не могу оторваться от земли. Я пытаюсь взлететь другим способом. Мне ясно, что я иду с юга на север, а потом с востока на запад, образую идеальный крест. Что бы я не делал, я не могу взлететь.>>

Я записываю первый образ. Полет на самолете (одномоторном). Для каждого символа во сне у бессознательного есть ассоциация. Язык бессознательного может быть декодирован. Задача в том, что бы разрешить ассоциациям приходить спонтанно.

Спросите себя, <<Что я чувствую об этом образе? Какие слова или идеи приходят на ум, когда я смотрю на него?>> Это значит, что вы буквально записываете все, что спонтанно ассоциируете с этим образом. Обычно, образ приносит несколько ассоциаций, как на рисунке 7.1.

После того, как вы исчерпали свои ассоциации, переходите к следующему образу. Мой следующий образ был аэропорт Hobby. Образ это человек, объект, ситуация, цвет, звук или речь. Я написал образ в центре круга на рисунке 7.2. Когда у меня закончились ассоциации, я перешел к другому образу. Я сделал тоже самое на рисунку 7.3, где я использовал образ пересекающихся полос. Для меня, невозможность оторваться от земли имело всего несколько ассоциаций. Я просто не мог оторваться от земли, не мог что-то сделать. Я также чувствовал, что застрял.

Что бы выбрать одну из ассоциаций, мы просто смотрим на них и ждем <<щелчка>>. <<Щелчок>> это нахождение наполненной энергией ассоциации.

Robert Johnson говорит: <<Один из способов найти суть сна, это идти туда, где энергия - ассоциация, которая вызывает всплеск энергии.>> Помните, отщеплённые части полны энергии. Также и символы бессознательного полны ей. Иногда, не совсем ясно, какая из ассоциаций имеет самый большой энергетический заряд. В таком случае, оставьте работу на какое-то время. Когда вы вернётесь к ней, вы найдёте источник.


\ж{Шаг второй: Соединение образов из сна с внутренней динамикой}

В этом шаге мы пытаемся соединить образы из сна с нашими внутренними частями. Что бы выполнить этот шаг, мы снова возвращаемся к образам. Мы спрашиваем: <<Какая это из моих частей? Кто внутри меня чувствует или ведёт себя подобным образом? Где находиться это моя черта личности?>> Затем записываем те примеры, какие вам приходят на ум.

Большинство, но не все, снов выражают внутреннюю жизнь спящего. Поэтому мы записываем примеры из нашей жизни, которые соответствуют событию из сна.

В моем случае я чувствовал себя застрявшим. Мне было около 45 лет, когда мне приснился этот сон. Я застрял в своей жизни. Мне нужно было найти какой-то баланс. Мандала это символ баланса и полноты. Я не знал куда податься. Недавно я добился определенного финансового успеха, но это не удовлетворяло меня. Я был на высокой точке в моей карьере, но чего-то не хватало. Сон был репрезентацией моего внутреннего состояния. Он также содержал в себе ключ к решению. Слово Торонто было загадкой для меня. Торонто это город, где я учился в священники. Была определенная духовная ассоциация с Торонто и я действительно чувствовал себя застрявшим духовно.

\ж{Шаг третий: Интерпретация}

Моей интерпретацией было то, что сон был глубоким подтверждением недостатка моей духовной неудовлетворенности. Хотя я и читал лекции по христианской теологии, что-то во мне было неудовлетворено в духовном плане. Я застрял и потерял баланс.


\ж{Шаг четвертый: Ритуалы}

Я рассказал об этом сне своему лучшему другу, о своей духовной бесплодности. Я разрешил моему сознанию сфокусироваться на Торонто. Я просто принял это слово и позволил мыслям свободно течь, что привело меня к прошедшим дня в семинарии в Торонто. У меня все ещё не было ответа на эту загадку.

Однажды, мне снова приснился тот же одномоторный самолёт. В этот раз я с лёгкостью оторвался от земли. Я летел в Торонто. Первое, что всплыло перед глазами после приземления, это настоятель, которого я встретил 20 лет назад в Траппистском монастыре в Нью-Йорке.

Я работал с этим сном и понял, что моё бессознательное пыталось мне сказать, что медитация была ключом к разблокированию моей заблокированной духовности. Я был глубоко тронут тем траппистским монахом много лет назад. Я потратил множество часов в медитации, когда учился на священника, но я не получил никакой выгоды от этого. Позднее, я прочитал книгу Robert Johnson <<He>>, и понял, что я был слишком молод и неопытен, что бы медитировать.

В книге <<He>>, Robert пишет о мифе о Перцифале. Он интерпретирует поиск Святого Грааля как миф, описывающий развитие мужественности. Он предполагает, что все мужчины, как Перцифаль, в подростковом возрасте находятся в Замке Грааля. Все молодые мужчины имеют могучее виденье, но они слишком молоды, что бы знать как актуализировать его.

В Торонто я учился на священника. Я провел первые монашеские годы в Рочестере, Нью-Йорк. В это время я и встретил траппиского настоятеля. Он был самым духовным деятелем из всех, с кем я был знаком. Со временем, я забыл про него. И вот, мне 45 лет, кризис среднего возраста и мои сны привели меня назад в Торонто, к этому настоятелю из Траппиского монастыря.

У меня было ещё несколько промежуточных снов, которые помогли мне прийти к полному осознанию смысла всего этого. В последующие несколько месяцев я осознал, что моя жизнь была кончена, если я продолжу стремится только к деньгам и мирскому престижу. Для меня стало ясно, что я должен был начать медитировать, если хотел двигаться дальше и расширить своё сознание и творческие силы.

Одни из моих достижений было нахождение в совете директоров нефтяной компании. Одна часть меня - очень стыдливая часть - любили быть членом совета. Но другая сторона чувствовала себя не в своей тарелке. Эти сны помогли мне начать медитировать.

Почти случайно в Лос Анджелесе я встретил Dennis Weaver. Я направлялся на консультацию по наркотикам. Денис был нашим приглашенным лектором о медитации. Денис медитировал с 1959 года, иногда по несколько часов в день. Он был лидером движения само-актуализации в Лос Анджелесе. Я был глубоко впечатлен его глубиной и внутренним спокойствие, излучаемым им. Вскоре после этого я начал медитировать и это привнесло большие изменения в мою жизнь.

Я вернулся в Торонто. Я бродил по местам, в которых бывал ранее. Я молился и медитировал в той же самой часовне, что и раньше. Произошли большие изменения. Мои сны привели меня к новой жизни. Мои сны зашли далеко в лечении моего стыда.


\ж{О любви к себе}

<<Ваш не нужно быть любимым, только не за счёт себя. Единственное взаимоотношение, которое действительно является центральным и важным в жизни, это взаимоотношение с собой. Из всех людей, которых вы будете знать за свою жизнь, вы тот, кого вы никогда не потеряете.>> Jo Courdet, Advice From A Failure.

Самым большим врагом токсического стыда является заявление о любви к себе. Сказать <<Я люблю себя>> может стать самым мощным инструментов в лечении связывающего вас стыда. Настоящая любовь к себе полностью трансформирует вашу жизнь.

\ж{Выбирая любовь к себе}

Scott Peck определил любовь <<Как волю к расширению себя ради обеспечения своего и чужого духовного роста.>> Это определение видит любовь как акт воли. Это значит, что любовь это решение. Я могу выбрать любить себя, не зависимо от того, что было в прошлом и того, что я чувствую о себе.

\ж{Упражнение: Чувственное ощущение себя}

Проведите подобный эксперимент. Представьте себя сидящим в вашем любимом кресле. Устройтесь поудобнее и расслабьтесь. Теперь закройте глаза и представьте, что человек, которого вы в настоящий момент любите и уважаете больше всего, сидит напротив вас. Этим человеком может быть супруг, любовник, ребёнок, родитель, герой и т.п. Закройте глаза и представьте этого человека, 3-4 минуты.

Теперь соприкоснитесь с тем чувством, которое проявляется, когда вы находитесь рядом с этим человеком. Я чувствую тепло, жизненность и благодарность, когда я вижу своего лучшего друга. Это моё чувственное ощущение этого взаимоотношения.

Теперь закройте глаза и представьте себя, сидящего напротив вас, 3-4 минуты.

В первый раз, когда я проделал это упражнение, я почувствовал, как я начал критиковать себя. Это иногда происходит и сейчас, когда я смотрю на себя в зеркале. Просто отметьте чувства, когда вы смотрите на самого себя. Один человек, с которым я работал, видел свои щеки слишком полными, а позу - искаженной. Большинство из нас имеют определённые негативные чувства о самих себе. Если вы стыдливы и ничего не делали для излечения своего стыда, вы скорее всего почувствуете сильное отторжение. Отторжение себя это корень токсического стыда.



\ж{Безусловное принятие себя}

Что бы противодействовать этому негативному чувству о себе, примите решение безусловно принимать себя. Вы делаете это через акт выбора.

\ж{Я люблю себя. Я буду безусловно принимать себя}

Говорите это часто и вслух.
Я отчетливо помню, когда впервые по-настоящему безусловно принял и полюбил себя. Это было потрясающе! Позднее я прочитал книге Gay Hendricks <<Learning To Love Yourself>> в которой он говорил о том же. Он описывал, как спрашивал у людей на своих воркшопах простой вопрос: <<Будете ли вы любить себя за это?>>

По-началу, когда я читал диалог одного из его терапевтических воркшопов с группой людей, я был озадачен. Несомненно, существуют вещи, которые мы делаем, не заслуживающие любви. Gay все продолжал, задавая человеку вопрос, будет ли он любить себя несмотря на то, что он сделал или не сделал, и я понял, что наша любовь должна проявляться в сторону того, кто мы есть, а не за то, что мы делаем. Вы достойны любви, без всяких но.

Понимание различий между быть и делать это одно из самых больших открытий для меня. Я так сильно старался достигать и делать все лучше и лучше, но что бы я ни делал, я все равно чувствовал глубокое ощущение дефективности, что является знаком интернализации стыда. Слова <<Я люблю себя не зависимо...>> это мощное противодействие голосу стыда. Слова <<Я безусловно принимаю себя>> могут изменить наши жизни.

Самого большого терапевтического успеха я достигал с проблемой веса у женщин. Успех пришёл как результат следующего упражнения. Женщина думала, что у неё лишний вес в 12 килограмм, относилась с презрением к своему телу и принижала себя сравнениями и навешиваниями ярлыков. Я работал с ней несколько месяцев, непрерывно противодействуя её сравнениям и принижениям вопросом <<Будешь ли ты любить и принимать себя за это?>>

Не зависимо от того, что она говорила, я задавал ей этот вопрос. Постепенно она начала принимать себя такой, какая есть. Я отказывался говорить с ней о диетах и упражнениях. Я знал, что пока она не примет себя такой, какая она есть, она никогда не изменится. Она не могла сбросить вес постоянно пристыжая себя. Как можно решить проблему, которая создается и мотивируется токсическим стыдом, усиливая токсический стыд? Каждый раз, когда она сравнивала себя или принижала негативным ярлыком, она начинала стыдливую спираль. Стыдливая спираль усиливает токсический интернализированный стыд и это настраивает её на обжорство, как на способ ослабить свою боль, вызванную стыдом. Таким образом навешивание ярлыков и одиозные сравнения это способ сохранить избыточный вес, а не избавиться от него.

Что бы излечить связывающий вас стыд, вы должны начать с самопринятия и самолюбия. Любовь создает союз. Когда мы принимаем решение безусловно любить себя, мы начинаем безусловно принимать себя. Это тотальное самопринятие приводит к искуплению. Мы едины с собой. Вся наша сила доступна нам, потому что мы перестаем рассеивать нашу энергию на охрану голодных собак в подвале.

Выбор любить себя возможен, даже если у нас есть негативные чувства о самих себе.

Мне часто не нравится поведение одного из моих детей, но это не значит, что я перестал их любить. Если мы принимаем решение безусловно любить себя, мы начнем по-другому чувствовать себя.

Когда мы выбираем любить себя, наша самоценность будет повышена. Много лет назад Sidney Simon и Kirschenbaum написали книгу Values Clarification. Они предположили, что ценность не является ценностью, если у неё не присутствует следующих семи факторов:
\begin{itemize}
\item она должна быть свободно выбрана
\item она должна быть выбрана рассмотрением альтернатив
\item она должна быть выбрана с ясным пониманием последствий
\item её должны ценить и лелеять
\item о ней должно быть публично объявлено
\item мы должны действовать согласно ей
\item мы должны неоднократно действовать согласно ей
\end{itemize}

Выбор любить себя это свободный выбор. Это простое решение. Альтернатива это стыдливый образ жизни с катастрофичными последствиями. Я призываю вас говорит <<Я люблю себя>>, вслух, что бы объявить, что вы безусловно любите и принимаете себя. Если вы неоднократно будете действовать согласно этому убеждению, ваше самолюбие и самоценность будет расти и укрепляться.

\ж{Уделяйте себе время и внимание}

Если вы принимаете решение любить себя, значит вы должны быть готовы уделять себе время и внимание.

Определение любви Scott Peck предполагает, что любовь это тяжелый труд. Он включает в себя расширение; это значит, мы должны расширить себя. Расширение себя требует работы.

Работа над любовью требует времени. Сколько времени вы уделяете себе? Уделяете ли вы время для надлежащего отдыха и релаксации или вы немилосердно загоняете себя? Если вы <<человек делающий>> то вы точно загоняете себя. Вам нужно все больше и больше достижений, что бы чувствовать себя нормально. Если вы готовы любить и принимать себя, вы позволите себе выделить время что бы просто быть. Вы выделите время, когда вам ничего не нужно делать и никуда не нужно идти. Вы позволите себе уединится. Вы найдете время для гигиены и упражнений. Найдете время для веселья и развлечений. Вы будете брать отпуска. Вы найдете время для работы над своей сексуальной жизнью. Вы позволите себе получать удовольствие и наслаждение.

Работа над любовью это работа над прислушиванием к себе. Вы слушаете себя, наблюдая за чувствами, потребностями и желаниями. Вам нужно обращать внимание на себя. Это может значить научиться техникам для работы со своими чувствами. Это может значить вступить в группу, где вы можете получить обратную связь. Работы над прислушиванием к себе требует дисциплины.

И снова, как говорит Scott Peck, дисциплина позволит нам усилить удовольствие от жизни. Если вы любите себя, вы готовы отложить вознаграждение, что бы что-то способствующее вашему росту могло произойти.

Когда я был стыдливым алкоголиком, я не мог даже подумать о том, что бы отложить удовлетворение. Как большинство детей травм и дисфункций, я никогда не думал, что чего-то может быть достаточно. Я не мог отложить удовлетворение, потому что моя стыдливая личность не верила, что я снова смогу это получить.

Дисциплина требует правды и ответственности за свою жизнь. Если я люблю себя, я буду жить реальностью. Я буду говорить правду и быть ответственным. Такое поведение усиливает мою самооценку. Я люблю подобное поведение в других, почему бы мне не любить его в себе?

В сообществе восстановления мы говорим <<Подражай, пока не получиться.>> Иногда, мы должна решить действовать правильным способом, что бы начать чувствовать. Это относиться и к самолюбию. Примите решение. Произнесите вслух. Действуйте, будто вы безусловно любите, цените и принимаете себя, и вы начнёте чувствовать больше самолюбие и принятия.

\ж{Самоутверждение}

Другая работа любви, которая усилит ваше самолюбие и излечит токсический стыд это настойчивость. Настойчивость основывается на самолюбии и самоценности. Это отличается от агрессивности. Агрессивность, обычно, поведение повязанного стыдом. Быть агрессивным значит победить любой ценой. Это часто включает в себя принижение другого человека. Принижение другого не может усилить самолюбие.

Я рассматриваю обучение самоутверждению и настойчивости как один из самых мощных способов излечения связывающего стыда. В процессе интернализации стыда в дисфункциональной семье, ваши потребности стали повязаны стыдом. Через какое-то время вы уже не знали, что вам нужно. Не было возможности научиться просить, что вам требуется. С нарушением вашей потребности в зависимости, вы пришли к выводу, что вы не можете ни на кого полагаться. Вы потеряли всякое чувство ваших человеческих прав как совершенно уникального и неповторимого человеческого существа.

Обучение настойчивости это способ научиться удовлетворять эти потребности. Вы учитесь тому, как говорить нет и просить то, чего вы хотите. Вы учитесь строить новые физические, эмоциональные, волевые и интеллектуальные границы.

Такие книги как When I Say No, I Feel Guilty - Manuel Smith,
Do You Say Yes When You Want To Say No? - Fensterheim и Your Perfect Right - Alberti и Emmons, показывают способы обучение тому, как отстаивать свои права и удовлетворять свои потребности. Представленные в этих книгах методы требуют практики.

Каждому из нас нужно создать свой Билль о правах. Нам нужно иметь полное разрешение в исполнении наших прав. Manuel Smith устанавливает следующий список прав. Вы можете добавить к нему свои варианты.

\begin{itemize}
\item У вас есть право судить своё поведение, мысли, эмоции и принимать полную ответственность за их исполнение и последствия.
\item У вас есть право не предоставлять никакого объяснения или оправдания своему поведению.
\item У вас есть право осуждать, если вы ответственны за осуждение проблем других людей.
\item У вас есть право менять своё мнение.
\item У вас есть право совершать ошибки и быть ответственным за них.
\item У вас есть право сказать <<Я не знаю.>>
\item У вас есть право быть независимым от доброжелательности других прежде чем взаимодействовать с ними.
\item У вас есть право быть нелогичным в принятии решений.
\item У вас есть право сказать <<Я не понимаю.>>
\item У вас есть право сказать <<Мне все равно.>>
\end{itemize}

When I Say No, I Feel Guilty



Подумайте о том, как вы любили того человека в упражнении. Если бы кто-либо обижал его или изводил его, что бы вы сделали? Если бы вы увидели, как он вредит и пристыжает себя, что бы вы сделали, что бы позаботиться о нем? Подумайте о работе и энергии, которую вы оказали из любви к своим детям? Будете ли вы любить себя таким же образом? Вы действительно заслуживаете этого. Никогда не существовало такого человека как вы и никогда не будет существовать. Вы уникальны, неповторимы и драгоценны.

\ж{Ошибки рефрейминга}

Стыдливый человек отчаянно пытается показать миру маску, говорящую <<Я больше, чем человек>> или <<Я меньше, чем человек.>> Быть более чем человеком значит никогда не совершать ошибок. Быть менее чем человеком, значит быть ошибкой. Здоровое отношение к ошибкам невероятно важно для поддержания самолюбия. Рефрейминг наших ошибок это способ справится с ними.

Рефрейминг значит изменение интерпретации или точки зрения. Вы накладываете новые рамки на образ или событие для изменения вашего взгляда на него. Новые рамки изменят смысл этого события. Рефрейминг ошибок значит думать о них в таком смысле, что бы удалить их катастрофические качества. Вместо катастрофичности, мы начинаем видеть ошибки как естественный и ценный компонент жизни. Это именно цель здорового стыда. Когда вы находитесь в контакте со здоровым чувством стыда, вы знаете, что вы можете и будете делать ошибки и используете ошибки как повод для обучения или как предупреждение, что бы остановиться и взглянуть на свои действия.

\ж{Ошибки как предупреждение}

Ошибки это как сигнал в вашем автомобиле, который предупреждает вас об опасности вождения с не пристегнутым ремнем. Если вы получите штраф за превышение скорости, это может быть предупреждением, что бы ездить медленнее и концентрироваться на вождении. Подобные ошибки в последствии могут спасти вам жизнь.

Токсический стыд с его маской перфекционизма трансформирует предупреждение в моральное обвинение. Вы настолько заняты защитой себя от внутреннего голоса, что пропускаете возможность прислушаться к предупреждению этой ошибки. Заведите привычку рефреймить ошибки в предупреждения. Фокусируйтесь на предупреждении, нежели на виновности.

\ж{Ошибки как разрешение спонтанности}

Знание того, что вы можете и будете совершать ошибки, позволяет вам проживать свою жизнь спонтанно. Здоровый стыд является условием для творчества. Знание того, что будете совершать ошибки позволяет вам искать новую информацию и новые решения. Он уберегает вас от мысли, что вы все знаете.

Страх ошибок убивает творческий потенциал и спонтанность.

\ж{Ошибки как учителя}

Не существует способа научиться чему-либо не совершая ошибки. Процесс обучения был определен как <<метод последовательного приближения>>. Понаблюдайте за тем, как дети учатся ходить. Они буквально учатся этому через падения. После каждого падения они немного регулируют свой баланс и пробуют снова. Каждая ошибка создает последовательное приближение и в конце-концов они могут ходить.

Ошибки эта форма обратной связи. Каждая ошибка говорит о том, что нам нужно исправить. Исправляя каждую ошибку, мы все ближе подходим к рабочей поведенческой последовательности.

Как учитель я знаю, что студенты, которые бояться делать ошибки, имеют проблемы с обучением. Они боятся заниматься новым материалом из-за возможности не справиться с ним. Такие студенты соглашаются на первую предложенную работу и часто остаются на ней на всю жизнь. Они слишком боятся найти новую работу, потому что им придется столкнуться с новыми правилами и препятствиями. Они не хотят проходить курсы по повышению квалификации, потому что неизбежные ошибки слишком болезненны.

И снова McKay и Fanning выразили это лучше всего:
<<Представление ошибок как необходимой обратной связи в процессе изучения позволяет вам расслабиться и сфокусироваться на постепенном овладении новым умением. Ошибки это информация о том, что работает, а что нет. Они не имеют ничего общего с вашей ценностью или интеллектом. Они просто шаги к цели.>>

\ж{Категории ошибок}

Существуют распространенные категории ошибок. 10 самых частых:

\begin{enumerate}
\item Ошибка данных. Вы неправильно записали номер телефона.
\item Ошибка суждения. Вы решили купить более дешевую обувь, которая потеряла свой вид за 6 месяцев.
\item Легкая ложь. Вы сказали, что больны и наткнулись в магазине на босса.
\item Прокрастинация. Вы продолжаете откладывать поход к дантисту. Наступили выходные и у вас сильная зубная боль.
\item Забывчивость. Вы отправляетесь за покупками и забыли дома все деньги.
\item Упущенные возможности. Золото, которые вы решили не покупать по 48 долларов за унцию, теперь стоит 432 доллара.
\item Злоупотребление. Вы съели слишком много чего-то и теперь у вас болит живот.
\item Трата энергии. Вы работаете над рукописью озаглавленную Places In The Heart и в этом время с точно таким же названием выходит фильм.
\item Неудача в достижении цели. Вы на летнем отдыхе и все ещё не избавились от лишнего веса.
\item Нетерпеливость. Вы пытаетесь закинуть рыбу в лодку и она срывается с крючка.
\end{enumerate}


Существует множество других категорий. Общее во всех этих примерах то, что ошибки всегда являются продуктом ретроспективности. McKay и Fanning пишут:
<<Ошибка это что-то, что вы сделали, о чем позднее, оглядываясь в прошлое, хотели бы сделать по-другому. Это также относиться к чему-то чего вы не сделали, а позднее решили, что должны были сделать.>>

Ошибка всегда определяется после самого факта её исполнения.

Мы всегда выбираем действие, которое, как нам кажется, вероятнее всего удовлетворит наши потребности. Нам кажется, что преимущества перевешивают недостатки. Действия в любой момент времени зависят от нашего осознания. McKay и Fanning определяют осознание как следующее:
<<Осознание это степень ясности, с которой вы воспринимаете и понимаете, сознательно или бессознательно, все факторы, относящиеся к конкретной потребности.>>f

Ошибки это результат поздней интерпретации. Следовательно, ошибки не имеют ничего общего с самооценкой. Если вы оцениваете свой выбор как <<плохой>>, потому что он был ошибочным в свете позднего осознания, вы наказываете себя за действия, которые вы не могли не сделать. Лучше оценивать свои прошлые ошибки как <<неразумные>>, <<вредные>> или <<не аффективные>>. Эти термины будут более точными в оценке ваших действий.

Расширение осознания является очевидным следствием проблемы ошибок. Если вы часто совершаете ошибки, вам стоит рассмотреть вариант расширения осознания как подход к решению этой проблемы. Это наиболее полезное решение. Клятва не совершать снова подобные ошибки это не самый лучший вариант, т.к. вы все равно их будете совершать, если не расширите своё осознание.

Идея того, что вы всегда принимаете самое лучше доступное вам решение, не освобождает вас от ответственности за ваши ошибки.


Ответственность значит принятие последствий ваших действий. У каждого действия есть последствия. Стать более ответственным значит расширить своё осознание на счёт последствий наших действий.


С экстернализацией стыда также повыситься ваше осознание. Стыдливый человек имеет очень низкий уровень осознания, т.к. блокированные и связанные эмоции искажают способность мыслить и быть в курсе. Интернализированный стыд формирует туннельное виденье, которое сужает осознание, что проявляется как искаженное мышление. С излечением стыда через различные процессы экстернализации, осознание увеличивается.

\ж{Привычка осознания}

McKay и Fanning описывают простую процедуру, которую они называют <<Привычкой осознания>>. Они предлагают определённые вопросы, которые вы задаете себе, что бы рассмотреть вероятные последствия, как краткосрочные, так и долгосрочные, определённых принимаемых решений. Вот эти вопросы:

\begin{itemize}
\item Был ли я в подобной ситуации ранее?
\item Какими будут негативные последствия моего решения?
\item Учитывая то, что я получу от этого решения, стоит ли оно этих последствий?
\item Имеются ли какие-либо альтернативные решения с меньшими негативными последствиями?
\end{itemize}

Главным ингредиентом привычки осознания является приверженность своему решению. Вы обязываете себя рассматривать возможные последствия всех важных действий. Это решение любить себя. Это решение уделить время для взвешивания и оценки последствий вашего выбора. В конечном итоге, ткань жизни состоит из ваших решений.


\ж{Излечение воспоминаний и изменение вашего образа перед самим собой}

<<Сейчас я покажу вам как вы можете изменить вашу личную историю>> - воскликнул тренер David Gordon. Я сглотнул и подумал, <<Минуточку, я знаю, что мы можем изменять будущее своими новыми решениями, но изменить прошлое? Да ладно!>>

\ж{Изменение личной истории}

David Gordon был моим основным тренером в новой волнующей модели изменения, называемой Нейро-лингвистическим программированием (НЛП). Техника, называемая изменением вашей личной истории была одной из тех, что я тогда изучил и в последствии использовал на протяжении последних шести лет. Из всех техник в этой книге, я рассматриваю эту как одну из самых мощных для исцеления связывающего вас стыда.

Далее следует моя адаптация этой техники. Я называю этот процесс <<Возвращение горячей картошки>>. Лучше всего она работает с воспоминаниями, связанными с опекунами (родителями, учителями, священниками), которые пристыжали вас, использую одну из межличностных стратегий передачи стыда.


\ж{Природа якорей}

Сама техника основана на феномене человеческого программирования, который в НЛП называют якорем. Якорь это как кнопка <<Вкл>> на стерео, которое проигрывает старую запись из памяти. Якорь запускает звуки, образы, чувства и даже запахи и вкусы старых воспоминаний. Слова же являются якорями. Слова это триггеры, которые стимулируют образы и чувства из воспоминаний.

Фактически, если вы взгляните на картину 9.1 вы увидите, что реальность, которую мы переживаем, можно представить только так, как мы о ней говорим. Я имею в виду, что мы не можем передать наш опыт именно так, как это на самом деле происходило. Когда мы говорим о нем, мы говорим о способе интерпретации опыта по средству двух систем - сенсорного восприятия и интеллектуального познания.

Сенсорное восприятие это наш первичный и непосредственный способ познания. Наше интеллектуальное познание всегда немного отдалено от реальности. Философ Gottfried von Leibniz говорил о том, что концепции (интеллектуальное знание) всегда опирается на восприятие (сенсорное знание). Каждая наша мысль несет в себе сенсорную информацию. Каждая мысль о которой мы размышляем, была в первую очередь воспринята, увидена, услышана, потрогана, распробована или учуяна. Концепции запускают сенсорные образы - визуальные картины, аудиальный внутренний диалог или чувственные (кинестетические) реакции.

Когда мы говорим о токсическом стыде, то многие воспоминания вызываются бессознательно. Эти стыдливые воспоминания часто опутаны коллажем из образов. При интернализации стыда, эти образы часто запускаются и отправляют стыдливого человека в стыдливую спираль. Кажется, что эти спирали действуют независимо от нас.

Стыдливая спираль также запускается внутренним диалогом. Такие диалоги основаны на наших старых убеждениях о самих себе и мире. Эти убеждения были созданы нашими стыдливыми опекунами. Аудиальная стыдливая спираль является результатом интроекции родительского голосов, который изначально были голосом пристыжающих опекунов. Они проигрываются как стерео записи. Терапевты трансакционного анализа предполагают, что таких записей в нашей голове на 25 000 часов.

<<Возвращение горячей картошки>> - это способ изменения старых образов через использование кинестетических якорей. Это форма пере-проживания прошлого с корректирующими ресурсами. Это также способ отдать то, что Pia Mellody называет индуцированным стыдом.

Когда опекун ведет себя бесстыдно через ярость, осуждения, критику, то мы берем на себя стыд, который они сами пытаются избежать. Фактически, это наш стыд, т.е. мы на самом деле переживаем пристыжение их бесстыдным поведением. Мы принимаем их осуждения как будто это про нас, когда в реальности это все о них. В этом смысле мы несем их стыд.

Что бы понять эту технику, позвольте мне процитировать Leslie Bandler, одного из пионеров и создателей НЛП.

<<Основная предпосылка моей работы в том, что у людей есть все ресурсы, требуемые для тех изменений, которые они хотят совершить... Эти ресурсы находятся в личной истории каждого из нас. Каждое наше переживание может служить нам как актив. Почти каждый из нас переживал уверенность в себе, смелость или расслабленность. Задача терапевта в том, что бы сделать эти ресурсы доступными в том контексте, в котором они нужны. Bandler,
Grindler, Delozier и я создали метод, называемый якорением, которые именно это и делает.>> They Lived Happily Ever After.


Leslie продолжает, объясняя, что определенный стимул, как старая песня, может вернуть прошедшие переживания, так и мы можем научиться намеренно ассоциировать воспоминание с определенным переживанием. Мы можем сделать это получив доступ к воспоминанию и одновременно коснувшись пальцев во время пере-проживания этого воспоминания. После установления ассоциации, мы сможем таким же касанием пальце воспроизвести это переживание.

Язык работает таким же образом. Около десяти лет назад я сидел на встрече с другом и увидел, как он плачет. Я спросил в чем дело и он ответил: <<Баффи умерла.>> Баффи была его собакой. Когда я это услышал, я подумал, что это странно, взрослому мужчине плакать из-за смерти какой-то собаки. Слово собака не имело важной ассоциации для меня. У меня никогда не было собаки. ребёнком я их боялся, т.к. был разносчиком газеты, а собаки были нашими врагами. Я не мог понять чей-то плач над собакой, т.к. у меня не было приятных тёплых сенсорных воспоминаний о собаке.

8 лет назад я купил моему сыну крошечного шелти. Мы назвали его Калли. Когда я прихожу домой, Калли встречает меня веселыми прыжками. Не важно когда или сколько раз я прихожу, Калли всегда рад меня видеть. Я очень привязался к этой собаке. Теперь у меня есть сенсорное переживание и я знаю, почему люди плачут, когда теряют своих собак. Теперь это имеет смысл для меня. Если кто-то скажет, что его собака умерла, то слово <<собака>> немедленно вызовет мои переживания о Калли. Слова это якоря, которые вызывают прошлые сенсорные переживания. Leslie Bandler пишет: <<Если я попрошу вас вспомнить, когда мы испытывали истинное удовлетворение собой, мои слова отправят вас на поиск, через ваши прошлые переживания... вы знаете, как можете разозлиться, вспомнив старый аргумент или испугаться, вспоминая ужасный инцидент. Таким образом погружаясь в воспоминания (внутренне генерируемое переживание) мы пере-проживаем множество тех чувств, которые возникли, когда это воспоминание было сформировано.>>

В моей версии техники НЛП я попрошу вас выбрать прошлое стыдное воспоминание и заякорить его. Для этого нужно просто закрыть глаза и позволить вашим воспоминаниям вернуть вас назад во времени, когда ваша Мама, Папа или учитель перекладывал на вас их стыд.

Один из моих клиентов помнил, как был пристыжен во втором классе. Он учился в церковно приходской школе, а священник был тем, кто раздавал табели детей. У этого священника была привычка кидать на пол табели, если ученик получал оценку D или F. У моего клиента была не диагностированная дислексия и он испытывал большие проблемы с чтением. Он получил F по чтению и священник бросил его табель на землю. Мой клиент испытывал стыд и унижение и почему-то не смог поднять свою табель. Все смеялись над ним в этот мучительный момент стыда. Это воспоминание, как и все не диссоциированные болезненные воспоминания, было легко заякорить. Он заякорил его двумя пальцами левой руки.

Затем я спросил его, каких ресурсов ему тогда не хватило, но которые у него есть сейчас, которые позволили бы ему пережить это событие лучшим образом. Он подумал и ответил: <<Теперь я ясно выражаюсь и научился быть настойчивым.>>

Я ответил: <<Закрой глаза и подумай о времени, когда ты ясно себя выражал. Воспоминание может прийти о любом времени твоей жизни. Ты выражаешь себя четко и ясно, говоря точно то, что хочешь сказать.>>

Во время поиска этого воспоминания, лицо клиента изменилось. Его челюсть расслабилась и он выглядел более уверенно. Я попросил его коснуться пальцами правой руки и держать их вместо в течении 30 секунд, пока он пере-проживает свою уверенность. Затем я попросил его сделать глубокий вдох и расслабиться. Далее я попросил его вспомнить приятное событие из прошлого, что бы разделить переживание вербальной уверенности от якоря, который мы создадим позднее - якорь напористости.

Вскоре я сказал: <<Теперь подумай о времени, когда ты был действительно напористым.>> Я подождал, что бы он полностью погрузился в детали этого воспоминания. Далее я сказал: <<Кто присутствовал тогда? Во что они были одеты? Во что был одет ты?>> Во время его пере-проживание напористости, я попросил его коснуться пальцами правой руки, точно также, как он это сделал с предыдущим якорем. Через 30 секунд я попросил его сделать глубокий вдох, расслабиться и вернуться к приятному воспоминанию.

До этого момента мы сделали:
1. Создали стыдливый якорь (X) сцены с табелем во втором классе с помощью прикосновения пальцев левой руки.
2. Создали ресурсный якорь (Y) с помощью прикосновения пальцев правой руки. Этот якорь воплощает в себе две сильные стороны, которых не было у моего клиента во втором классе, в этом случае это вербальная экспрессивность и напористость.

Сейчас мы готовы переделать старое воспоминание. Помочь понять этот процесс может предпосылка из кибернетики. Эта предпосылка утверждает, что мозг и центральная нервная система не может различить реальное и воображаемое переживание, если воображаемое переживание представлено достаточно живо и детально.

Большинство людей может достичь сексуального возбуждения при помощи своего воображения. Это значит, что присутствует полная кинестетическая реакция без присутствия другого человека. Параноидальные личности живут в угрожающем и сверхбдительном состоянии только при помощи своих фантазий и галлюцинаций. Нормальные люди часто создают ужасный стресс и тревогу, волнуясь на счёт будущего - того, что ещё даже не случилось. Это все способы программирования поведения просто используя воображение.

На третьем этапе этого процесса я прошу клиента вернуться назад в стыдливую ситуацию с новыми ресурсами вербальной экспрессивности и напористости. Это осуществляется одновременным прикосновением и активацией двух якорей X и Y. Я прошу клиента не изменять ничье поведение. Он должен сфокусироваться только на своей реакции. Он свободен реагировать любым способом, каким он хочет, используя его вербальную экспрессивность и напористость. Я призываю его сказать священнику все, что он чувствует.

Лучше всего, когда слова приходят спонтанно. Их можно проговорить вслух или про себя. Это должно быть энергичное выражение гнева о стыде.

Наконец я прошу человека отдать бесстыдному опекуны обратно стыд, который он нес для него все эти годы. Обычно я символизирую его как черный сырой мешок. Символическая отдача очень важна. Когда переживание улучшается и человек внутренне чувствует себя лучше, я прошу его сделать глубокий вдох, расслабиться и открыть глаза.

\ж{Хорошо сформированные якоря}

Эту работу можно проводить одному и каждую сцену можно прорабатывать много раз. Я лично работал над более чем 100 стыдливыми воспоминаниями. Некоторые сцены я перерабатывал десятки раз. Ключевой фактор этой работы это хорошие ресурсные якоря. Стыдливый якорь обычно не сложно сформировать из-за его высокой заряженности болью. Достижение хорошо сформированного ресурсного якоря потребует времени, практики и терпения.

Условия для хорошо сформированного якоря таковы:

\begin{enumerate}
\item Чистое состояние доступа - Это значит, что лучший якорь это тот, который заключает в себе самую высокую энергетическую заряженность. Нужно выбирать самое интенсивное ощущение чувства.
\item Своевременное применение - Нам нужно устанавливать якоря, когда энергия переживания ближе всего к пику.
\item Может быть продублировано - Мы можем проверить, создали ли мы хороший якорь протестировав его. Когда мы снова коснемся пальцев, запуститься прошлое переживание. Если энергетика переживания высокая, значит якорь сформирован хорошо.
\end{enumerate}

Последний пункт решающий. Всегда проверяйте ресурсный якорь хотя бы один раз перед работой над исправлением переживания.

Несколько особенно интересных вещей об этом методе. Первый это то, что человек использует свои настоящие ресурсы. Это решающий фактор для стыдливых созависимых людей, у которых такое слабое осознание своих внутренних сил, которые убеждены, что им нужна помощь снаружи. Использование личных ресурсов человека это и есть хорошая терапия. Сила уже находиться в том, кому мы хотим помочь. Все мы уже владеем нужными ресурсами для изменений, но токсический стыд блокирует осознание этой силы.

Второе, что мне нравится в этой технике, это то, что её можно протестировать. Ближе к концу работы с упомянутым ранее клиентом, я попросил его расслабиться, закрыть глаза и вернуться во второй класс, в тот самый день с инцидентом. Я попросил его активировать самый первый якорь - якорь со стыдом. Я попросил его прочувствовать прошлое стыдливое переживание и обратить внимание на любые изменения в переживании. Клиент сообщил о существенном изменении в переживании, что также было заметно по выражению его лица.

Когда мы в первый раз создавали якорь, его голова наклонилась вперед, он нахмурился, дыхание было частым, а щеки покраснели. Когда мы проводили тест, после двух рабочих сессий, его голова была в вертикальном положении, дыхание расслабленным, а цвет кожи оставался без изменений. Эти неврологические сигналы соответствовали сообщению клиента о пере-проживании старой боли.


\ж{Возвращение горячей картошки}

\begin{enumerate}
\item Закройте глаза и в течении 3-5 минут фокусируйтесь на своём дыхании. Следите за входящим и выходящим воздухом и полностью расслабьтесь. 
\item Позвольте вашему разуму вернуться назад во времени к стыдливому переживанию в присутствии другого человека. Когда вы почувствуете досаду и расстройство этим переживанием, дотроньтесь вашим большим пальцем левой руки до любого другого пальца этой руки. Держите их вместе 20 секунд. Сделайте глубокий вдох, разожмите пальцы и расслабьтесь. Перенесите ваше сознание в какое-нибудь знакомое место, например, ваш дом.
\item После этого подумайте о ресурсе или нескольких ресурсах, доступные вам сейчас, которые будь они у вас в той стыдливой ситуации, могли бы вам помочь справится с ней по другому. 
\item Подумайте о времени, когда вы использовали требующие ресурсы (реальное событие из вашей жизни) и войдите в эти воспоминания, с как можно большим числом деталей. Во что вы были одеты? Какова цвета волосы были у другого человека, глаза и т.д.?
\item Когда вы почувствуете требуемый ресурс, коснитесь большим пальцем правой руки до другого пальца этой руки. Держите в течении 30 секунд. Сделайте глубокий вдох и разожмите пальцы. Повторите процесс с любым другим нужным ресурсом, который бы вам помог в прошлом стыдливом событии.
\item Перенесите ваше сознание в какую-нибудь настоящую знакомую сцену (например в вашу машину).
\item Теперь представьте, что вы готовитесь вернуться в прошлую стыдливую ситуацию. Представьте, что вы могли бы вернуться во времени с новыми ресурсами, которые вы только что заякорили. Представьте, что вы собираетесь пере-прожить то переживание, используя заякоренные ресурсы.
\item Теперь одновременно дотроньтесь до ваших двух якорей (большой палец правой руки до пальца и большой палец левой руки до другого пальца). Вернитесь в стыдливое событие и переделайте его. Скажите пристыжающему человеку, как вы злы на него и что ещё вы ходите сказать или сделать. (Не меняйте ничьего поведения - только своё). Оставайтесь в переживании до тех пор, пока ваше внутренней состояние не измениться. Если у вас возникнут с этим сложности, то вернитесь в настоящее и заякорите больше ресурсов. Затем снова вернитесь в прошлое с новыми ресурсами. Помните, что нужно отдать обратно чужой стыд - стыд, который они избегают своим бесстыдным поведением.
\item Подождите пару минут, затем снова вспомните прошлое переживание без якорей, что бы обнаружить, действительно ли сенсорное переживание изменилось.
\item После изменения прошлого переживания, шагните с этим в будущее. Представьте подобную же ситуацию в будущем. Представьте себя в ней с вашими ресурсами. В этот раз не нужно использовать якоря.
\end{enumerate}

\ж{Изменение самооценки}

Другая мощная техника, используемая мной на протяжении многих лет называется Самооценочное мышлением. Я обязан Stephen и Carol Lankton за ядро этого процесса. Они были блестящими подопечными Milton Erickson, который вдохновил их на эту технику.

В их книге The Answer Within: A Clinical Framework Of Erickson
Hypnotherapy, Lanktons представили свой процесс изменения собственной самооценки. Далее следует моя адаптация их работы.

Наша самооценка это как линза камеры. Она обуславливает то, как мы видим и взаимодействуем с миром. Этот фильтр, который определяет границы нашего переживания и возможного выбора. У стыдливых людей негативная самооценка. Они видят себя как дефективных личностей. Часто, стыдливый человек даже не в курсе своей негативной самооценки, т.е. её скрывает прикрытие ложного <<Я>>. Мы настолько идентифицируемся с нашей Ролью или Сценарием, что больше не осознаем наших глубинных чувств о себе.

Изменение вашей самооценки потребует тяжёлой работы. Думайте об этом как о работе любви. Она потребует практики и приверженности. Основной процесс включает в себя визуализацию.

Я уже объяснял, как работает визуализация. Ваш мозг и центральная неравная система не различает реальные или воображаемые переживания. Визуализация будет работать несмотря на то, верите вы в неё или нет.

Каждый способен вызывать внутренней ощущение, но не каждый умеет визуализировать. Некоторым придется этому научиться.

Таким образом первый шаг в использовании визуализации для изменения самооценки это научиться визуализировать. Далее находится подготовка к выполнению упражнения самооценочного мышления.

\ж{Предварительный разогрев}

Первое, что нужно сделать это расслабиться. Самая эффективная визуализация возможно тогда, когда наш мозг вырабатывает альфа волны. Альфа волны это результат полного расслабления. Когда вы находитесь в расслабленном состоянии, ваш мозг находиться во власти состояния повышенной внушаемости.

\ж{Поместите следующую инструкцию на аудио запись}

ядьте в удобном месте. Удостоверьтесь, что ваша голова имеет опору, а температура воздуха комфортна. Начните фокусироваться на вашем дыхании. Обычно, вы не осознаете своего дыхания, т.е. с помощью фокусировке на нем вы делаете бессознательное сознательным. Осознавайте то, что происходит с вашим телом, когда вы дышите. Осознавайте входящий и выходящий воздух. Обратите внимание на различие в этом воздухе. Сфокусируйтесь на этом различии. Теперь представьте, что вы видите пар на выдохе - как при выдохе в холодную погоду.

Начиная со лба, отметьте, есть ли там напряжение. Если есть, на выдохе снимите его. Представьте напряжение как черные точки в паре при выдохе. Дышите пока пар полностью не очистится.

Далее сфокусируйтесь на области вокруг глаз. Если область напряжена - также продышите его. Далее сфокусируйтесь на лицевых мышцах. На шее. На плечах. На руках. На груди. На животе. На ягодицах. На коленях. На икрах и лодыжках. На ступнях.

Теперь позвольте расслабиться всему вашему телу. Представьте, что стали как пустая бамбуковая стрела. Теплая золотая энергия проходит через макушку вашей головы через все тело и выходит из пальцев ног. Позвольте расслабиться каждой клетке вашего тела.

Далее представьте черную цифру семь на белом полотне или наоборот, как вам проще всего. Если у вас это не получается, то нарисуйте цифру семь до начала этого упражнения. Практикуйте рассматривание этого рисунка, а затем, закрыв глаза, представьте эту картину на внутреннем экране.

Начните с представления цифры семь, затем цифры шесть, пять, четыре, три, два, один. Сфокусируйтесь на цифре один. Позвольте единице стать огнем свечи. Приглядитесь к оранжевым, красным, желтым и синим аспектам пламени. Посмотрите на центр пламени; почувствуйте тепло и услышьте потрескивания огня. Позвольте пламени стать теплым огнем в камине. Почувствуйте тепло; запах огня; услышьте треск. Представьте, что вы поджариваете зефир. Попробуйте зефир на вкус. Позвольте вашему языку и вкусовым рецепторам посмаковать этот вкус.

Теперь позвольте воображению унести вас в летний денек. Вы идете по тропинке... (пауза 10 секунд)... Вы чувствуете теплый ветер на своём лице... (пауза 10 секунд)... Слева от вас яблони полные больших красных яблок... (пауза 10 секунд)... Вы подходите и срываете одно из яблок... (пауза 10 секунд)... Вы кусаете его и чувствуете восхитительный сок... Далее вы видите апельсиновое дерево... (пауза 10 секунд)... Вы срываете апельсин. Вы чистите его, потирая пальцами внутреннею сторону кожицы... (пауза 10 секунд)... Вы вдыхаете аромат и пробуете его на вкус... (пауза 10 секунд)... Впереди вас ряды желтых цветов... (пауза 10 секунд)... Вы чувствуете их запах, приближаясь к ним... лютики, жимолость... (пауза 10 секунд)... ваши любимые цветы. Продолжайте идти, пока не достигнете белого песчаного пляжа... (пауза 10 секунд)... Снимите вашу обувь и позвольте ваши ногам почувствовать чистый белый песок. Почувствуйте, как он проходит между ваших пальцев... (пауза 10 секунд)... Посмотрите налево, на тропинку... (пауза 10 секунд)... Начинайте идти по ней... По обе стороны тропинки растут деревья... (пауза 10 секунд)... На деревьях поют и щебечут птицы... (пауза 10 секунд)... Услышьте ветер, прорывающийся через листья... В конце тропинки находиться небольшое озеро... Вы слышите всплески воды... (пауза 10 секунд)... Рыба выпрыгивает из воды... (пауза 10 секунд)... Через озеро вы видите стадо коров, подходящих к озеру. Вы слышите их мычание... (пауза 10 секунд)... Сядьте на берегу озера... Просто расслабьтесь и представьте самого себя, идущим к вам навстречу... Посмотрите на себя. Какого цвета ваши волосы? Посмотрите на свои глаза... уши... нос. Посмотрите на все ваше тело... Представьте, что повернулись спиной... посмотрите на спину. Повернитесь в сторону... Взгляните на свой профиль... Позвольте себе медленно уйти... Посидите на берегу озера и поразмыслите над вашим образом... Медленно начинайте считать назад от семи... Позвольте себе вернуться в бодрствование, когда вы досчитаете до единицы...

Возможно, у вас были проблемы с виденьем своего образа. У большинства стыдливых людей это так. Если это так, вы можете сделать следующее:

\ж{Упражнение телесного образа}

Первая часть этого упражнения делается с открытыми глазами. Встаньте перед зеркалом во весь рост: исследуйте своё лицо, лоб, брови, глаза, щеки, улыбку, рот, родинки, стрижку и её цвет. Попрактикуйте улыбку и серьезный взгляд. Станьте экспертом своего собственного лица.

Теперь сделайте тоже самое с остальными частями своего тела. Просканируйте вниз и изучите свою шею, плечи, руки. Ознакомьтесь со своей грудью и животом, бедрами и ногами. Повернитесь и осмотрите, насколько это возможно, другую сторону своего тела. Отметьте свою позу. Встаньте прямо и медленно упадите вниз. Промаршируйте на месте, размахивая руками. Вы можете сделать пару фотографий, что бы рассмотреть себя в разных позах. Это не критическая оценка. И это не время для инвентаризации тех частей тела, которые вы хотите изменить.

Когда вы станете экспертом своего собственного внешнего вида, вы готовы для второй части. Делайте это утром, находясь в кровати, перед тем как встать. Визуализируйте свой подъем. Услышьте будильник. Почувствуйте теплоту постели. Начинайте вставать и снова вернитесь под теплое одеяло. Затем встаньте с кровати. Почувствуйте своими ступнями холодный пол. Взгляните на свою комнату: вашу мебель, вещи, окна. Оденьтесь. Почувствуйте ткань на воем теле. Посмотрите на цвет одежды. Проведите свои обычные гигиенические процедуры - почистите зубы, умойте лицо, побрейтесь, зачешите волосы. Отметьте ощущение воды, запах косметики или бритвенного лосьона. Отметьте любой зуд или боль в вашем теле. Сделайте сцену как можно более живой и детализированной.

Напомните себе, что вы все ещё лежите в кровати. Откройте глаза и на самом деле сделайте все то, что только что представили. Осознавайте реальные ощущения и сравните их с воображаемыми. Внимательно отметьте любые различия. Делайте это упражнение каждое утро на протяжении недели. Каждый раз добавляйте те детали, которые ранее упускали. Неделя этого упражнения значительно обогатит вашу визуализацию.

После того, как вы освоите это, вы подошли к Центральному Образу Себя (ЦОС). Практикуйте это, закрыв глаза и представив свой ЦОС. Вы можете делать это во-время своего дня: на работе, а автобусе, на обеде, в любое время когда у вас есть свободная минута.

\ж{Самооценочное мышление}

Шаг первый: Интегрирование новых сильных сторон в ваш ЦОС
Шаг включает в себя рассмотрение своего ЦОС. Если есть что-то, что вы хотели бы изменить (вашу осанку, энергетику, одежду, выражение лица) убедитесь, что это можно изменить. Подумайте о ком-нибудь, кто улыбается так, как вам хотелось бы улыбаться. Когда вы видите его улыбку, заякорите её также, как мы делали это в упражнении с передачей горячей картошки. После якорения, закройте глаза и представьте ЦОС. Посмотрите на своё лицо и активируйте якорь желаемой улыбки. Держите якорь до тех пор, пока у вас не появится эта улыбка. Вы можете повторять этот процесс множество раз. Вскоре вы сможете просто закрыть глаза и видеть эту желаемую улыбку в вашем исполнении. Вы можете сделать тоже самое с осанкой, уверенностью, глазным контактом, вашей походкой и речью. Вы также можете видеть себя тяжелее, легче, или с желаемым поведением. Самое важное это искать эти черты характера или поведение беспристрастным способом.

Шаг второй: Добавление ближнего человека
Второй шаг подразумевает вовлечение в значимые взаимоотношения. Ваша стыдливая идентичность образовалась в результате разрушения межличностного моста с ближним человеком. Хорошая идентичность требует хотя бы одной пары отражающих глаз. Поэтому, вам требуется кто-то, кому вы доверяете и считаетесь с его мнением, что бы он зеркалировал ваш новый образ.

Закройте глаза и представьте ЦОС с новой чертой теплой представительной уверенности. Вплывите в своё тело. Встаньте в свою обувь, смотрите через свои глаза, почувствуйте касание одежды на вашем теле.

Смотрите через свои глаза и представьте доверяющего, любящего и не осуждающего друга, выходящего к вам. Удостоверьтесь, что это такой друг, который всегда говорит правду. Пообщайтесь с ним и услышьте его обратную связь о вашей новой улыбке и уверенности. Посмотрите на его истинное одобрение вас в его глазах. Скажите себе: <<Я люблю свою новую улыбку и моё чувство уверенности. Я наслаждаюсь временем, проведенным с моим другом, а он - со мной.>> Аффирмации это отличный поддерживающий материал для работы с образом себя.

Шаг третий: Практика сценариев
В третьем шаге вы проходите через различные сценарии, относящиеся к поведению, которое вы хотите изменить. Например, я много работал над изменением моего угодничества. Я делал следующее:

Представьте, что вы разговариваете с хорошим другом. Ваш друг приглашает вас пойти в новый тайский ресторан. Он очень хвалит их кухню. Почувствуйте в себе желание сказать да и угодить своему другу. Обратите внимание на собственное тело. Вы устали и вам не очень нравится тайская кухня. Вы вспоминаете, как однажды вас чуть не вырвало, после того как вы отведали одно экзотическое тайское блюдо. Вы слышите свой голос, когда отвечаете ему: <<Нет, я устал и не люблю тайскую кухню. Мы можем сходить куда-нибудь завтра, но только не в тайский ресторан.>> Услышьте разочарование вашего друга. Пожмите плечами, сделайте глубокий вдох и скажите своему другу, что с радостью сходите с ним в какой-нибудь другой ресторан. Посмотрите ему в глаза и скажите, что не хотели бы врать и притворяться в его присутствии.
Когда он уйдет скажите самому себе: <<Я могу быть честен во взаимоотношениях. Я могу говорить то, что хочу и чувствую. У меня есть право любить и не любить. Я люблю себя, когда говорю правду.>>

Вы можете практиковать множество других типов сценариев. Например, сказать <<Я люблю тебя>> кому-либо; сказать <<Нет>>; попросить о прибавке; о приёме на работу; возвращении ненужного товара в магазин; о приглашении на свидание; встрече новых людей; об успешном проведение лекции. Выберите любую ситуацию, инфицированную вашим токсическим стыдом.

Lanktons предлагают использовать позитивные сценарии вместе с сильным ЦОС, для построения новых привычных шаблонов в преодолении стрессовых состоянии. Допустим, вы начинаете думать о походе в банк и прошении кредита. В это время ваши мускулы напрягаются; горло становится тугим, а душа уходит в пятки. При первых же знаках стресса, вы можете создать привычку сознательного фокусирования на сильном и уверенном ЦОС и на одном из успешных сценариев. Lanktons предлагают повторить процесс с шестью и более успешными сценариями. С укреплением этой привычки, она станет бессознательной. Вашей реакцией станет автоматическое ожидание позитивного результата.

Я хочу напомнить вам, что ваша стыдливая идентичность была сформирована негативными стыдливыми сценами и языковыми импринтами. Коллаж стыдных воспоминаний соединён друг с другом и действует бессознательно. В такой же манере (только позитивно) самооценочное мышление включает в себя постройку позитивной внутренней карты, которая будет направлять ваши действия.

Шаг четвертый: Исходящий образ
Цель четвертого шага в создании новых позитивных изменений из наших воображаемых позитивных целей. Lanktons называют это <<Исходящими образами>>. Вам потребуется войти внутрь воображаемых картин, в которых вы наслаждаетесь желаемой целью и у вас образуются новые фантазии, которые исходят из изначальной. В моем примере с отказом в угодничестве, я представил, что мой друг позвонил мне в тот же вечер, что бы сказать, что он ценит мою честность в отказе идти куда-то, куда я совершенно не хотел идти. Я слышал уважение в его голосе и мне было приятно.

Шаг пятый: Шаг в будущее
Финальный шаг по самооценочному мышлению это шаг в будущее. Достигаем мы этого притворяясь, что желаемая цель уже достигнута.

Это способ посмотреть назад, из воображаемого будущего и понять что-то важное о том, как сделать мечты реальностью.

Это мощные техники и они будут работать, не зависимо от того, верите вы в них или нет. Все что вы должны сделать, это самоотверженно практиковать их. Мы тратим столько энергии для прикрытия своего стыда. Если вы используете хотя бы небольшую её часть, то эти техники воображения могут изменить вашу жизнь.

\ж{Противодействие и изменение ваших личностей}

<<Мы все находимся в пост-гипнотическом трансе индуцированном в раннем детстве.>> Ronald Laing.


<<Каждый из нас сам делает себя либо несчастным, либо сильным. Объем работы, необходимой и в первом, и во втором случае, — один и тот же.>> Дон Хуан.

Я посмотрел на её лицо. Оно практически блестело. Она сияла, возможно самая красивая женщина из тех, которых я консультировал. Когда она шла по холлу в сторону моего офиса, я был поражен её женственностью и элегантностью. её красота была слишком ошеломляющей. Когда она начала говорить о себе, я подумал, что она выпрашивает у меня комплименты и похвалу, принижая себя.

<<Я ужасная мать. Мой ребёнок заслуживает лучшего. Скоро я потеряю работу. Я не могу разобраться с новой компьютерной системой. Я всегда была туповатой. Я не виню своего мужа из-за развода. Я должна была выйти замуж за Сидни. Он был слепым. Ему не пришлось бы смотреть на моё тело.>> Она продолжала и, кажется, её невозможно было остановить. Два раза тон её голоса изменился. Один был скрипучим, а другой плаксивым и слабым.

Я работал с ней на протяжении года. её звали Офелия. Возникла ясная картина заброшенности. её настоящий отец был алкоголиком. Он ушел от матери Офелии, когда ей было 3 года. её первый отчим обжигал спичками её кожу, говоря, что это было примером адского пламени. Другой мужчина, брат отчима, брал её покататься и делал с ней <<забавные вещи>>. Она сказала, что он был единственным, кто реально оказывал ей внимание. Однажды, он подарил ей щенка. Это было одно из редких приятных воспоминаний из детства. её мать была официанткой в баре и часто кричала на Офелию своим скрипучим голосом. В старших классах средней школы мать Офелии выбирали самой сексуальной девушкой школы и она постоянно сравнивала себя с Офелией. Когда ей было 13 лет, мать говорила ей: <<Тебе бы лучше нарастить грудь и задницу, если ты хочешь, что бы мальчики приглашали тебя на свидание. В твоем возрасте у меня уже был размер 36С.>> Один из ухажеров матери залез в кровать к Офелии и заставил её делать определённые сексуальные вещи. Офелия говорила об этом с нескрываемой болью, но также, будто это была победа над её матерью. В конечном итоге, эта женщина подверглась серьезному насилию. Она была глубоко стыдлива и поддерживала стыд с помощью негативного внутреннего диалога, который запускал непрерывную стыдливую спираль.

\ж{Внутренний голос}

Негативный внутренний диалог это то, что Robert Firestone называет <<внутренним голосом>>. Eric Berne упоминал его как набор родительских записей. Fritz Perls и школа Гештальта называет его <<интроецированным родительским голосом>>. Aaron Beck называет их <<автоматическими мыслями>>. Как не назови, у всех нас в голове есть эти голоса. У стыдливых людей доминируют негативные пристыжающие, самообесценивающие голоса.

Robert Firestone пишет:
<<Голос>> можно описать как язык коварного саморазрушающего процесса, который существует в каждом человеке. Голос представляет из себя внешнею точку зрения в отношении самого себя, изначально произошедшую из подавленных родителями враждебных чувств по отношению к ребенку.>> The Fantasy Bond.

В основном голос стыдливого человека говорит, что они не достойны любви, никчемны и плохи. Голос поддерживает плохой детский образ.

Голос может сознательно переживаться как мысль. Большинство из них частично сознательны или полностью бессознательны. Большинство из нас не осознает привычную активность этого голоса. Мы осознаем его в определённых стрессовых ситуациях раскрытия, когда активируется наш стыд. После совершения ошибки, человек может назвать себя <<дураком>>. Или сказать <<Ну вот снова. Я такой недотепа.>> Перед важным интервью голос мучает вас такими мыслями как <<Почему ты думаешь, что сможешь справиться с ответственностью на такой работе? Кроме того, ты слишком нервный. Они узнают какой ты нервный.>>

Избавиться от голоса невероятно сложно из-за изначального разрушения межличностного моста, который приводит к воображаемой связи. Чем больше в жизни заброшенного ребёнка насилия и пренебрежения, тем сильнее он создаёт иллюзию связи со своими родителями. Эту иллюзию Robert Firestone называет <<Фантазией Связи>>.

Что бы создать фантазию связи, ребёнок должен идеализировать родителей и сделать самого себя <<плохим>>. Цель этой фантазии связи в выживании. ребёнок отчаянно нуждается в своих родителях. Они не могут быть плохими. Если они плохие или больные, он не сможет выжить. Поэтому, фантазия связи (которая делает родителей хорошими, а ребёнка - плохим) это как мираж в пустыне. Она даёт ребенку иллюзию того, что в его жизни существует забота и поддержка. Годы спустя, когда ребёнок покидает своих родителей, фантазия связи устанавливается внутренне. Она поддерживается с помощью голоса. То, что изначально было внешним, родительский крик, ругань и карающий голос теперь становится внутренним. По этой причине, процесс противодействия и изменения внутреннего голоса вызывает серьезную тревогу. Firestone замечает, что <<Не существует глубоких терапевтических изменений без этой сопровождающей процесс тревоги.>>


В начале работы по противодействию и изменению внутреннего голоса важно, что бы вы понимали, каким сильным этот голос может быть. ребёнок, включивший родительский голос, берет на себя роль родителя, а в случае стыдливых родителей, искаженную точку зрения о себе. И как заметил Firestone, ребёнок включает в себя <<отношение родителей в их самом злом и отвергающем состоянии. Дочь и сын содержат в себе чувства омерзения и деградации, которые лежат за их заявлениями.>>

Дети в стыдливых семьях не могли ничего сделать, кроме как поверить, что они плохие и недостойные любви. Мы просто были не в состоянии понять, что наши родители были стыдливыми, нуждающимися и, в некоторых случаях, эмоционально больными. Голос также имеет тенденцию обобщать, двигаясь от конкретной критики к другим областям нашей жизни. Если мама переносила её стыд через компульсивный перфекционизм о близости и чистоте, то этот критический перфекционизм будет обобщен до всех остальных плохих привычек и личных дефектов. Дети будут относиться к себе и другим с таким же насмешливым сарказмом, с каким относились к ним их родители.

Голос это не положительная система ценностей. <<Скорее>>, как говорит Firestone, <<он интерпретирует и утверждает порочным образом внешнюю систему ценностей, через само-атаки и бичевание.>> Голос может быть противоречивым, инициируя определённые действия, а потом осуждая их.

Firestone предложил доказательства, что <<в своей самой патологической форме, суицидальные и смертоносные пациенты сообщают о переживании голосов как фактических галлюцинаций, инструктирующих их исполнить деструктивные импульсы.>> В менее патологической форме, когда голос представляет собой родительскую систему ценностей, тон голоса мстительный и ведет к ненависти к себе, нежели к исправлению поведения. Даже когда человек признает свою ошибку или вину, голос все равно остается праведным и карательным. Голос может сказать: <<Ты никогда не научишься. Ты такой слабый и неуклюжий. Признайся, ты просто чертовски плох.>> Голос принимает категорическое суждение, что человек дефективен и ущербен и никогда не измениться.

В основном голос состоит из стыдливых защитных механизмов первичных опекунов. Точно так, как стыдливые родители не могут принять свои слабости, желания, чувства, уязвимости и потребности в зависимости, они также не могут принять нуждаемость детей, их чувства, слабости, уязвимости и зависимости. Firestone пишет, что голос это результат <<подавлений глубинного желания родителей разрушить живость и спонтанность детей, каждый раз, когда он или она вторгается в их оборону.>>

Мы должны помнить, что стыдливые опекуны сами когда-то были обиженными детьми. Их боль, унижение и стыд был подавлен. Их гнев в сторону пристыжающих родителей не был выражен из-за страха их потери. Гнев был направлен во внутрь, против себя и превратился в ненависть к себе. Родительские защитные механизмы против их боли и стыда предотвращают извержение этих чувств в сознание. Если родитель позволит ребенку выражать эти чувства, это будет угрожать из собственной защите. Родитель должен остановить чувства нуждаемости и боли у ребёнка, что бы сам родитель не ощущал своих собственных чувств.

\ж{Внутренний голос как автоматические мысли}

Важно научиться обращать внимание на внутренний диалог, ваш собственный внутренний голос. Самым деструктивным аспектом внутреннего голоса называют автоматические мысли.

Представьте следующую ситуацию: на переполненном футбольном матче женщина вскрикивает, встает и даёт пощечину сидящему рядом мужчине, а затем убегает со стадиона. Несколько людей видели это. Каждый реагирует по-разному. Один испугался; подросток разозлился; мужчина средних лет подавлен; терапевту стало любопытно; священник смущен. Одно и тоже событие запустило совершенно разные эмоции в каждом свидетеле.

Причина лежит в автоматических мыслях каждого свидетеля. Испуганный мужчина ребёнком неоднократно получал шлепки от своей визгливой матери. Он слышал кричащий материнский голос: <<Что у тебя вместо мозгов?>>

Подросток подумал <<Женщинам позволяется бить мужчин. Также как моя сестра могла бить меня и её никогда не наказывали за это. Это не честно.>>

Недавно разведённый мужчина средних лет подумал: <<Неужели никто больше не ладит друг с другом? Это печально.>>

Терапевт подумал: <<Интересно, что он ей сказал, что бы вызвать такую реакцию?>>

Священник подумал: <<Неужели это одна из моих прихожанок? Как стыдно!>>

В каждом случае эмоции свидетеля были результатом мыслей. Эмоциональная реакция следовала за мыслями, которые интерпретировали событие. Наша ментальная жизнь битком набита мыслями, многие из них появляются бессознательно и автоматически.

Интернализированный стыд заставляет вас концентрироваться на определенной группе автоматических мыслей, и исключать все противоречащие мысли. Эта озабоченность создает что-то вроде туннельного виденья в котором вы думает только определенный вид мыслей и замечает только определённые аспекты вашей среды. Aaron Beck использует фразу <<избирательная абстракция>> при описании этого туннельного виденья. Избирательная абстракция значит, что мы смотрим только на один набор сигналов в нашей окружающей среде и исключаем все остальные. Туннельное виденье это продукт токсического стыда.

\ж{Противодействие внутреннему голосу}

Я надеюсь теперь ясно, что негативный голос культивирует и усиливает токсический стыд. Он инициирует и усиливает стыдливую спираль. После установки система внутреннего голоса, она становится главной динамикой автономного функционирования токсического стыда. Было разработано множество техник для противодействия и изменения голоса в нашей голове.

\ж{Адаптация работы с голосом по Firestone}

Robert Firestone проделал основополагающую работу по выявлению истоков деструктивности голоса. Он разработал несколько путей для привнесения этих враждебных мыслей к сознанию пациента. Он пишет, что <<процесс формулирования и вербализации негативных мыслей ослабляет деструктивный эффект голоса в поведении пациента.>>

В терапии голоса пациента учат экстернализировать внутренние критические мысли. Делая это, они раскрывают свою атаку на себя и, в конечном счете, развивают способы для изменения своего негативного отношения в более объективную, беспристрастную точку зрения. С экстернализацией голоса через вербализацию, происходит освобождение сильных чувств, результатом которого становится эмоциональный катарсис, сопровождающийся инсайтами.

Исторически, терапия голоса развилась из наблюдений Firestone за <<нормальным>> и невротическим поведением. В особенности он отметил, как группа так называемых нормальных терапевтов, рассердилась и заняла оборонительную позицию при упоминании определённых вещей о себе, которые они расценили критичными и негативными.

<<Их оборонительная позиция>>, пишет Firestone, <<обычно была не связана с точностью или неточностью получаемой ими обратной связи, но, казалось, совпадением c их собственной негативной самооценкой>>. Другими словами, чем сильнее и реактивнее была их защитная реакция, тем сильнее они, скорее всего, также критиковали себя. Firestone пришёл к выводу, что <<оценки от других, когда они подтверждают искаженное представление человека о самом себе, как правило, вызывают навязчивый мыслительный процесс>>. Из-за того, что мы сами мучим себя критическими мыслями и замечаниями, мы чувствуем угрозу каждый раз, когда кто-то другой атакует нас таким же способом.

\ж{Методы экстернализации пристыжающего голоса}

Методы Firestone в основном используются в контексте индивидуальной и групповой терапии. Я адаптировал эти методы для использования вне терапии. Я попробую показать вам, почему внутренний голос так силен и почему вы не хотите от него отказаться. Если при работе со следующими упражнениями вы почувствуете себя перегруженным - немедленно остановитесь Это значит, что вам требуется делать их с кем-нибудь, кто сможет вам помочь.

\ж{Дневник чрезмерной реакции}

Первый метод прямиком вытекает из ранних работ Firestone, в которых он тестировал запуск процесса навязчивого критического голоса. Он включает в себя ведение дневника ваших защитных чрезмерных реакций. Лучше всего это делать, когда вы вовлечены в какую-либо группу с обратной связью, но также работа возможна в контексте вашей обычной жизни.

Каждый вечер, перед сном, поразмышляйте о событиях прошедшего дня. Расстроило ли вас что-либо? Когда именно вы проявили чрезмерную реакцию? Какой был контекст? Кто присутствовал? Что было сказано вам? Как то, что было сказано вам отличается от того, что вы говорите себе?

Например, 10 декабря моя жена и я разговаривали о перестройке комнаты в нашем доме. В определённый момент разговора, я почувствовал, что тон моего голоса ускорился и усилился. Вскоре, я разглагольствовал о всем стрессе, который несёт моя текущая работа. Я слышал свои слова: <<Не ожидай, что я буду следить за этой работой. Я даже не могу справится со своими основными обязательствами>>. Позднее, я записал эту вспышку в свой дневник.
Я использовал следующую форму:

Дата: Среда, 10 декабря, 8:45.\\
Тема: Моя жена.\\
Содержание: Обсуждение реконструкции комнаты в нашем доме.\\
Чрезмерная реакция: После её слов: <<Мне нужна будет помощь.>> я ответил, во все более возбуждённом тоне: <<Не ожидай, что я буду следить за этой работой и т.д.>>\\
Голоса: Ты гнилой муж. Ты ничего не умеешь ремонтировать. Ты жалок. Твой дом разваливается. Настоящие мужчины знают как чинить и строить. Хорошие отцы следят за своим домом.

Важно уделить время этим голосам. Я рекомендую войти в расслабленное состояние в тихом месте. Действительно позвольте себе послушать то, что вы говорите самому себе. Запишите и произнесите это вслух. Будьте спонтанны в выражении голосов. Как только вы начнёте говорить вслух, вы можете удивиться внезапному автоматическому излиянию.

В групповой работе Firestone призывает человека эмоционально и вслух выражать свои чувства. Он говорит: <<Скажите это вслух>> или <<Действительно отпустите>>. Я призываю вас делать тоже самое. Ляпните спонтанно все что придёт на ум. Скажите это во втором лице. Позвольте себе войти в эмоциональный заряд, вызванный голосом.


\ж{Отвечая голосу}

Как только вы выразите голос, вы можете начать отвечать на него. Вы бросаете вызов как содержанию, так и диктату голоса. В записи моего дневника я ответил, что я хороший муж и предоставил прекрасный дом. Моя мужественность не зависит от выполнения чего-либо. Я тяжело работаю и могу позволить заплатить кому-нибудь за ремонта. Даже если бы я знал, что делать, я бы все равно нанял работника. У меня есть чем занять своё время.

Я повторил этот диалог на следующий день. Я всегда отвечаю как эмоционально, так и фактологически (логически). Firestone рекомендует действовать, сознательно не соглашаясь с голосом, или даже действовать полностью против него. В моем примере, я позвонил знакомому плотнику и сказал ему, что конкретно мне нужно и ушел из дома. Я играл в гольф, довольный, что могу позволить себе нанять людей для починки моего дома.

\ж{Выслеживание внутреннего критика}

Второй способ раскрытия стыдливого голоса пришёл из гештальт терапии. Я просто называю его выслеживанием внутреннего критика.

Внутренний самокритичный диалог идет у всех стыдливых людей. Это почти всегда происходит бессознательно. Следующее упражнение поможет вам сделать его более сознательным, и даст инструменты, что бы лучше интегрировать и принимать себя. Я взял это упражнение из книги Awareness - John O. Stevens.

Усядьтесь поудобнее и закройте глаза. Теперь представьте, что вы смотрите на себя, сидящего перед вами. Сформируйте какой-то визуальный образ вас, сидящего перед вами, возможно, как отражение в зеркале. Как этот образ сидит? Во что этот образ одет? Какое у него выражение лица?

Теперь начните молча критиковать этот образ себя, как если бы вы критиковали другого человека. Скажите себе, что вы должны или не должны делать. Начинайте каждое предложение со слов: <<Ты должен...>>, <<Ты не должен ...>> или их эквивалентом.
Создайте длинный список критических замечаний. Слушайте свой голос во время этого.

Теперь представьте, что поменялись местами с этим образом. Станьте эти образом и молча начните отвечать на эти замечания. Что вы говорите в ответ на эти критические комментарии? Каков тон вашего голоса? Как вы себя чувствуете, отвечая на критику?

Теперь снова поменяйтесь ролями и станьте критиком. Продолжая этот диалог, будьте бдительны на счёт того, что вы говорите и как вы это говорите, ваши слова, тон голоса и т.д. Делайте паузу и просто слушайте ваши собственные слова и позвольте себе пережить их.

Меняйте роли, когда вам этого хочется, но продолжайте диалог. Замечайте детали того, что происходит внутри вас. Замечайте то, что чувствуете физически в каждой роли. Узнаете ли вы кого-то, кого вы знаете в голосе, критикующем вас, который говорит: <<Ты не должен...>> Что ещё привлекает ваше внимание в этом взаимодействии? Продолжайте молчаливый диалог в течении нескольких минут. Замечаете ли какие-либо изменения?

Теперь посидите в тишине и проведите обзор этого диалога. Возможно, вы пережили какое-то расщепление или конфликт, какое-то разделение между вашей сильной, критичной и авторитарной частью, которая требует вашего изменения, и другой менее сильной частью, которая извиняется, уклоняется и приводит оправдания. Это как если бы вы были разделены на родителя и ребёнка. Родитель постоянно пытается взять все под контроль и изменить вас в кого-то другого, а ребёнок постоянно уклоняется от этих попыток изменения. Когда вы наблюдали за диалогом, вы могли заметить, что голос, который критиковал и требовал был похож на голос одного из ваших родителей или какой-то другой властной фигуры в вашей жизни.

Этот критикующий голос может быть активирован в любой ситуации уязвимости и раскрытия. После его активации, также в действие приходит стыдливая спираль. Очень важно экстернализировать этот внутренний диалог, т.к. это один из главных способов, которым вы продолжаете чувствовать себя разделенным и не принимающим себя. Это упражнение помогает сделать критический диалог сознательным. Это первый шаг в экстернализации голоса.

Второй шаг заключается в трансформации критического сообщения в конкретное поведение. Вместо <<Ты эгоистичен>> скажите <<Я не хочу мыть посуду>>. Вместо <<Ты тупой>> скажите <<Я не понимаю алгебру>>. Каждое критическое замечание это обобщение. Как таковые они не соответствуют действительности. Переводя эти обобщения в конкретные формы поведения, вы увидите реальную картину и сможете принять себя более сбалансированным и интегрированном способом.

Третий шаг предлагает превратить обобщения в позитивные утверждения, которые противоречат им. Например, вместо <<Я эгоистичен>> скажите <<Я бескорыстен>>. Важно вербализовать это утверждение и услышать себя, произносящего это. Я рекомендую прийти к близкому человеку (не пристыжающему) и вербализовать эти позитивные утверждения в его присутствии.

Остановка навязчивых стыдливых мыслей
Это упражнение адаптировано из работы Bain, Wolpe и Meichenbaum. Лучше всего останавливать первые появляющиеся мысли, которые запускают стыдливую спираль. Это четырехступенчатое упражнение является моей адаптацией работы Joseph Wolpe.

Техника заключается в прерывании стыдливых мыслей резкими командами и вводом новых мыслей, более самоутвержающих на их место. Вызывающие стыд мысли попадают в три категории: самопринижение; катастрофичные мысли о будущем; критичные и пристыжающие мысли о раскаянии и сожалении.

Пристыжающие мысли о будущей болезни и катастрофе может привести к хронической тревоге. <<Если бы>> я не сделал то-то и то-то это прямой путь к запуску стыдливой спирали. Самопринижение как <<я слишком стеснительный, что бы завести друзей или получить то, что мне нужно>> или <<я такой тупой>> - также способы запуска стыдливой спирали. Навязчивость по-поводу ваших неудач и ограничений запускает спираль, которая приводит к серьёзной депрессии. Чем сильнее навязчивость - тем сильнее спираль. Цель остановки мыслей в том, что бы остановить спираль в самом её начале.

Сделайте паузу и запишите пять ваших самых пристыжающих мыслей. Например, следующие пять мыслей, над которыми я работал несколько лет назад.

\begin{enumerate}
\item Твои штаны так туго облегают тело, это просто отвратительно. (Одержимость весом)
\item Как отец я неудачник. (Одержимость родительскими обязанностями)
\item Я думаю, что я серьёзно болен. (Одержимость физическими болезными)
\item Какой в этом смысл. Я все равно умру. (Одержимость смертью)
\item Ты очень эгоистичен. (Одержимость моралью)
\end{enumerate}

Попробуйте найти те мысли, которые постоянно всплывают и пристыжают вас. Проранжируйте эти мысли по степени вызываемой ими тревоги и стыда. Выберите одну из мыслей для работы, не обязательно самую негативную. В первый раз вам нужно получить положительный опыт по изменению мыслей, поэтому, желательно выбрать мысли, которые, как вам кажется, вы можете изменить.

Остановка мыслей требует серьёзной приверженности быть постоянно начеку. Вы не можете просто желать остановить мысль, нужно самому её <<выпроводить>>. Это включает в себя концентрацию на пристыжающих мыслях, а затем внезапное отключение и освобождение вашего разума. Далее привожу четыре шага для остановки навязчивых пристыжающих мыслей.

\ж{Шаг первый: Представьте мысль}

Закройте глаза и создайте ситуацию, в которой скорее всего возникнет навязчивая мысль. Позвольте себе проникнуться этой ситуацией. Если у вас не получается визуализировать, ощутите чувство, которое вызывает пристыжающая мысль. Или услышьте голос, говорящий вам эту мысль. Представьте себе как можно больше деталей об этой сцене: вашу одежду, цвета, запахи, чувства, голоса других людей в сцене...

Теперь начните отслеживать цепь мыслей, которая у вас тогда возникла. Погрузитесь в разговор с самим собой. Будьте полностью в этой сцене, перед началом второго шага. Если вы чувствуете стыд, это хороший знак, потому что если вы добровольно можете усилить стыдливые чувства, значит вы также можете их ослабить.

\ж{Шаг второй: Прерывание мысли}

Прервать мысль можно при помощи множества техник испуга. Для этого подойдут различные таймеры. Я предпочитаю использовать диктофон. Включите его и запишите себя, говорящего <<СТОП>> через разные промежутки времени (не менее 2 минут). Сядьте или лягте и расслабьтесь. Закройте глаза и погрузитесь в стыдливую ситуацию. Действительно почувствуйте все детали этой ситуации. Включите свою запись, когда начнёте разговор с самим собой. Когда услышите слово <<СТОП>>, следующие 30 секунд вам ум должен быть чист. Потом попробуйте снова. Цель провести 30 секунд с чистым умом, без мыслей.

\ж{Шаг третий: Прерывание мыслей без посторонней помощи}

Теперь найдите место, где вас не побеспокоят, когда вы будете кричать слово <<СТОП>>. Заведите будильник на 3 минуты. Позвольте себе снова вернуться к навязчивой мысле. Позвольте себе пережить её со всеми сопутствующими чувствами. Когда зазвучит сигнал будильника, крикните <<СТОП>> и заметьте, как долго ваш ум будет свободен от болезненной мысли. Повторите снова. Вместо крика <<СТОП>> вы можете использовать прыжок, щелчок пальцами, удар по столу или мой любимый вариант - щелчок резиновой лентой по запястью.

Когда вы сможете останавливать мысль на 30 секунд и более, переходите на повседневный голос. Снова заведите будильник и повторите упражнение. Повторяйте это, пока не сможете останавливать мысль на 30 секунд и более. Теперь переходите на шёпот.

Когда сможете останавливать мысль шёпотом, переходите на произнесение слова <<СТОП>> про себя. В этот раз не заводите будильник, просто говорите про себя <<СТОП>> каждый раз, когда в ваш ум входит навязчивая мысль. 

\ж{Шаг четвёртый: Замена мыслей}

Теперь вы можете останавливать мысли сразу после их появления. Какого бы успеха вы в этом не добились, ваш ум не будет пуст больше чем на минуту. Природа не любит пустоты, так что в течении минуты старые мысли вернуться, если вы не замените их новыми позитивными мыслями.

Вот несколько примеров фраз, которые вы можете говорить себе: <<Это огорчительно, но не опасно>>, <<Ты живёшь по одному дню за раз>>, <<Ты делаешь один шаг за раз>>, <<Сделай глубокий вдох, сделай паузу, расслабься>>, <<Скоро все закончится; ничего не вечно. Позволь этому пройти>>, <<Попрощайся со своим прошлым; тебе можно все забыть>>, <<Посмотри на то, что тебе нравится в себе>>, <<Все мы не идеальным>>, <<Что бы быть не идеальным требуется смелость>>, <<Выполни сегодня одно задание и все будет в порядке>>.

Это просто предложение. Выбирайте понравившиеся и придумайте свои варианты.

Эти позитивные мысли называют скрытыми утверждениями. Техника замены мыслей, после их остановки, скрытыми утверждениями была впервые разработана Meichenbaum. Он называл это <<Stress Inoculation Training>>.

Эффективные утверждения напоминают вам о вашей силе по контролю над стыдливой спиралью и стыдливыми реакциями. Утверждение с <<Ты>> обычно более эффективно, чем <<Я>>. Постановка утверждения во втором лице налагает определённую дистанцию, что предполагает определённую степень внешнего контроля.

Еще одна важная деталь утверждений заключается в фокусировании на фактах. Боль в моей груди обычно указывает на газы, а не надвигающую ишемию, посланную Богом в наказание за мои слабости.

Прежде всего помните, что наш стыдливый внутренний диалог усиливался много лет. Вы должны практиковать остановку стыдливых мыслей и замену их скрытыми утверждениями. Это умение, и как все умения, его развитие требует времени. У вас будут неудачи, поэтому начните не с самых пристыжающих и сложных мыслей. Также советую использовать резиновую ленту на запястье и щелкать ей каждый раз, когда вы говорите про себя <<СТОП>>. Далее вы заменяете остановленную мысль скрытым утверждением.

\ж{Работы Albert Ellis и Aaron Beck}
Следующее это моя адаптация работы Albert Ellis и Aaron Beck. Эти люди внесли большой вклад в наше понимание того, как изменять производящие стыд мысли и внутренний диалог. Хотя я и не согласен с Ellis в том, что все чувства напрямую связаны с мыслями или внутренним диалогом, я верю, что его техники хороши в работе над техническим обслуживанием наших стыдливых концепций о себе и искаженном мышлении, которое они производят.

Наша стыдливая идентичность основывается на убеждение, что мы ущербные и дефективные личности. Подобные убеждения являются фундаментом стыдливого мышления, которое становится эгоцентричным туннельным зрением, состоящим из следующих типов искажений.


\ж{Стыдливое искаженной мышление}

\ж{Катастрофизация}

Головная боль, которая сигнализирует об опухоли мозга. Памятка о встрече с боссом значит, что вас уволят. Катастрофизация это результат отсутствия границ или чувства достоинства. Не существует границ того <<что если>>, которое может произойти.


\ж{Чтение мыслей}

Чтение мыслей значит, что вы делаете предположение (без доказательств) о том, как люди реагируют на вас. <<Я могу сказать по их лицам, что они готовы уволить меня>>. <<Она думает, что я незрелый, иначе она бы не задавала мне подобных вопросов>>. Эти предположения обычно рождаются из интуиции, догадок, смутных предчувствий или прошлого опыта. Чтение мыслей зависит от проекции. Вы представляете, что люди думают о вас также плохо, как вы думаете о себе. Как стыдливый человек, вы очень критичны к себе. Вы предполагаете, что другие также относятся к вам.

\ж{Персонализация}

Стыдливые люди эгоцентричны. Я сравниваю это с зубной болью. Когда у вас болит зуб, все о чем вы можете думать, так это о нем. Вы становитесь зубоцентричным. Таким же образом, если разорвана ваша личность и вы чувствуете боль, вы становитесь эгоцентричным.

Стыдливые люди принимают все на свой счёт. Вышедшая замуж женщина каждый раз когда её муж жалуется на усталость, думает, что он устал от неё. Мужчина, жена которого жалуется на выросшие цены на продукты, слышит атаку на его способность быть кормильцем.

Персонализация включает в себя привычку постоянного сравнения себя с другими людьми. Это последствие перфекционисткой системы, которая способствует стыду. Перфекционистская система требует сравнения. <<Он лучший организатор, чем я>>. <<Она знает себя лучше, чем я>>. <<Он видит все глубже. Я действительно поверхностен>>. Сравнения никогда не кончаются. В основе лежит предположение, что твоё достоинство сомнительно.


\ж{Сверхобобщение}

Искажение, которое стало результатом грандиозности токсического стыда. Один неправильный шов значит <<Я никогда не научусь шить>>. Отказ в свидании значит <<Никто никогда не захочет пойти со мной на свидание>>. В этом искажённом мышлении вы делаете широкое, обобщённое заключение, основанное на одном инциденте.

Сверхобобщение ведёт к универсальным классификациям, таким как <<Никто меня не любит... Я никогда не найду работу... Все всегда будет сложно... Почему я никогда не могу сделать все правильно?... Никто не полюбит меня, если по-настоящему узнают меня...>>. Другие сигнальные слова: все, каждый.

Другая форма сверхобобщения это номинализация. В номинализации процессы превращаются в вещи. <Мой брак болен>> - это номинализация. Брак это динамический процесс. Только некоторые его аспекты могут быть проблемными, а не весь брак. Недавно я слышал классический пример. Парень говорил: <<Эта страна идет на дно>>. Эта страна включает в себя множество динамических процессов и людей. Некоторые аспекты из этих аспектов напрягают этого человека, но вся страна не является сущностью.

Сверхобобщения способствуют все большей ограниченности образа жизни. Они представляют грандиозный абсолютизм, который предполагает, что некоторые непреложные законы управляют вашим счастьем. Подобная форма искаженного мышления также усиливает стыд.

\ж{Черно-белое или поляризованное мышление}

Другим следствием стыдливой грандиозности становится поляризованное мышление. Главным знаком подобного искаженного мышления является настаивание на дихотомическом выборе: вы воспринимаете все через крайности. Не существует середины. Люди и вещи или плохи и хороши, чудесны или ужасны. Самым разрушительным аспектом этого искажения является его влияние на то, как вы судите себя. Если вы не блестящи и безошибочны, значит должны быть полным неудачником. Нет места ошибкам.

\ж{Быть правым}

Как стыдливая личность, вы должны непрерывно доказывать, что ваша точка зрения и действия правильны. Вы живете в совершенно оборонительной позиции. Так как вы не можете совершить ошибку, вы не заинтересованы в правдивости мнения других людей, только в защите своей правды. Это искажение мышления держит вас в клетке, т.к. вы очень редко слышите новую информацию. Вы не получаете новой информации, которая могла бы помочь изменить ваши убеждения о себе.

\ж{Долженствования}

Karen Horney писала о тирании долженствования. Долженствование это прямой результат перфекционизма. В этом искаженном мышлении вы оперируете с помощью списка негибких правил о том, как вы и другие должны себя вести. Одна клиентка сказала мне, что её муж должен брать её на воскресные поездки. <<Любой мужчина, который любит свою жену, должен катать её загородом, а потом отвести на ужин в хорошее место>>. Тот факт, что её муж не хотел этого делать, значило, что он был эгоистичен и <<думал только о себе>> Самыми распространёнными словами-сигналами этого искажённого мышления являются: должен, обязан. Стыдливый человек с подобным искажением делает свою жизнь и жизнь других несчастной.

\ж{Заблуждения контролирующего мышления}

Контроль это крупное прикрытие токсического стыда. Контроль это продукт грандиозности и искаженного мышления. Вы видите себя как беспомощного и внешне контролируемого или как всемогущего и ответственного за всех вокруг вас. Вы не верите, что у вас есть реальный контроль над исходом вашей жизни. Это держит вас в стыдливом цикле.

Противоположное заблуждение это заблуждение в могущественности контроля. Вы чувствуете себя ответственным за все и всех. Вы несете на своих плечах весь мир и чувствуете вину, когда у вас это не получается.


\ж{Когнитивный дефицит или фильтрация}

В этом искажении мышления вы выбираете элемент или ситуацию и исключаете все остальное. Выбранная вами деталь поддерживает убеждения о вашей личной дефективности. Одного из моих клиентов, который был неплохим консультантом по менеджменту, один раз высоко оценили за его отчет по маркетингу. Его босс спросил, сможет ли он сделать следующий отчет в более короткий срок. Мой клиент впал в депрессию. Когда я его спросил об этом, я понял, что он был одержим тем фактом, что его босс говорил о нем как о ленивом. Он совершенно пропустил мимо ушей похвалу, в стыдливом страхе о своей дефективности.

Фильтрация это способ усиления ваших мыслей. Это запускает мощную стыдливую спираль.

\ж{Обвинения и глобальное навешивание ярлыков}

Обвинения это прикрытие стыда и способ его передачи другим. Обвинения трансформируются в глобальное навешивание ярлыков. В продуктовом магазине есть гнилая еда. Цены просто грабеж. Сдержанная и тихая девушка на свидании просто тупая простофиля. Ваш босс безвольный идиот.

Обвинения и глобальное навешивание ярлыков это способ отвлечься от собственной боли и ответственности. Это искажает мышление и держит вас подальше от честного взгляда на себя и ощущения своей боли. Ваша боль поможет вам измениться.

\ж{Экстернализация искажений мышления}

Что бы начать работу со стыдливыми искажениями мышления, вам нужно вернуться во время, когда вы переживали болезненный стыдливый эпизод. Следующий трехступенчатый процесс поможет идентифицировать ваши искаженные мысли в этой ситуации. Это также поможет реструктуризовать ваше мышление.

Три ступени:
\begin{enumerate}
\item Письменно опишите вызывающую стыд ситуацию или событие.
\item Определите ваши искажения мышления.
\item Реструктуризуйте и устраните стыдливое мышление через переписывание искажения.
\end{enumerate}

Третий шаг вызывает у людей больше всего проблем. Наши искажения настолько укоренились, что у нас вызывает проблемы найти более логический способ мышления. Далее следует руководство для логической коррекции искаженного мышления.

\ж{Катастрофизация}

Самым логичным измерителем катастрофизации будет честная оценка с точки зрения реалистичности шансов или доли вероятности. Каковы шансы? Один из тысячи (0,1\%), один из десяти тысяч (0,01\%) или один из ста тысяч (0,001\%)?

\ж{Чтение мыслей}

Чтение мыслей это форма воображения и фантазии. В долгосрочной перспективе вам лучше не делать выводов о людях. Смотрите на все ваши интерпретации о людях как на галлюцинации. Используйте это слово, когда вы интерпретируете. Говорите <<Моя фантазия или галлюцинация заключается...>>. Лучшей политикой будет поиск доказательств вашего заключения.

\ж{Персонализация}

Заставляйте себя искать доказательства значения хмурого взгляда босса. Проверяйте их, если это возможно. Забудьте о привычке сравнения. Не делайте никаких выводов, если только у вас нет разумных свидетельств и доказательств.

\ж{Сверхобобщение}

Используйте технику трех колонок для сверхобобщений.

\begin{tabular}[c]{|l|l|l|}
\hline
Доказательства моего заключения & Доказательства против... & Альтернативное заключение\\\hline
 & & \\
\end{tabular}

Напишите на карточке (3 Х 5) <<Абсолютов не существует>> и положите её на свой стол. Бросайте вызов таким словам как - все, каждый, никогда, всегда через их преувеличение. Говорите: Действительно ли я имею в виду, что я никогда-никогда и т.д.? Учитесь использовать слова: возможно, иногда, часто.

Проверяйте на номинализацию спрашивая себя можете ли вы положить это в тачку. Вы не можете положите брак или страну в тачку.

\ж{Поляризованной мышление}

Используйте карточку <<Абсолютов не существует>>. Черно-белое мышление это тоже форма абсолютизма. Это основа грандиозного мышления. Не существует черного или белого мышления. Мир серый. Думайте в процентном выражении. Примерно 5% времени я эгоистичен, но остальное время я щедрый и любящий.

\ж{Быть правым}

Если вы всегда правы, вы перестаете слушать и учиться. Ключ к преодолению стремления всегда быть правым, это активное слушанье. Carl Rogers проделал пионерскую работу в развитии этого умения. Как активный слушатель вы слушаете как ради содержания, так и ради самого процесса. Вы учитесь слушать своими ушами, так и глазами. Вы учитесь давать обратную связь и уточнять.

Вот пример. Мужчина рассказывает мне о его прекрасном детстве и отличном отце. Он описывает, как его отец работал с утра до вечера. Он допускал это, потому что его отец так тяжело работал, у него никогда не было времени для сына. Он даже не мог прийти на его игру на звание чемпиона. В этой части рассказа, дыхание мужчины изменилось; его руки напряглись и он отвел от меня взгляд. Я мягко сказал ему, что я услышал. Я сказал ему, что увидел, когда он говорил об игре на чемпионство, и спросил его, что он чувствует по этому поводу. Он ответил: <<О, это нормально. Я все понимаю, но я поклялся, что всегда буду рядом со своими детьми.>> Когда он сказал это, его голос звучал гневно. Поэтому я уточнил и это. Слушая и уточняя мы учимся видеть вещи с точки зрения других людей. Важно помнить, что другие верят в то, что говорят также сильно, как мы верим в свои заключения. Я люблю спрашивать себя: <<Чему меня может научить точка зрения другого человека?>>

\ж{Долженствования}

Слово <<должен>> должно стать для вас красным флагом. Гибкие правила и ожидания не используют это слово, потому что всегда существуют исключения и особые обстоятельства. Жёсткость это признак психического заболевания; гибкость - психического здоровья. Без гибкости не существует свободы.

\ж{Заблуждения контролирующего мышления}

Помимо игры в Бога, вы несете ответственность за то, что происходит в вашем мире. Ранее я предложил, что неврозы и расстройства личности это расстройство ответственности. Изучение того, как быть ответственным и позволять другим такую привилегию значит жить реальностью. Спросите себя: <<Какой сделанный мной выбор привел к подобной ситуации? Какое решения я могу принять, что бы изменить её?>> Также помните, что уважать других значит позволять им жить своей жизнью, страдать, чувствовать боль и решать свои проблемы.

\ж{Когнитивный дефицит или фильтрация}

Перестаньте использовать такие слова как ужасный, отвратительный, ужасающий и т.п. Выпишите фразу - <<Не нужно преувеличивать. Я могу справиться. Я могу это вытерпет>>. Моя любимая фраза взята у Abraham Low. Скажите себе: <<Это огорчительно, но не опасно>>.

Что бы остановить фильтрацию, вам нужно сместить фокус. Переведите ваше внимание на стратегии преодоления и решения проблемы, нежели на одержимость самой проблемой. Сфокусируйтесь на таких темах, как опасность и потеря. Далее подумайте о вещах, которые представляют собой безопасность или имеют для вас ценность.

\ж{Обвинения и глобальное навешивание ярлыков}

Примите ответственность за ваше поведение и решения. Сфокусируйтесь на своих проблемах. Смотрите на бревно в своём глазу, а не на соринку в глазу ближнего. Когда вы навешиваете ярлыки, спросите себя: <<Чего я пытаюсь избежать?>> Если вы поймёте, что ничего не избегаете, тогда будьте конкретны, нежели глобальны. Мой босс чаще всего консервативен. Он редко рискует. Это точное описание. Называние его безвольным идиотом говорит о вашей потребности выражения гнева о своей ситуации, в которой вы должны отчитываться перед ним.

\ж{Изменение внутреннего голоса через позитивные аффирмации}

Это буквально техника позитивного промывания мозгов. Она пытается заменить старые негативные суждения и критическое пристыжение новыми реалистичными и позитивными аффирмациями о себе. Так как большинство старых критических голосов пришли от чьего-то другого мнения о вас, они представляют собой субъективное мнение о вас, нежели то, кем вы действительно являетесь. Новые аффирмации помогут вам так изменить внутренний диалог, что бы вы были тем человеком, каким хотите быть.

Техника аффирмаций включает в себя запись позитивных утверждений о себе, 15-20 раз (идеально дважды в день). После записи утверждений вы ждете первой пришедшей спонтанной реакции. Обычно, это негативная реакция. Подождите около минуты. Если реакции нет, продолжайте записывать аффирмацию, ту же самую, что и до этого.

Цель реакции в экстернализации всех негативных стыдливых сообщений из бессознательного. Монотонность записи снова и снова застает механизм контроля стыда врасплох. Важно помнить, что о самых стыдных частях нашего образа, нам неоднократно говорилось. Например: <<Почему ты можешь быть как твоя сестра/брат/и т.п.>> или <<Ты небрежен, ленив, туп и т.п.>>

Схема аффирмации:

Аффирмация \underline{\hspace*{4cm}} Реакция
\begin{enumerate}
\item Я, \underline{\hspace*{2cm}}, часто добрый и любящий. Ждите спонтанной реакции.
\item Я, \underline{\hspace*{2cm}}, часто добрый и любящий. Все, что приходит.
\item Я, \underline{\hspace*{2cm}}, часто добрый и любящий. Все, что приходит.
\item Повторите предыдущее утверждение. Все, что приходит.
\end{enumerate}
	
Делать это нужно 21 день.

\ж{Пусть аффирмации работают на вас}

\begin{enumerate}
\item Работайте с одной и той же аффирмацией каждый день. Лучшее время это перед сном, после сна и в любое время, когда вы чувствуете себя подавленно.
\item Записывайте каждую аффирмацию 10-20 раз.
\item Произносите и записывайте каждую аффирмацию в первом, втором и третьем лице:
\begin{itemize}
\item <<Чем больше я, \underline{\hspace*{2cm}}, люблю себя, тем больше другие любят меня>>.
\item <<Чем больше ты, \underline{\hspace*{2cm}}, любишь себя, тем больше другие любят тебя>>.
\item <<Чем больше она/он, \underline{\hspace*{2cm}}	, любит себя, тем больше другие любят её/его>>.
\end{itemize}
Всегда записывайте в аффирмацию своё имя. Записывание во втором и третьем лице очень важно, т.к. ваши убеждения пришли от других именно в такой форме.

\item Продолжайте ежедневно работать с аффирмациями, пока они полностью не интегрируются в вашем сознании. Вы это поймёте, когда реакция вашего разума будет позитивной.
\item Запишите аффирмации на аудио и проигрывайте их, когда это возможно. 
\item Очень эффективно смотреть на себя в зеркале и произносить вслух аффирмации. Продолжайте их произносить, пока вы не увидите себя со счастливым и расслабленным выражением лица.
\item Используйте визуализации вместе с аффирмациями.

На основе реакций, накопившихся за определенный период, вы можете увидеть шаблоны вашего негативного голоса. Вы также можете услышать голос, о котором раньше не подозревали. Эти негативные голоса могут стать источником новых противоположных позитивных аффирмаций.

\end{enumerate}


\ж{Аффирмации для самооценки}

\begin{enumerate}
\item Я, \underline{\hspace*{2cm}}, люблю себя. Я привлекательный человек.
\item Я, \underline{\hspace*{2cm}}, весьма доволен собой.
\item Я, \underline{\hspace*{2cm}}, сильно радую других, а другие радуют меня.
\item Я, \underline{\hspace*{2cm}}, самоопределившийся человек, и я предоставляю другим это же право.
\item Я, \underline{\hspace*{2cm}}, имею право сказать людям <<нет>>, без потери их любви.
\item Чем больше я нравлюсь себе, тем больше другие нравятся себе.
\item Я, \underline{\hspace*{2cm}}, привлекательный и симпатичный, и чем больше я признаю это, тем правдивее это становится.
\item Я, \underline{\hspace*{2cm}}, заслуживаю уважения за свои успехи и достижения, в независимости от их сложности.
\item Я, \underline{\hspace*{2cm}}, стоящий(-ая) мужчина/женщина, даже если я \underline{\hspace*{2cm}}.
\item Я, \underline{\hspace*{2cm}}, любим, нахожусь ли я с кем-либо или нет.
\item Я, \underline{\hspace*{2cm}}, драгоценный и несравненный, нравится-ли мне это или нет.
\end{enumerate}

\ж{Работа с токсическим стыдом во взаимоотношениях}

<<Существует только одна настоящая проблема - проблема человеческих отношений. Мы забываем, что нет ни надежды ни радости, кроме как в человеческих отношениях>>. Antoine de Saint Exupery.

Распространенная шутка среди участников 12-ступеначатой программы звучит так: <<У нас нет отношений; мы берем заложников>>. Эта одна из тех шуток, цель которой, облегчить боль, испытываемую стыдливыми людьми, которые пытаются установить интимные взаимоотношения. На самом деле, я бы назвал близость главной проблемой, которая стала результатом интернализированного стыда.

Близость требует способности быть уязвимым. Быть близким значит рискнуть раскрытием внутреннего мира друг-другу; обнажить наши самые потаенные чувства, желания и мысли. Быть интимным значит быть собой и безусловно любить и принимать друг-друга. Это требует уверенности в себе и смелости. Такая смелость создает новое пространство во взаимоотношениях, истинную близость. Это пространство не твоё или моё; оно наше.

Для меня, как стыдливого человека, все это было невозможно. У меня не было взаимоотношений с собой. Я прятался, не только от других, но и от себя. Я был человеком действия, потому что не мог войти внутрь себя.

Там никого не было. У меня не было личности. Мои взаимоотношения с собой были отвержением и презрением. Больше всего я боялся раскрытия. У меня не было личности, которой я мог поделиться.

\ж{Проблемы созависимости взрослых детей}

Ранее я уже говорил, что созависимость и токсический стыд были из одной реальности. В вопросе взаимоотношений слово созависимость очень точно определяет проблему. Фраза Взрослый ребёнок также помогает увидеть эту проблему.

\ж{Привязанность и постоянство связи}

Из-за травмы заброшенности, стыдливые люди становятся взрослыми детьми, которые формируют созависимые отношения. Этими отношениями властвует страх заброшенности. Это результат <<постоянства связи>> о которой говорит Alice Miller.

Как Взрослому Ребёнку, мне сложно что-либо отпустить. Я храню записи, которые сделал в первый год колледжа 30 лет назад! У меня полные коробки всякой мелочи, которые я храню много лет. Изменения для меня очень сложная задача. Заброшенность создает чувство нехватки. Мне лучше держаться за то, что есть, т.к. больше уже может и не быть. По этой же причине мне сложно откладывать удовлетворение своих потребностей.

Мне сложно быть гибким во взаимоотношениях. Кажется, что просто невозможно оставить какие-либо отношения. Также я пытался создать такие отношения, в которых я становился настолько важным, что бы другой человек не мог меня покинуть.

\ж{Контроль}

Контроль это огромный враг близости. По определению, интимность исключает контроль одним человеком другого. Контроль это продукт разрушенной воли. Это попытка воли желать то, чего невозможно желать. Вы не можете изменить другого человека. Вы не можете исправить своих родителей, любовника или ребёнка. Вы не можете контролировать их жизнь или их боль.

\ж{Исполнение роли супруга}

При отсутствии аутентичной личности, вы ищите отношений с той личностью, которую чувствуете - ложной личностью. Если вы были жертвой, то единственное известное вам взаимоотношение, это отношение с вашим гонителем. Обратное также правдиво. Я был Суррогатным Супругом своей матери и Семейным Смотрителем. Как Суррогатный Супруг, я всегда искал женщин, о которых я могу заботиться. Это является инсценировкой воображаемой связи, о которой я говорил ранее. Воображаемая связь это капкан, опутанный созависимостью. Он основан на постоянстве связи, которая была создана травмой заброшенности. После формирования воображаемой связи, у нас есть только одно взаимоотношение и мы повторяем его снова и снова.

Выход из всего этого идет через работу с источником боли и внутренним ребёнком. Закрепление связи стало результатом закрепления и заморозки нашей аутентичной личности, неразрешенной травмой заброшенности. Каждый новая инсценировка воображаемой связи это наша попытка провести работу с горев. Мы выбираем похожих людей, для того, что получить ещё один шанс на разрешение проблемы. Каждый новый партнер представляет аспекты одного или обоих наших родителей. Мы пытаемся сделать партнера нашими родителями, что бы мы смогли разрешить конфликт и двигаться дальше. Так как мы уже не дети, это никогда не срабатывает.

\ж{Чрезмерные инвестиции во власть, достоинство и ожидания}

Из-за того, что любое взаимоотношение взрослого/ребёнка это взаимоотношение незрелых детей, результатом этого становится чрезмерное инвестирование во власть и достоинство другого человека. Подобная инвестиция выходит из потребности заброшенного ребёнка в заботящемся родителе. Ожидание того, что партнер предоставит то, что не смог предоставить один из родителей - заблуждение. Это нереалистичное ожидание и заканчивается разочарованием и гневом.

\ж{Проекция отщепленных частей личности на партнера}

Один из самых разрушительных аспектов стыдливых взаимоотношений это проекция собственных отщепленных частей на нашего партнера. В фильме Terms of Endearment, были искусно представлены эти отношения. Джек Николсон был представлен мужчиной, который идентифицировал себя с дикой, импульсивной сексуальной частью. Ширли Маклейн была представлена сексуально подавленной перфекционисткой, контролирующей, нравоучительной вдовой, живущей напротив. Каждый был воплощением крайних полярностей токсического стыда. Джек Николсон был рассеянным и вел себя менее чем человеком. Ширли Маклейн была святой и вела себя более чем человеком. Пара многому научила друг-друга, пока двигалась в танце между притяжением и отталкиванием. Наконец, каждый помог другому интегрировать части, которые были отщеплены. Он позволили ей принять свою сексуальность, в то время как она познакомила его с консервативной и заботящейся частью.

Когда я консультировал людей, находящихся в разрушительных отношениях, то чаще всего они относились друг с другом через отщепленные части. Щедрый мужчина чаще всего женился на эгоистичной женщине; перфекционистка выходила замуж за растяпу; заботящаяся женщина влюблялась в эмоционально недоступных мужчин. Вместо того, что бы учиться друг у друга через интеграцию своих отщепленных частей, они жили с выражением этих частей в своих партнерах. И так как каждый отщепил от себя часть, выраженную партнером, они осуждали и злились на эту часть в своём партнере.


\ж{Коллаже привлечения/отталкивания}

Вариацию упражнения <<Примирение со всеми вашими жителями>> предложил преподобный Mike Falls. Майкл консультирует уже на протяжении 20 лет. Когда к нему приходит клиент с проблемой отношений, Майкл предлагает сделать следующее.

Он предлагает им просмотреть серию журналом и выбрать все фотографии, понравившихся им людей. Далее они создают коллаж из всех этих фотографий на большой доске.

Далее они снова проходят по журналам и выбирают изображения людей, которые их отталкивают. Они также должны сделать большой коллаж. Коллаж приятных людей, это, скорее всего, ваши части, с которыми вы чрезмерно идентифицированы. Отталкивающие изображения, возможно, являются вашими отщепленными частями. Когда вы будете в курсе этих отщепленных частей, вы можете провести диалог с ними в такой же манере, какой я описывал в 4 главе.

Лучше всего сделать коллажи как с противоположным полом, так и со своим. Часто, мужчины стыдятся своих женских черт, а женщины - мужских.

Карл Юнг верил, что часть тени каждого человека это противоположная ему сексуальность. Каждый мужчина и каждая женщина это союз мужских и женских гормонов. У мужчин больше мужских гормонов и меньше женских. У женщин все наоборот. Женственную теневую часть мужчины Юнг называл анимой. Мужественную теневую часть женщины - анимусом. Интеграция теней анимы/анимуса имеет решающее значение для полной интеграции человека.

Ранее я описал, как жесткие культурные сексуальные роли формируют ложные личности, например, через чрезмерную идентификацию с одной из частей. Мужчин стыдят за женственные черты, называя <<маменькиным сынком>>. Женщин стыдят за их мужественность.

Я упоминал Terms of Endearment как фильм, который драматично изобразил мужскую/женскую полярность. Другой фильм, также показывающий эту полярность, это The African Queen, в котором снимались Хамфри Богарт и Кетрин Хемберн. Богарт чрезмерно идентифицируется с мужской энергией, а Хепберн с женской. В конечном итоге происходит интеграция отщепленных частей личности каждым персонажем и каждый из них трансформируется добавленной энергией, представленной другим. Оба фильма получили награды киноакадемии. Было бы интересно узнать, как много фильмов, получивших награды, представляли подобную борьбу за интеграцию и целостность.

\ж{Опасные ситуации в отношениях}

Определенные ситуации в отношениях могут быть более уязвимы к стыду, чем другие. Критика и отказ болезненны для любого, но они мучительны для стыдливых людей. Далее позвольте мне описать определённые ситуации, которые регулярно запускают стыдливую спираль. К этим ситуациям нужно подготавливаться и знать о них.


\ж{Разговор с родителями}

Наши родители это наши первичные взаимоотношения, поэтому они представляют собой постоянный риск запуска стыдливой спирали. Если в прошлом вас серьёзно стыдили, будьте бдительны даже в случайном разговоре с родителями. Если вы усердно работали над ослаблением стыда и проделали работу над второй стадией, вы будете готовы избежать попадания на крючок. Если такой работы не было, вы в опасности. Даже простой разговор по телефону может запустить старый слуховой импринт.


\ж{Властные фигуры}

Одной из общих характеристик детей алкоголиков является страх перед властными фигурами. Практически всегда это имеет отношение к пристыжающему насилию в первичных взаимоотношениях. Это также может быть связано со стыдливыми инцидентами в школе. Я знаю профессора психологии, который испытывает стыд каждый раз, когда мимо проезжает полицейская машина. Его мать раньше угрожала ему, говоря, что за ним приедет полиция и заберет его в тюрьму. Такое происходит не так часто. Многие стыдливые люди переживают вызванные стыдом реакции просто в присутствии босса или властной фигуры.

\ж{Новые взаимоотношения}

Стыд часто активируется в новых отношениях. Самая частая его форма представляет собой критический внутренний диалог, которые обычно начинается сразу, после ухода другого человека. Пристыжающий голос обычно предлагает такое: <<Ну, снова все испортила!>> или <<Отличная работа, мр. Зацепа!>> или <<Ты со своим бормотанием>>. Новые взаимоотношения рискованны, потому что они раскрывают нас кому-то, кому мы ещё не раскрывались.

\ж{Когда ты или они злятся}

Большинство стыдливых людей имеют дефицит гнева. Мы не знаем как выражать гнев и мы невероятно уязвимы к манипулированию гневом. Я вспоминаю одного парня, который мне сильно не нравился. В один день он выразил в мою сторону гнев, совершенно по непонятной мне причине. В действительности, я его хвалил. Несколько других людей сказали, что он завидует мне. Когда я хвалил его, он реагировал гневом. Он слышал совершенно не то, о чем я ему говорил. Я неделями размышлял об этом инциденте. Я хотел позвонить ему и разобраться. Мне пришлось применить самоутверждающие разговоры с собой, что бы остановить себя. Его гнев был о нем и его личной истории, а не обо мне.

Большинство из нас стыдили гневом или яростью. Когда кто-либо выражает гнев, наша первая реакция это страх. Различные техники в разделе о критике могут помочь справиться с этим гневом.

\ж{Когда причиняют боль вам или это делаете вы}

Из-за того, что нам причинили боль, мы боимся причинять боль другим. Часто мы не можем справиться, когда нам больно. Если ваши родители манипулировали вами при помощи обид, вы в особенности уязвимы к этому. Стыдливые родители манипулируют детьми при помощи обид всякий раз, когда они ведут себя не так как им хочется. <<Твои дети никогда не узнают, как ты обижал своего отца>> или <<Я не знаю, смогу ли я тебя простить. Ты сильно обидел меня>>. Большинство обид это чистая манипуляция. Они используются, что бы получить то, что нужно человеку. В здоровых отношениях партнёры несут ответственность. Если я тебя обидел, я могу признать своё участие в этом. Я также знаю, что часть этой обиды связана с тобой и твоей личной историей.


\ж{Успехи}

В книге Man Against Himself, Karl Menninger описывает людей, у которых были срывы после достижения успеха. Некоторые даже совершили самоубийство. Стыдливые люди не верят, что у них есть право быть счастливым. Глубоко внутри токсический стыд говорит им, что у них нет права на деньги и веселье, когда другие люди живут в нищете и страданиях. Успех не ограничивается только материальным благополучием. Вы можете чувствовать токсический стыд за любой вид вознаграждения. Чаще всего, это проблема семейной системы. Если другие члены семьи остаются в своих замороженных жестких ролях, а один из членов прорывается и создает свою новую уникальную жизнь, этот человек может чувствовать стыд за то, что он так изменился и достиг успеха. Помните, в дисфункциональных семьях никто не должен бросать свою роль.

\ж{Получение внимания и похвалы}

Стыдливые люди плохо принимают комплименты и похвалу. Глубоко внутри токсический стыд кричит: <<У тебя нет права на любовь и внимание>>. Если вы провели тщательную работу, представленную в главе 5, вы знаете, что вы достойны любви. Ваши безусловные любовные отношения с собой станут основой для вашего принятия всей любви и похвалы, которая ваша просто так.

\ж{Критика}

Годы назад я написал пол книги о том, как жить с критичным человеком. Почему-то я так и не закончил эту книгу. Я чувствовал, что критика была большой проблемой в человеческих отношениях и что люди нуждались в помощи, для защиты от неё. Определенно, стыдливых людей критика отталкивает и причиняет боль. Она также привлекает их, т.к. позволяет межличностную передачу стыда другим.

Я никогда не верил в ценность так называемой <<конструктивной критики>>. В нашей группе мы даёт друг-другу обратную связь. Обратная связь это наблюдение высокого качества без интерпретации. В групповой обстановке обратная связь может быть невероятно полезна. Но критика, как я её определяю, всегда субъективная интерпретация, основанная на личном опыте и личной истории человека. Как таковая, она не очень полезна.

Я призываю стыдливых людей избегать критики других и предлагаю следующие техники, как способ справится с критикой в свою сторону.

Главный принцип в обработке критики это никогда не защищать себя. Если вы защищаете себя, вы берете на себя токсический стыд.

\ж{Затуманивание}

Затуманивание это адаптированная техника, взятая из тренинга по настойчивости. В этой технике вы признаете правду, возможность правды или вероятность правды. Вы не защищаетесь. Вы просто позволяете критическому замечанию пройти через вас, как через туман. Например, вы говорите с матерью по телефону. Она говорит: <<Твои дети недисциплинированные. У них будут проблемы в школе>>. Вы отвечаете: <<Ты права. В школе они могут попасть в неприятности>>. Вы признаете возможность правдивости утверждения вашей матери. Затем она может сказать: <<Ну, когда ты собираешься их дисциплинировать?> Вы отвечаете: <Я дисциплинирую их тогда, когда это понадобится>>. Это достаточно расплывчато и признает правдивость утверждения.

\ж{Прояснение}

Прояснение это способ прибить вашего критика к стене и раскрыть его намерение передать стыд. Давайте представим, что ваша супруга говорит: <<Ты ведь не собираешься надевать эти коричневые брюки?>> Вы отвечаете: <<Что тебе не нравится в этих коричневых брюках?>> Вне зависимости от того, что отвечает критик, вы снова просите пояснения. Если критик отвечает: <<Они выглядят дёшево>>, вы говорите: <<Что тебе не нравится в дешёвых брюках?>> или <<Почему дешёвые брюки тебе напрягают?>> Эти вопросы заставляют критика переместится во взрослую часть их личности. Взрослый не загрязнён подавленными чувствами. Взрослый ориентирован в сторону логики и объективности.

Обычно результатом этой техники становится рассеивание энергии критика. Один вопрос за другим выкурит настоящую причину, лежащую за критикой. Реальная причина или чисто субъективна или попытка, через критику, прикрыть собственный стыд и перекинуть его на вас. Эта техника не всегда работает. Все они не всегда работают. Однако, чем больше у вас техник, тем больше у вас есть защитных решений для утверждении себя.

\ж{Дать отпор}

Дать отпор значит то, что значит. Вы даете отпор своему критику. Это форма напористости. В даче отпора, я предлагаю следовать следующему руководству:

\begin{enumerate}
\item Оставайтесь в собственной шкуре. Говорите то, что воспринимаете (слышите и видите), интерпретируете, чувствуете и хотите.
\item Используйте <<Я>>. Будьте ответственны в том, что воспринимаете, интерпретируете, чувствуете и хотите.
\item Используйте поведенческие детали, основанные на сенсорном восприятии, нежели оценочных словах.
\item Смотрите человеку в глаза.
\end{enumerate}

Недавно мне пришлось дать отпор. Я только что купил кабриолет БМВ. Это самая дорогая машина, которая у меня когда-либо была. У меня есть определенный культурный стыд из-за моего бедного детства. Всякий раз когда я нахожусь рядом с богатыми людьми, я испытываю стыд. Я чувствую себя хуже, как будто я не в своей тарелке. Этот же стыд проявляется, когда у меня есть что-то (как новый БМВ), что стоит больших денег.

Когда я показал машину одному из моих родственников, он дал мне критический комментарий. Он сказал: <<Вау, красавица. Спорю что целая семья могла бы жить на те деньги, что ты за неё заплатил>>. Когда я это услышал, мой разум отключился. Я снова почувствовал стыд. Голос внутри меня произнёс: <<Ты мог купить что-нибудь в пол цены и отдать немного денег нуждающимся>>. Я противопоставил ему: <<Я люблю себя поэтому радуюсь жизни покупая хорошую новую машину>>. Я посмотрел на своего родственника и сказал: <<Когда ты делаешь такие комментарии, я понимаю, что ты чувствуешь себя плохо из-за моей удачи. Каким-то образом, моя удача запускает твой стыд. Мне жаль, что у тебя есть этот стыд и собираюсь послать тебе мою новую книгу о лечении стыда>>. В этот момент мой родственник начал длинную защитную тираду о моей чувствительности. Он допустил, что он не желал мне вреда и что я все не так понял. Он сказал, что очень рад за меня; что я заслуживаю этого. Я согласился и уехал. Отпор может запустить гнев в вашем критике. В этом случае, я просто говорю: <<Я буду рад поговорить с тобой, когда ты перестанешь злиться>> и ухожу. Уход это утвердительное поведение перед лицом хулиганской или преступной критики.

\ж{Коломбивание}

Коломбивание взято из ТВ детектива Коломбо. Детективы бывают разных размеров, форм и стилей. Коломбо небрежен и неопрятен. Он постоянно задает вопросы. Кажется, он в восторге от людей, которых допрашивает. Однако, есть в этой очевидной абсурдности проникновенный блеск. Он никогда не пропускает самые незначительные детали. Он проверяет все. Он мастер конкретных специфических деталей.

Когда вы Коломбите вашего критика, вы играете тупого и задаете множество вопросов. Вы говорите: <<Правильно ли понимаю... Вы думаете, я должен перестать носить такую прическу... Что такого в моей прически, что вам не нравится?>> Когда они ответят на вопрос, вы повторяете вопросы. Цель дойти до источника и раскрыть их субъективность. Критика обычно связана с токсическим стыдом критикующего, а не с вашей прической. Через Коломбивание вы избегаете своих защитных механизмов и вытаскиваете критикующего из его прикрытия критического родителя.

\ж{Исповедь}

Эта реакция полезна, когда вы совершенно точно и однозначно сделали то, в чем вас критикуют. Если вы разлили молоко вы говорите: <<Да, я разлил молоко>>. Просто подтвердите. Не добавляйте таких вещей, как <<Как глупо с моей стороны!>> Десятая ступень 12-ступенчатой программы утверждает, что: <<Когда мы совершаем ошибку, мы быстро признаем её>>. Это этап поддержки. Цель его в поддержании нашего фокуса на здоровом стыде. Мы можем и будет совершать ошибки. Нам не нужно извиняться за них. Это часть человеческого натуры. 

\ж{Подтверждение}

Эту технику вы можете использовать в разговоре с родителями. Каждый раз, когда кто-то начинает вас критиковать, скажите про себя: <<Вне зависимости от того, что ты скажешь, я остаюсь достойным человеком>>. Повторяйте это утверждение снова и снова.

Вы также можете заякорить это позитивное утверждение. Проговаривайте утверждение вслух, при этом визуализируя себя, стоящим прямо и уверенно. Вы смотрите другому человеку в глаза. Когда вы почувствуете силу и мощь этого ощущения - заякорите его. После этого, в любой ситуации с критикой вы можете активировать этот якорь.


\ж{Утешение}

Я использую этот метод, когда совершенно ясно, что я непреднамеренно нарушил личные границы другого человека. Цель утешения это разрешить другому человеку выразить свои чувства, а не винить или защитить себя.

Утешение это точно такое же поведение, как и активное слушание. Предположим, моя машина заблокировала выезд. Меня нет дома, я ушел на пробежку. Когда я возвращаюсь, моя жена расстроена и злится. Она говорит: <<У меня запись к дантисту и я опаздываю. Ты должен был спросить, нужна ли сегодня мне моя машина>>. Я отвечаю: <<Я понимаю, что ты расстроена и зла. Я прямо сейчас передвину машину>> или <<Я понимаю твоё разочарование>> или <<Я понимаю, как это обидно>>.

Утешение это форма ответственности. Оно позволяет признать расстройство другого человека нашим непреднамеренным проступком и совершить разумные поправки. С другой стороны это позволяет избежать запуска стыдливой спирали. Непреднамеренное причинение боли также является частью человеческой натуры.


\ж{Когда все остальное не работает - сбивание с толку}

Эту технику я советую использовать не в интимных отношениях. её нужно использовать, если другие варианты не принесли результата. Сбивание с толку это способ заставить кого-то от вас отстать. Используйте её когда вы чувствуете себя уязвимо и вы не можете дать отпор или прояснить.

В сбивании с толку вы используете громкое или придуманное вами слово. Например, сотрудник ругает вас за ваш долгие ланч. Вы не хотите давать отпор. Ранее вы уже попадали в эту ситуацию с этим человеком и все закончилось безрезультатно. Поэтому, вы смотрите на него и говорите: <<Да, трафик сегодня был праздным>>. Использование незнакомых слов или слов неподходящих под контекст часто помогает остановить критика. Вы можете увидеть озадаченный вид на лице другого человека. Его ум занят поиском смысла того, что вы только что сказали. Вы просто улыбаетесь и уходите.

Это упражнение вовлекает вашу веселую детскую часть. Вы можете почувствовать себя ликующим, когда видите озадаченность другого человека. Это даёт вам контроль. Помните, критика это прикрытие стыда и способ контроля другого человека. Сбивание с толку это техника, позволяющая вам поддерживать контроль.

Ничего не работает во всех случаях. Если одна из техник не сработал, пробуйте другую. Они формируют целый арсенал поддержки, которая защитит вас от межличностной передачи стыда.

\ж{Отказ}

Не существует большего потенциала для болезненного стыда чем отказ. Это трюизм для большинства взаимоотношений. Но для стыдливых людей, отказ подобен смерти. Мы отвергли самих себя, и для кого-то снаружи отвергнуть нас, значит доказать то, чего мы боимся больше всего - что мы дефективны и ущербны как личность. Отказ для нас значит, что мы на самом деле нежеланны и не достойны любви.

Существует различные степени отказа, варьирующиеся от неулыбчивости работника в магазине до отказа заветной любви. Боль от подобного отказа ощущается как физически, так и эмоционально. Чувствуется, как нож в груди. Я испытывал подобное только раз; Я определённо не хотел бы пережить это снова.

Все техники, перечисленные мною, могут быть полезны когда вы переживаете горе разбитого сердца. Чем больше человек провёл работы с первоначальным источником боли и оставил собственную фантазию о связи с семьей, тем лучше он сможет справиться с отказом. Если человек все ещё находиться в воображаемой связи, то для него отказ подобен смерти. Для связанного фантазией человека, отказ воздействует на обиженного и одинокого ребёнка, который так и не разрешил своё первоначальное горе.

Я искренне рекомендую вам провести работу с источником боли и внутренним ребёнком, как способ облегчения боли от ваших потенциальных будущих потерь. Чем более вы дифференцированы и разделены, тем легче вам справится с разделением и одиночеством.

Я рекомендую книгу Judith Viorst - Necessary Losses. Она представляет то, что я бы назвал философией потери. Она поможет вам принять факт потерь как необходимую часть человеческого существования.

Однажды я думал написать подобную книгу. Я хотел её назвать <<Я печалюсь, значит существую>>. Я хотел показать, что жить хорошо значит хорошо горевать. Все, что вы когда либо делали, закончилось. Жизнь это длительное прощание. Скорбь это процесс, которые все завершает. Конец работы со скорбью это новое рождение.

Когда вы проходите процесс скорби по личному отказу, вам нужна легитимизация, социальная поддержка и время. Вам требуется любящий ближний. Требуется, что бы ваши чувства были отражены и подтверждены. Лучше, что бы у вас было более чем один близкий человек. В этом и заключается преимущество таких групп как 12-ступенчатая программа.

Скорбь проходит через все описанные стадии: шок, отрицание, торг, депрессия, злость, раскаяние, грусть, обида, одиночество и т.д. Вам нужно время, что бы пройти через все стадии скорби. Худшее, что вы можете сделать, это устремится в новые отношения. Я видел, как это приводило к катастрофам. Новые отношения прикрывают ядро скорби и образуется новый слой неразрешенного горя. Скорбь по отвержению требует времени. Оставайтесь рядом с заботящими и поддерживающими отношениями. Вы достойный и драгоценный человек, несмотря на то, что другой человек покинул вас.

Наконец, помните, что интернализированный стыд это результат вашей детской травмы заброшенности. Ваш худший сон (отказ) уже произошел и вы пережили это. Вы были нуждающимся, уязвимым и незрелым ребёнком и вы все равно пережили это. Вы можете пережить это снова.

\ж{Создание <<Сирены Стыда>>}

Малые отказы это часть <<ужасной будничной>> жизни. Я использую адаптацию техники, впервые услышанной у Terry Kellogg, которая помогает справляться с ежедневными будничными отказами. Она включает в себя формирование <<сирены стыда>>. Сирена стыда это такой якорь. Когда кто-либо обижает вас, не замечает, даёт вам оценку или просто отвергает, делайте следующее:

\begin{enumerate}
\item Представьте, что у вас есть сирена, которую вы можете включить потянув за ухо. Когда вы тянете своё ухо, вы слышите сирену, которая кричит: <<Стыд, Стыд, Стыд, Стыд, Стыд>>. Когда вы слышите её, громко и четко...

\item Скажите следующее: <<О, это просто чувство... На самом деле я достойный человек...>. Произнесите это себе несколько раз. Таким образом вы экстернализируете интернализированный стыд. Вы трансформируете его из состояния обратно в чувство. Чувства повышаются и понижаются. И они заканчиваются.

\item Позвоните хотя бы одному человеку в вашей самой значимой группе поддержки. Попросите этого человека подтвердить ваше достоинство. Попросите: <<Скажи мне, что я прекрасный человек и достоин любви>>. Это восстанавливает межличностный мост.
\end{enumerate}

Если вы привьете себе привычку использования сирену стыда, она станет для вас естественной. Я заметил, что со временем у меня все меньше и меньше чрезмерных реакций на пренебрежения и недооценивания другими, когда я использую эту сирену стыда.


\ж{Любовь это работа}

Я делал многое для работы с моим стыдом в межличностных отношениях. Я потратил сотни часов на изучение и практику эффективных техник коммуникации. Я был на нескольких семинарах по настойчивости и осознанию. Все это улучшило мои навыки отношений.

\ж{Парное путешествие}

Я повторял своё убеждение, что любовь это работа. Она включает в себя приверженность. Я должен принять решение, что бы продолжать её. Десять лет назад я почти разорвал свой брак. Это было бы трагической ошибкой. Помните, стыдливые люди имеют постоянное ядро грандиозности. Если все получает не по-моему, значит я ухожу. Все или ничего!

Мой собственный брак был живым доказательством заключения, к которому пришла Susan Campbell в своей книге The Couples Journey. Susan основала эту книгу на длительном исследовании, которое она провела на большом числе пар, которые были вместе на протяжении более 20 лет. Она узнала, что каждая из пар прошла определённые стадии и этапы борьбы, на пути к близости.

\ж{Романтическая стадия}

Каждая пара была влюблена. Это была Романтическая стадия. Эта стадия характеризуется слиянием границ. Это сильное и всеобъемлющее ощущение. Пара чувствовала, что они могут преодолеть все что угодно. Вскоре после свадьбы начинается новая стадия.


\ж{Борьба за власть}

На этой стадии границы восстанавливаются. Больше нет слияния различий. Начинают действовать правила семейной системы каждого человека. Это стадия узнавания различий друг-друга. Должны пройти переговоры по правилам о деньгах, сексе, болезнях, социализации, празднованиях, поддержании домашнего очага, а с рождением детей, по правилам воспитания. Для большинства пар это занимает около 10 лет. Далее идет период стабильности. На время все тихо и спокойно. Но вскоре старение, пустующее <<гнездо>> и индивидуальные процессы открывают третью стадию.


\ж{Владение проекциями и принятие личной ответственности}

Эта стадия характеризуется путешествием к переоценке ценностей, личной ответственностью и стремлением к высшему смыслу. Каждый партнёр принял свои проекции анимы/анимуса. Мужчина стал индивидуализирован включением в себя женской стороны, а женщина - мужской. Они приняли свою генеративную потребность в самореализации. По мере того, как каждый партнёр становится все более цельным, начинается новый и плодотворный этап.


\ж{Интимное плато}

Благодаря тому, что партнёры стали цельными внутри себя, они могут стремится к своему партнёру из желания, нежели из нуждаемости. Больше не нет латания дефицита друг-друга. Новая связь основана на выборе и решении, нежели на воображаемой связи и нуждаемости.

Каждый может любит более щедро. Каждый даёт, потому он/она действительно этого хочет. Возникает новое плато интимности. Возвращаются некоторые качества из первой стадии. Каждый очарован уникальностью и различиями другого. Каждый становится заветным другом своего партнёра. Образуется связь глубокого уважения и признательности.

Путь к интимности знаменуется следующим: здоровым конфликтом; умением переговаривать и честно бороться; терпением, тяжёлой работой и смелостью, что бы рискнуть быть индивидуальностью. Кроме всего, это знаменуется волей к принятию дисциплинированной любви.

Суть всего этого заключается в том, что достижение любви и интимности во взаимоотношениях это динамический процесс. Такой процесс отличается приливами и отливами. Знаменуется конфликтами и индивидуализацией. В конце концов, это того стоит. Я согласен с St. Exupery, что <<...нет ни надежды ни радости, кроме как в человеческих отношениях>>.

\ж{Духовное пробуждение}

Работа по трансформации токсического стыда напрямую ведёт к духовности. Здоровый стыд говорит, что у нас есть ограничения; нам нужна помощь; мы не Боги. Существует кто-то или что-то выше нас. Здоровый стыд это источник духовности.

Когда вы знаем наши ограничения, мы знаем, что существует что-то большее чем мы. В 12-ступенчатой программе это большее зовётся <<Высшей Силой>> или Богом, как вы понимаете Бога. Я лично называю эту силу Богом. Кроме того, я считаю, что такая сила не может быть меньше, чем личной. Вершина человеческой жизни это индивидуальность, разделённая в интимной любви. Если Бог это Высшая Сила, Бог не может быть меньше, чем наша человеческая самореализация. Я верю, что духовность включает в себя личное единство с личным Богом. Мой путь достижения этого единства проходит через взаимоотношение с Иисусом Христом.

Для стыдливого человека <<духовное пробуждение>> невозможно, пока не будет проведена работа по экстернализации. Без этой работы, наше ЭГО остается разорванным и отчужденным.

\ж{Полное человеческое сознание}

Все описанные мною упражнения в предыдущих главах, имеют отношение к реконструкции вашего ЭГО и интеграции отчужденной энергии.

Это чрезвычайно важная работа на пути к цельности. Но вам нужно понимать, что ЭГО это не истинное <<Я>>. На рисунке 12.1 представлено распространенное восприятие полноты человеческого сознания. Маленький круг в центре это ЭГО. Оно представляет собой основную психосоциальную границу. Это суженное сознание, которое имеет дело с социокультурной идентичностью. Главная забота ЭГО это выживание. Главная цель ЭГО это удовлетворение наших потребностей в зависимости и выживании. Когда наше ЭГО сильное, мы знаем, что может удовлетворить наши потребности. Мы знаем, что можем получить достаточно еды, одежды, тепла, любви и защиты. Сильное ЭГО имеет важное значение для выживания.

Второй круг имеет отношение к хранению текущих и прошлых переживаний. Он также представляет из себя склад из запрещенных чувств, потребностей и желаний. Этот круг называют личным бессознательным или подсознание. Все наши части, которые были пристыжены и отщеплены, хранятся в подсознании. Подсознание это жилище наших субличностей. Карл Юнг называл эти части нашего сознания тенью. После интеграции нашей тени мы готовы к расширению. Расширение ведет к полному спектру сознания. На картине 12.1 внешний круг представляет этот полный спектр сознания. Stone и Winkelman называют этот уровень осознанием. Трансперсональные психологи часто обращаются к этому кругу как к paraconscious или высшему сознанию. Paraconscious это сфера нашего истинного бытия или самости. Достижение этого уровня сознания трансформирует. На этом высшем уровне сознания вы видим и воспринимаем все совершенно по-другому. Это уровень открытия. Здесь мы находим новую личность. На картинке 6.1 ( >>374987 ) я называл это открытие работой III стадии. 

\ж{Духовное пробуждение}

Расширенное сознание это другое название духовного пробуждения, что значит рост и расширение осознания. Духовность говорит о целостности и полноте.

Опасно и контрпродуктивно работать над этим расширением, до того, как будут объединены нижние уровни (ЭГО). Духовные мастера часто говорит о процессе интеграции ЭГО как о путешествии в пустыню. В христианском Писании, Иисус уходит в пустыню на 40 дней и 40 ночей до начала своей духовной работы. Мистики говорят о <<темной ночи души>>. Темная ночь души это этап приготовления перед переходом на путь объединения. Путь объединения это путь блаженства - состояние истинной близости с Богом.

Духовные мастера также говорят нам, что если работа с ЭГО не окончена, нас вернёт назад к ней. Большая часть работы с ЭГО это работа с границами и замороженной энергией. Если только эта энергия не будет разморожена и развязана, нас снова вернёт к ней. Вы видели, что одним из главных путей, каким ваша замороженная и пристыженная энергия обрабатывается, это через инсценировки. Это также может случиться с духовностью, если работу с ЭГО оставить незавершенной.


\ж{Духовная инсценировка}

Я показывал вам, каким благочестивым и праведным может быть прикрытие токсического стыда. Побегами праведности являются перфекционизм, осуждение и обвинение. Для меня, одним из верных способов узнать, что данный стиль духовности не является истинно духовным, это применить следующие критерии. Насколько обвиняющим и осуждающим оно является?

Много лет назад я задался вопросом об осуждающей болтовне Jim Bakker и Jimmy Swaggart. Когда я их слушал, их слова вызывали раскол. Это были <<мы>> и <<они>>; они осуждали и обвиняли других. Они видели соринку, но не видели бревно. Они прятались от своего стыда. Свой неразрешенный стыд они разыгрывали сексуально. Они прикрывали собственные неразрешенные проблемы с ЭГО и неразрешенный стыд через обвинения и осуждения других. Они действовали как духовные лидеры, но их ЭГО проблемы оставались нерешенными.

Вне зависимости от того, сколько мы проведём молитв и хорошей работы, если потребности ЭГО не удовлетворены, они будут постоянно тянуть нас назад на тот уровень, где эти неудовлетворенные потребности существуют.

В моем случае, хотя я и вел трезвый образ жизни уже 10 лет, учил взрослых теологии и исследовал сферу древней духовной мудрости, я все ещё был компульсивен. Мой Внутренний ребёнок все ещё был в синяках и не вылечен. Я все ещё был в ненасытном стремлении заполнить пустую дыру в моей душе. Подобное стремление сильно отличается от здорового стремления к Богу. Проблема компульсивности в неудовлетворенных развивающихся ЭГО потребностях. Хотя святые могут выглядеть компульсивными, они не является тем, о чем говорит их стремление к Богу. Их стремление к Богу исходит из высшей потребности человека. Эта потребность возникает, когда зависимые потребности были адекватно удовлетворены. Картинка 12.2 показывает мою адаптацию того, что Абрахам Маслоу называл иерархией человеческих потребностей.

Нижняя половина пирамида представляет зависимые потребности. Маслоу называл их дефицитарными потребностями. Удовлетворение этих потребностей зависит от других. Если они неудовлетворены, то энергия, которая должна быть освобождена при их удовлетворении, остается замороженной. Эта энергия постоянно выражает себя в шаблонной проекции и повторяющейся компульсии.

Согласно Маслоу, эти базовые человеческие потребности иерархичны. Вас не будет волновать структура и стимуляции, если у вас недостаток еды, тепла или жилья. Тоже самое правдиво и для высшей потребности. Вы не будете искать истины, красоты и Бога, если ваше ЭГО имеет незаконченные дела.

Духовность это базовая человеческая потребность. Эта причина, по которой мы развиваем наше ЭГО. Высшая реальность всегда объясняет нижнюю. ЭГО служит как платформа для расширения. Стыдливое ЭГО боится отпустить контроль. Оно всегда настороже, что бы его не застигли врасплох. Сильная структура ЭГО позволяет вам отпустить и расшириться. Мы должны отпустить, что бы расти. Как вы увидите далее, медитация, транспорт расширения сознания, требует сильного ЭГО. Что бы успешно медитировать, вы должны уметь отпускать контроль. Сильное интегрированное ЭГО это как первая ступень ускорителя ракеты, позволяющая вам выйти в пространство высшего сознания.

\ж{Медитация}

Один из способов достижения высшего сознания это медитация. Истинная медитация это конечное преодоление токсического стыда. Цель медитации в непосредственном единении с Богом. Физическая любовь дала вам осознание единства. Истинная любовь привела вас в пространство источника всех единений. Молитва позволяет вам вести диалог с источником единства. Медитация позволяет вам соединиться с источником всего единства в отношении блаженства. 

\ж{Техники}

Существует множество способов медитации. Важно понимать, что техники медитации не являются целью медитации. В обычном смысле этого слова у медитации нет цели.

Медитация это поиск непосредственной близости с Богом. Различные техники направлены на создание условий для такой близости. Главным условием для этого уединенного союза является тишина. Какая бы не была техника, она направлена на создание этой тишины. Техники варьируются от простого осознания дыхания до активных упражнений кружащихся дервишей. По середине находятся мандалы, мантры, музыка, ручные искусства, умственные образы и массажные упражнения. Выбор техники зависит от личного предпочтения. Каждая техника направлена на отвлечения вашего ума и поглощения всего вашего сознательного внимания.

После определенной практики человек может достичь состояния <<не ума>>. Это состояние называют тишиной. После достижения тишины, активируется ранее не использованная умственная способность. Это форма интуиции. С помощью этой способности, человек может напрямую познать Бога. Духовные мастера представляют довольно похожее свидетельство этого состояния. Они говорят об этом интуитивном знании как о <<объединяющем сознании>> или божественном сознании или высшем сознании. Это <<знание>> непосредственно. С этим новым виденьем, у человека появляются новые инсайты и просветления.

\ж{Три пути достижения высшего сознания}

Далее я опишу три подходя к высшему сознанию. Для лучшего результата, я рекомендую вам записать инструкции по медитации на аудио.

\ж{Рефрейминг вашей жизни через взгляд вашего магического ребёнка}

\ж{Вступление - Миф о магическом ребёнке}

Рисунок 3.1 показывает различные слои прикрытия токсического стыда. В центре круга находится алмаз. Этот алмаз представляет собой то, что Wayne Kritsberg называл <<Магическим ребёнком>>. Магический ребёнок это психическая энергия, которая устояла перед натиском токсического стыда. Магический ребёнок проявляется, когда мы принимаем и питаем раненного внутреннего ребёнка.

В процессе перевоспитания вы признаете, что в вас есть часть аутентичной личности, которая все пережила. Эта та часть, которая купила эту книгу и ведет вас к восстановлению и излечению токсического стыда. Магический ребёнок это та ваша часть, которая может смеяться даже перед лицом боли; которая может веселиться и наслаждаться приятными моментами в жизни несмотря на ваш токсический стыд. Магический ребёнок это часть вас, о которой говориться в писании: <<...если не обратитесь и не будете как дети...>>. Магический ребёнок это ядро вашей сущности. Магический ребёнок это то, что дзен мастера называют <<сознанием начинающего>>. Магические ребёнок это то, что делает вас вами.

Как теолог, я вижу Магического ребёнка образом Бога в нас. Это часть вас подобная Богу. При создании, Бог посмотрел на все пути, каким его реальность могла бы проявиться. Вы стали инкарнацией одного одного из этих путей.

Это мифический способ мышления. Мифы это способ структурирования значения трансцендентных реалий. Мы используем мифы и символы, когда говорим о Боге. Вся речь Бога мифична и символична. Мы не можем избежать вопроса о Боге, т.к. этого требует наш здоровый стыд.

Paul Tillich любил ругать своих учеников, за разговоры о том, что речь Бога только лишь символична. Символы вовлечены в реальность, которую они пытаются описать. Символы более целостны, нежели логика. Более половины Иудейско/Христианского писания написана в символах (виденья, сны, притчи, псалмы и т.д.).

В моем мифе, каждый из нас уникальное и неповторимое творение Бога и каждый из нас воплощает в себе некоторые аспекты божественной реальности. Каждый из нас пришёл в мир, что бы проявить эту уникальную часть божественной реальности. Делаем мы это будучи самими собой. Чем больше мы являемся собой, тем больше мы являемся подобны Богу. Что бы действительно быть собой, нам требуется принять нашу вечную миссию и судьбу. Это состоит из проявления в полной мере нашего богоподобия в человеческой манере. Я следую Иисусу Христу, потому что для меня он выражает именно это.

Наша судьба известная Магическому Ребёнку. Когда мы проделаем работу по ослаблению стыда, наш Магический ребёнок становится доступен нам и мы можем продолжить наш путь к самости и истинному бытию. Травма заброшенности сбросила нас с пути. На мгновение мы потеряли путь. Магический ребёнок подталкивал нас к работе над восстановлением. Когда мы разрешаем нашу скорбь, то продолжаем наш путь. Мы реинтегрируем ЭГО и устанавливаем ЭГО границы. Это формирует нашу человеческую идентичность. Однако, даже после полного восстановления ЭГО идентичности, даже когда она позитивна и жизнеутверждающа, она социально и культурно ограничена. Она связана; ограничена языком и суженным осознанием. Наша истинная сущность вечна и прочна. Она сохранилась на протяжении всех изменений. Она выжила, как наш Магический ребёнок.

Используя эту мифологию, я приглашаю вас расширить своё сознание с вашим Магическим ребёнком. Так как эта часть была скрыта, но жива, пришло время для расширения и развертывания. Следующая медитация это один из способов продолжить это развертывание.

\ж{Медитация Магического ребёнка}

Найдите тихое место, где вас не побеспокоят. Выключите телефон. Найдите удобное кресло. Устройтесь на нем поудобнее. Не скрещивайте свои руки и ноги. Выберите для этого время, когда вы не очень устали.

Медитация проходит эффективнее всего, когда ваш мозг производит альфа или тета волны. Альфа и тета волны создают измененное состояние сознания, которое возможно в промежутке между бодрствованием и сном. Это правильный контекст для медитации.

Запишите следующее:
Начните с фокусировки на дыхании...Дыхание это жизнь. Оно символизирует самый фундаментальный ритм жизни, держание и отпускание. Во время вдоха... и выдоха... представьте океан, с поднимающимися волнами (на вдохе), переламывающимися и обрушивающимися на берег (на выдохе). Услышьте мощь океана, когда волны обрушиваются на берег. (Делайте это в течении 2 минут).

Пусть ваш ум наполнится дыханием... Осознайте вашу грудь на вдохе... и выдохе... (1 минуту)... Теперь осознайте воздух на вдохе и выдохе... Осознайте разницу в воздухе на вдохе и выдохе... На вдохе он теплее или холоднее?... На выдохе?... (1 минуту)...

Теперь продышите свой лоб и прочувствуйте любое напряжение, которое могло бы там быть и продышите его... Повторите... Теперь область вокруг глаз; ищите любое напряжение и продышите его... Повторите... Теперь область вокруг рта и челюсти; ищите любое напряжение и продышите его... Повторите... Продолжайте этот процесс с шеей, плечами, руками, пальцами, грудью, животом, ягодицами, коленями, икрами, ступнями и пальцами ног...

Теперь расслабьте все своё тело... Расслабьте каждую мышцу и каждую клетку... Представьте, что вы пусты внутри, как человеческий бамбуковый стебель... Вдохните теплый золотой солнечный свет через макушку, через все ваше тело и пальцы ног... Повторите процесс несколько раз...

Представьте, что вы стоите перед тремя ступеньками ведущими к двери... Поместите все ваши беспокойства в воображаемый шар солнечного света... Сделайте этот шар своими руками... Положите все ваши тревоги в этот шар и закопайте его... Позднее вы их сможете забрать... Идите вверх по лестнице и откройте дверь... За дверью вы увидите ещё 3 ступеньки ведущие к двери... Сделайте ещё один шар из солнечного света и поместите в него все ваши предположения и жесткие убеждения, а потом закопайте этот шар... Вы заберете его когда закончите... Идите вверх по лестнице и откройте дверь... За дверью вы увидите ещё 3 ступеньки ведущие к двери... Сделайте ещё один шар из солнечного света, придав рукам форму чаши. В этот раз поместите в него своё ЭГО. Включите в него все ваши роли, которые вы играете... Поместите их в шар один за другим...

Теперь откройте дверь и выйдите на крыльцо... Представьте, что вы смотрите в бездну космоса... Посмотрите прямо перед собой и представьте лестницу из света... Когда она полностью сформируется, посмотрите на верх лестницы... Там появится ваш Магический ребёнок... ребёнок начнет спускаться к вам по лестнице... Отметьте все, что сможете об этом Магическом Ребенке. Во что он одет?... Посмотрите на его лицо, когда он подойдет ближе... Отметьте его глаза... волосы. Когда ребёнок взойдет на крыльцо обнимите его... Почувствуйте связь с мощной часть вашего существа... Поговорите с ним... Представьте, что вы можете обсудить вашу жизнь... Момент, когда вы были зачаты... каково было эмоциональное состояние матери?... Что на счёт отца?... Представьте их союз и ваше зачатие с точки зрения Бога. Спросите вашего Магического ребёнка о вашем предназначении... Зачем вы здесь? Кто вы? В чем ваша уникальность?... Какую уникальную часть Бога вы проявляете?... Примите любой ответ Магического ребёнка... Ответ может прийти в форме слов... Вы можете получить символы или коллажи символов... Вы можете ощутить сильное чувство... Просто примите то, что получите... даже если не понимаете, что это. Если вы получите ясный ответ о предназначении, рассмотрите вашу жизнь с этой перспективы... Представьте значимых людей, которые повлияли на вашу жизнь. Вы можете увидеть кого-то, кто привнес негативное влияние на вас как неотъемлемую часть вашего плана или божественной цели. Вы также можете увидеть кого-то, кто позитивно повлиял на вас в прошлом, но играл менее важную роль в вашей судьбе. Пройдите по всем событиям вашей жизни шаг за шагом... Смотрите на них как на части большого плана и цели... Смотрите на них с точки зрения вашего Магического ребёнка... Позвольте фильму вашей жизни прокрутиться до настоящего времени... Поразмышляйте о том, что вы только что пережили... Почувствуйте присутствие вашего Внутреннего ребёнка... Почувствуйте ощущение единства и согласованности вашей жизни... Посмотрите на все с расширяющейся точки зрения... Посмотрите на все... всю вашу жизнь... совершенно по-другому, чем раньше... (1 минуту)...

Обнимите вашего Магического ребёнка... Скажите ему, когда вы встретитесь с ним снова... Услышьте, как ребёнок говорит, что он здесь, что бы направить вас... Ваш ребёнок союзник... Он был всегда с вами, даже в самые тяжелые времена... Теперь пришло время для созревания и расширения...

Представьте вашего ребёнка уходящего по волшебной лестнице из света... 2 минуты поразмыслите над тем, что только что вам открылось... Позвольте себе помечтать об интеграции... Позвольте всей вашей жизни прийти к единению с вашим предназначением... Почувствуйте готовность посвятить себя вашей цели... Теперь, 2 минуты помечтайте...

Начните ощущать место, в котором вы находитесь в комнате... Почувствуйте одежду на теле... воздух на лице... звуки комнаты... Позвольте чувству вашего достоинства и ценности переполнить вас... Скажите себе, что никогда не существовала никого, такого же как вы... (10 секунд)... и никогда не будет существовать... (10 секунд)... Примите решение идти вперед и делить себя с другими... (10 секунд)... Пройдите через дверь и вниз по ступенькам... Поднимите шар из света с вашим эго... Реинтегрируйте ваше эго... Почувствуйте, как вы возвращаетесь в сознание... Пройдите через ещё одну дверь и вниз по лестнице... Решите, нужны ли вам старые предположения и убеждения... Если да, возьмите шар из света... Если нет, пройдите через третью дверь, вниз по лестнице и остановитесь... Решите, хотите ли ваши тревоги обратно... Помните, что многие тревоги это формы страха и они имеют свою мудрость в себе... Мудро боятся некоторых вещей... Вы решаете, хотите ли часть или все тревоги... Выберите те, которые хотите из светового шара... Оставьте остальные закопанными... Выйдете в какое-нибудь красивое место и посмотрите на небо... Представьте облако в форме цифры 3... Почувствуйте ступни и руки; почувствуйте жизнь, возвращающуюся в ваше тело... Почувствуйте каждую клетку и мускулу... Представьте, как предыдущее облако сдуло и сформировалось новое облако, в форме цифры два... Почувствуйте себя, возвращающегося в бодрствование... Представьте, как сдувает предыдущее облако и формируется новое, в форме цифры один... Когда вы увидите цифру один, откройте глаза.

Всегда сидите несколько минут после медитации. Позвольте себе интегрировать переживание. Части переживания могут вернуться к вам позднее во время дня. Медитация это внутреннее переживание. Внутренняя жизнь оперирует на своём языке, в виде образов, символов и чувств.

Как стыдливый человек я редко бывал в своём внутреннем замке. Я был занят охраной и защитой себя, что бы обо мне не узнали. Я был так занят охраной внешнего, что никогда не был внутри. Я жил во дворе.

Медитация требует времени и практики. Моя грандиозность и импульсивность хотела всего и сразу. Я хотел, что бы ворота открылись и явился Бог... Мой Магический ребёнок часто укалывал мою грандиозность. Я использую моего Магического ребёнка как внутреннего проводника. Однажды я спросил у него, что мне нужно было сделать, что бы разрешить духовную дилемму. Его ответом было: <<Начни с расчистки своего стола!>> Я был ужасно расстроен этим ответом. Я хотел, что бы он отправил меня в монастырь или поголодать с неделю. Расчистить стол? Да ладно! Путь духа очень прост. Он простой, но трудный. Путь интеллекта ЭГО сложен. Анализировать и интеллектуализировать сложно, но легко.

Ответ о предназначении вашей жизни обычно очень простой. Для меня предназначение это быть тем человеком, каким я должен быть. Это значит, что я должен быть собой. Это единственная вещь, ради которой я не должен работать. Быть собой, значит любить себя таким образом, каким я описал. Это включает также любить других, т.к. я не могу любить себя, без желания роста и расширения, а это возможно только через любовь.

\ж{Взгляд на себя глазами Высшей Силы}

Эта медитация короткая. её можно выполнить за 10-15 минут. Она позволяет вам посмотреть на себя с точки зрения вашего высшего сознания ли Высшей Силы.

Используйте ваше собственное переживание Бога, как вы понимаете Бога. Запишите следующее на аудио:

Закройте свои глаза и сфокусируйтесь на дыхании... (10 секунд)... Осознавайте своё дыхание на вдохе... и выдохе. Сфокусируйтесь на разнице ощущения воздуха на вдохе и выдохе... Холоден ли он на вдохе?... Теплый ли он на выдохе?... Почувствуйте разницу настолько полно как вы только можете... (30 секунд)...

Теперь сделайте несколько глубоких вдохов... при вдохе и выдохе начинайте представлять цифру 5... (20 секунд)... Затем цифру 4... (20 секунд)... Затем представьте цифру 3... (20 секунд)... Цифру 2... (20 секунд)... Цифру 1. Представьте как цифра 1 превращается в дверь и откройте её... (10 секунд)... Вы увидите длинный извилистый коридор, ведущий к световому полю... Пройдите по коридору, замечая двери по обе его стороны... У каждой двери есть символ... Пройдите к полю света... (10 секунд)... Пройдите через световое поле в древнюю церковь или храм... (10 секунд)... Осмотрите это святое место... (20 секунд)... Присядьте в комфортном месте и позвольте символическому образу вашего Бога или Высшей Силе войти в церковь или храм... Позвольте образу подойти к вам и присесть напротив вас... Помните, что это присутствие истины, красоты, добра и любви... Представьте, что вы могли бы выплыть из своего тела в его присутствии... Когда вы увидите себя, сидящего напротив себя, создайте якорь. Держите якорь, делая следующее:

Представьте, что вы есть ваша Высшая Сила. Вы Создатель жизни, любви и всех людей на Земле. Вы смотрите на себя. Вы смотрите на себя глазами любви. Вы в сердце и разуме самой любви. Вы видите себя полностью и совершенно. Вы начинаете видеть и распознавать качества и аспекты себя, которые никогда раньше не видели... (20 секунд)... Вы видите и слышите, что ваша Высшая Сила ценит в вас... (20 секунд)... Вы чувствуете себя полностью и безусловно принятым... (30 секунд)... Держа в уме то, что ваша Высшая Сила любит и ценит в вас, особенно те аспекты, о которых вы были не в курсе, медленно возвращайтесь в своё тело. Будь полностью собой. Отпустите якорь. Чувствуйте всю любовь и ценность того, что вы есть. Поблагодарите вашу Высшую Силу и уходите из места, где вы были. Когда вы подойдете к выходу, представьте красивое природное место. Выйдете на него. Почувствуйте себя частью Вселенной. Почувствуйте себя необходимой частью природы. Вы должны быть здесь... (30 секунд)... Посмотрите на небо и представьте облако в форме цифры 1. Скажите себе, что вы будете помнить и ценить это чувство. Представьте как облако превращается в цифру 2 - затем три - почувствуйте свои руки и ступни. Осознайте своё тело. Представьте как облако превращается в цифру 4, знайте, что вы приходите в сознание. Представьте цифру 5 и медленно откройте глаза.

\ж{Состояние не ума - Создание тишины}

Медитация направлена на усиления <<бытия>>. Когда вы находитесь в контакте со своим бытием, вы в единстве со всем, что есть. Больше не существует разделенности. Без разделенности нет объекта или события вне вашего постижения.

Один из мастеров медитации, Suzuki Rashi, говорил: <<До тех пор, пока вы практикуете медитацию ради чего-то, это не истинная практика>>. В медитации мы просто позволяем себе быть. Чем меньше мы думаем и делаем, тем больше мы просто есть. Так как Бог, как я понимаю Бога, само бытие, то разрешить себе быть, значит войти в союз с Богом.

В этой медитации вы можете начать испытывать чистые моменты простого бытия. Эти моменты чувствуются открытыми и просторными, потому что они лишены личных потребностей, значений и интерпретаций. Это большое пространство один из способов описать тишину. Медитация учит нас контакту с этим пространством (тишиной). Так как это пространство лежит вне постоянного поиска личного смысла, оно может повлиять на радикальную трансформацию того, как мы живем. На стыдливом жаргоне это значит, что вы перестанет быть сверхнастороженным. Медитация может привести вас к этому сильному ощущению живости. Подобное чувство живости не имеет связи с тем, что мы делаем; это то, кто мы есть. Как сказала Jacquelyn Small: <<Нет ничего, что должно быть сделано; есть только тот, кто просто есть>>.

Создание состояния тишины или состояния не ума требует дисциплины. Лучше всего это можно сравнить с каплями, падающими на камень. За многие годы камень стачивается.Следующее, это один из способов достижения состояния не ума.

Начните эту медитацию, использую инструкцию для Магического ребёнка. Создание состояния не ума требует следования осознанию, куда бы оно не было направлено. Начните с вашего дыхания и отпускания ваших мыслей. Не пробуйте контролировать или направлять их в более приятном направлении. Каждый раз, когда вы осознаете ваши мысли, мягко верните осознание обратно на дыхание.

Начните с осознания ощущения воздуха, проходящего через ваши ноздри. Отметьте, в какой части ноздрей вы чувствуете прикосновение воздуха при вдохе и при выдохе... Осознайте теплоту или холодность воздуха... Продышите ваш лоб и осознайте любое ощущение на лбу...

Продолжайте обращать внимание на любое ощущение вокруг глаз, рта, в шее и плечах... Просто осознавайте ощущение...

Продолжайте через все ваше тело. Не пропустите ни одной части... Вы можете заметить какую-то часть, полностью лишенную ощущений... Продолжайте фокусироваться на этих частях. Если ощущение не появится, продолжайте...

Когда вы доберетесь до пальцев ног, начните заново... Делайте это примерно 10 минут... Затем, осознайте все ваше тело как целое. Почувствуйте все ваше тело как одну массу различных типов ощущений... Теперь вернитесь к частям - фокусируйтесь на глазах, рте, шее и т.д. Затем снова осознайте все ваше тело как целое.

Отметьте глубокое спокойствие... Отметьте спокойствие вашего тела... Вернитесь к частям и снова к спокойствию...

Постарайтесь не двигать ни одной частью тела... Каждый раз, когда вы почувствуете позыв сделать это, не поддавайтесь, просто осознавайте этот позыв. Это может быть болезненно по-началу. Вы можете напрячься. Просто осознавайте напряжение... Оставайтесь с ним и напряжение рассеется... (1 минуту).

Теперь представьте, что вы входите в святое место... Подойдите к круговому алтарю с земляным полом... В земле зарыта книга и вы знаете, где именно... Выкопайте её и начинайте просматривать её, пока какая-либо страница не привлечёт ваше внимание... Вы можете увидеть символ... или прочитать что-то привлекающее вас... Примите то, что придёт к вам и верните книгу на место... Выйдите из святого места и посмотрите на себя, выходящего оттуда... Посмотрите на символ, чувство или сообщение, которое вы получили из книги, как будто оно находиться на экране... Если вы пережили какое-то чувство, представьте его в форме образа. Представьте себя взаимодействующего с этим символом, чувством или сообщением... (1 минуту)... Затем позвольте всему горизонту потемнеть, стать ещё темнее пока тьма не заполнит все вокруг...

Пристально вглядитесь в темноту... (1 минуту)... Представьте огонь свечи в центре темноты... Представьте, что свет от свечи начал усиливаться пока не заполнил все пространство светом... Пристально глядите в поле чистого белого света; позвольте себе пропитаться этим светом... Позвольте себе раствориться в свете... Осознавайте ничто. Нет ничего, кроме великой бездны и пустоты... Растворитесь в пустоте... (3 минуты)...

Медленно начинайте представлять на горизонте цифру 1; затем цифру 2, 3, 4, 5, 6. Когда вы увидите цифру семь, откройте глаза. Посидите в задумчивости несколько минут.

Быть в состоянии не ума, значит быть свободным от любого мыслительного содержания. В тишине, вы останавливаете свой внутренний голос. Вы отключаете умственную болтовню. Разум пуст и сфокусирован на пустоте. Такое состояние это состояние чистого бытия. Бытие это основа всех существ, что есть. Существуют человеческие существа, животные, деревья. Каждый это форма бытия. Когда разум достигает состояния пустоты, он выходит за рамки всех вещей к основанию всех вещей. Вы прибываете к чистому бытию, т.е. подойти к пустоте, значит подойти к основе всего. Вы объединяетесь с самим бытием. В таком состоянии, вы связаны со всем.

\ж{Единство сознания - блаженство}

В этот момент вы прибываете к единству сознания. В этом состоянии не существует разделения. Все дихотомии синтезированы. Противоположности совпали. Завеса внешнего разрушилась. Вы видите взаимосвязь всего сознания. Это состояние также называется блаженством (bliss). Это состояние чистой радости и мира. В этом состоянии все трансформировано. Все ваши эгоистичные понимания растворяются. Вы не становитесь лучшей версией себя; вы становитесь совершенно другим. Вы видите всю вашу жизнь как совершенную; как что-то, что должно быть. Вы видите всю картину, не только её части.

\ж{Целостность ЭГО}

Вы прибываете к тому, что Erik Erikson называл целостностью ЭГО. Целостность ЭГО значит, что вы полностью и безусловно приняли себя. Erikson пишет: <<Это принятие себя и жизненного цикла, как то, что должно быть... обладатель целостности готов защитить достоинство своего образа жизни от всех угроз>>. (Childhood and Society).

С целостным ЭГО вы действительно можете сказать: <<Если бы я должен был прожить жизнь снова, я прожил бы её точно также!>>. Целостность ЭГО это тотальное самопринятие, полное преодоление токсического стыда. После достижения этого состояния, оно не имеет противоположного. Вы снова жаждете его. 

\ж{Синтез противоположностей}


В состоянии блаженства вы перестаете видеть противоположности. Не существует <<нас>> и <<их>>. Вы переживаете единство. Границы ЭГО, над созданием которых вы так тяжело работали, стали структурой, которая позволила вам превзойти все границы. Наивный мистицизм вашего Магического ребёнка расширился до рефлексивного мистицизма взрослого. Вы чувствуете единство со всем творением. Вы осознаете, что разделённость это иллюзия.


\ж{Высшая Сила}

Мистики и физики говорят, что в состоянии блаженства, нам доступна Высшая Сила. Подключение к всему сознанию даёт нам ресурсы понимания и знания, которые куда более мощнее, чем мы когда-либо себе представляли. Единственным условием, необходимым для таких знаний является отпуск ЭГО контроля. Слоган 12-ступенчатой программы говорит: <<Отпусти и впусти Бога>>. Слоган призывает вас отпустить контроль ЭГО.


\ж{Разум над материей}

Физики также говорят, что и разум и материя это формы энергии. Материя менее сияющая форма энергии; она вибрирует на меньших частотах. Разум или сознание это более высокочастотная энергия; оно может оказывать мощное воздействие на материю. Это может включать в себе такие феномены как психокинез.

Delores Kieger обучила тысячи людей, как облегчать боль через наложение рук. Этот феномен достигается через концентрацию и развитие кинестетической сенсорной остроты. Любой может научиться этому. Разум может создавать события все себя через использование воображения. Это включает в себя создание богатства, а также силу разума в лечении тела.


\ж{Молитва}

Молитва может иметь мощное влияние на события в естественном порядке. Молитва зависит от высшего уровня духовности. Молитва зависит от Бога как мы понимаем Бога. В молитве мы отпускаем и впускаем Бога. Мы отдаём контроль и позволяем появится детскому доверию и вере.


\ж{Интенциональность сознания}

Все сознание является намеренным. Это значит оно целенаправленно. Я предположил, что даже самое странное поведение людей имеет смысл. Когда мы находимся в высшем сознании, мы едины со всем сознанием. Мы разделяем целенаправленность всего сознания. Многие физики и трансперсональные психологи верят, что существует <<план>> для наших жизней. Когда мы смотрим назад на нашу жизнь, часто становится ясно, что такой план существует. Я знал о некоторых вещах в 12 лет, которые на самом деле произошли в моей жизни. За многие годы мне стало ясно, что если я уйду с пути и перестану пытаться все контролировать, все сложиться как надо. Наше ЭГО, с его ограниченным виденьям, не видит леса за деревьями. <<Отпустить и впустить Бога>> значит повернуться к высшему сознанию.

\ж{Непривязанность}

Духовные мастера и святые всегда практиковали отпускание. Они называли это непривязанностью. Они говорили нам, что наши страдания растут из нашей привязанности. Наша эмоциональная инвестиция во что-то вызывает боль. Так как все человеческие события имеют конец, эмоциональная боль неизбежна, до тех пора, пока вы эмоционально привязаны.

В фильме Zorba The Greek, Zorba выставляет свою непривязанность, когда встречается с бедствием. Он полностью посвятил себя проекту строительства древесного канала вниз по склону горы. Никто не могу работать более усердно и с большей приверженностью. После многих месяцев тяжёлой работы, проект был готов для тестирования. Когда древесина поплыла вниз по склону, она набрала большую скорость и разрушила канал. Zorba был ошеломлен. Затем он начала смеяться. Он смеялся и смеялся. Затем он начала танцевать. Его смех и танец были видом космического смеха и танца. С перспективы единого сознания, никакое явление не имеет важности. Важна целостность.

Целостность это то, о чем эта мудрость. Мы можем понять детали, когда видим целое. Как говорил Sri Aurobindo: <<Вы должны познать высшее, что бы понять низшее>>. Единое виденье и непривязанность это плоды состояния блаженства.

Существуют и другие плоды, вытекающие из духовного блаженства. Некоторые из них это безмятежность, уединённость и служение. Каждый выражается по-разному, согласно уникальности образа жизни человека.

\ж{Безмятежность}

Безмятежность характеризуется тем, что Robert Frost называл <<лёгкой ездой в упряжке>>. С безмятежностью, ваши жизнь станет менее проблематичной и более спонтанной. Вы будете действовать без анализирования всего и вся и без размышлений. Вы перестанете пытаться все выяснить. Перестанете чрезмерно реагировать; сверхнастороженость покинет вас. Вы будете наслаждаться каждым моментом. Вы перестанете верить в дефицит и забудете об импульсивности и мгновенном удовлетворении. Вы будете принимать богатство жизни момент за моментом. Вы будете видеть, то что видите, слышать, то что слышите, знать, то что хотите и желаете, и знать, что вы можете удовлетворить ваши потребности и желания. Безмятежность изменяет жизнь в детское виденье, в котором <<луг, ручей и каждое обычное явление>> приобретает новизну. Те, кто безмятежны, любят землю и все вещи. Жизнь сама по себе великолепна.

\ж{Уединение}

Каждый из нас один. Это жёсткое ограничение нашего материального состояния. Одиночество это жизненный факт. От того, как мы принимаем наше одиночество будет зависеть то, будет оно токсичным или питающим. Токсичное одиночество питает токсический стыд. Это следствие само-разрыва. Питающее одиночество это плод блаженной духовности. Оно вытекает из единства с Богом, дающее нам непосредственное знание о нашей личности. Из такого знания вытекает самолюбие, самопринятие и самоценность.

Потому что вы любите и цените себя, вы хотите проводить время с собой в одиночестве. Это называется уединением. Когда вы понимаете радость уединения, вы хотите этого больше; вы также хотите этого для тех, кого любите. Вместо вашего старого стыдливого собственничества, вы станете защитником вашего собственного единения и ваших возлюбленных.

Это отлично выразил Рильке: <<После принятия понимания, что даже между самыми близкими людьми продолжает существовать бесконечная пропасть, может вырасти удивительная жизнь бок о бок, если они преуспеют в любви с дистанцией между ними, которая позволяет им увидеть целостность другого на фоне небес. Хороший брак это тот, в котором каждый назначает другого опекуном своего уединения>>.

Уединение возможно, если вы провели работу с ЭГО, особенно работу с источником боли. С завершением этой работы вы признаете свою разделённость. Страх разделённость является причиной, по которой вы остались в воображаемой связи в первую очередь. Воображаемая связь это иллюзия; иллюзия того, что вы всегда будете защищены вашими родителями.

Когда вы примете вашу разделённость и одиночество, вы придёте к убеждению, что ваше ЭГО достаточно сильное, что бы позаботиться о вас. Ваше ЭГО достаточно сильное, что бы вы могли выжить в одиночку. Это также является условием для медитации. Когда вы переживёте блаженное единение с Богом через медитацию, вы узнаете себя настоящего. Вы также узнаете о месте, в котором вы никогда не одни. С таким осознанием, уединение становится наиболее желанным состоянием.

\ж{Служение}

Духовное блаженство синтезирует жизненные полярности. Чем больше вы будете познавать и радоваться уединению, тем больше вы будете хотеть служить другим так, что бы они усиливали свою духовность. Как говорил Ken Wilber: <<Чтобы понимать и интуитивно знать свою истинную сущность значит быть приверженным актуализировать свою сущность во всех существах, в соответствии с исконным обетом. Вместе с бесчисленными существами, я клянусь освободить их.>> No Boundaries.

Служение также может значить приверженность поклонению таким путём, какой подходит вашим убеждениям. Вы можете захотеть вернуться в свою старую церковь и вероисповеданию. Если вернетесь, у вас будет новый взгляд и расширенное осознание. В этом случае мы сможете увидеть поклонение и ритуалы как способы воплощения объединяющих воспоминаний вашей религиозной традиции. Участие в символической инсценировки коллективных воспоминаний о Боге ваших предков может принести вам переживание себя как части прошлой и настоящей живой традиции. Вы можете пережить ваши действия в настоящем как привнесение прошлого в настоящее и настоящего в будущее.

Служение значит забота о других и отдача назад того, что вы получили. 12-ступенчатая программа призывает участников нести духовное пробуждение другим страдающим от токсического стыда. Всем нам, кто вышел из укрытия, требуется нести этот свет другим. Принести сообщение значит моделировать, а не заниматься нравоучением. Это делается теми, кто <<не спешит языком, а торопится делом>>. Это значит, что не существует гуру. Существуют те, кто прошёл чуть дальше в своём пути. Как только вы делаете кого-то в гуру, вы отказываетесь от своей собственной власти. Служение и любовь для других вытекает напрямую из служения и любви к себе. Мы действительно не может дать того, чего у нас нет. Мы не можем научить детей ценить себя, если мы стыдливы. Мы не можем привести наших клиентов туда, где мы сами не были как терапевт.

Служение это истинный знак духовного блаженства. В блаженстве мы понимаем то, о чем говорил Paul Claudel:

<<Нет никого из моих братьев... Без кого я могу обойтись... В сердце подлейшего скупца, самой убогой проститутки, самого несчастного пьяницы есть бессмертная душа со святым стремлением, которая лишена дневного света. Я слышу из разговор, когда сам говорю и плач, когда сам падаю на колени. Нет никого из них, без кого я могу обойтись. Также как существует множество звёзд на небесах и их счёт не подлежит моему расчёту, также есть множество живущих существ... Они нужны мне все в моей похвале Богу. Множество живых существ, но ни с одним из них я не теряю связи на священной вершине, где мы вместе произносим Отче наш>>.

\ж{Empowerment}

Блаженство рождает расширение прав и возможностей. Мы двигаемся от нашего детского убеждения, что мы всегда будем жертвой детской спонтанности и оптимизма. Мы принимаем наше воображение и творческие способности. Мы отказываемся быть жертвой. Мы становимся артистичным творцом собственной жизни. Мы рискуем. Мы идём к тому, чего действительно хотим.

Когда вы дошли до конца книги, я искренне надеюсь в том, что вы прошли долгий путь в излечении связывающего вас стыда. Сделав это, вы открыли себя удивительным, но ограниченным возможностям вашей человеческой натуры. Такие возможности моделируются великолепным излиянием человеческого творчества. Великие музыканты были ограничены законами музыкальной гаммы, но в этих границах разнообразие их композиций просто невероятно.

Великие художники были ограничены их полотном, но ходить по музею изобразительного искусства может быть удивительным и переполняющим переживанием. В наших человеческих ограничениях все равно существуют чудеса. Вы одно из этих чудес!


\ж{Эпилог}

Страх раскрытия лежит в сердце стыда. Когда мы позволяем нашему стыду быть раскрытым другим, мы раскрываемся перед собой. Как однажды сказал автор книги Shame:

<<Раскрытие перед собой лежит в сердце стыда: мы обнаруживаем, в переживании стыда самые чувствительные, интимные и уязвимые части себя>>.

В этом смысле, излечение связывающего вас стыд, это процесс откровения. Из-за того, что стыд существует в самом ядре вашего существа, когда мы принимаем стыд, то начинаем обнаруживать, кто мы есть на самом деле. Стыд одновременно скрывает и показывает нашу истинную сущность.

Лечение токсического стыда также является революционным. Когда вы действительно чувствуете ваш токсический стыд, это побуждает вас изменить его. Это может произойти только с готовностью выйти из укрытия. Мы должны двинуться от наших страданий и принять нашу боль. Мы должны почувствовать то, как плохо на самом деле мы себя чувствуем. Такое чувство заставляет нас измениться. Это революционно.

Токсический стыд, с его более чем человек/менее чем человек грандиозностью, это проблема, включающая в себя отрицание ограниченности человека. Быть человеком требует смелости. Это требует смелости, потому что быть человеком значит быть несовершенным. Alfred Adler первым использовал фразу: <<Смелость быть несовершенным>>. Как я попытался показать, наши семьи, религии, школы и культура основана на перфекционисткой системе. И перфекционизм это главная причина токсического стыда. Перфекционизм настраивает нас на то, что нас можно измерить, что также настраивает нас на вечные разочарования. Перфекционизм безграничен. Нет никаких ограничений. Вас никогда не может быть достаточно. Требуется смелость, что бы бороться с этими перфекционисткими системами. Но это того стоит!

Смелость быть несовершенным порождает образ жизни, характеризующийся спонтанностью и юмором. Когда вы принимаете, что ошибки естественный продукт ограниченного человеческого осознания, вы перестает сидеть на иголках. Вы чаще рискуете и чувствуете себя свободнее, что бы исследовать и быть творческим.

Самое важное, вы будете чаще смеяться. Чувство юмора может быть главным критерием и измерителем излечения человеком интернализированного стыда. Уметь смеяться над событиями, другими людьми и над собой требует истинной человечности. Что бы иметь чувство юмора вы должны оседлать полярность более/менее чем человека. Это требует от вас быть жонглером парадоксов.

Чувство юмора основано на сопоставлении не сочетаемого. Иметь чувство юмора значит воспринимать жизнь менее угрюмо и более серьёзно. Как писал Walter O'Connell: <<Юмор это результат повторного решения человеческих парадоксов>>. Каждый парадокс человека имеет две крайности. Через их воссоединение мы приобретаем энергию и надежду. Это и есть плоды юмора. Это также даёт нам перспективу и баланс. Это позволяет нам смеяться как над нашим разросшимся ЭГО, так и нашими недостатками. Дайте себе разрешение наслаждаться каждой минутой каждого дня вашей жизни. Идите к просветлению. Вы поймёте, что прибыли, когда осветитесь!

Заметка для психотерапевтов
Существует растущее осознание в современной психологии относительно отсутствия исследований по стыду. Joseph Campos пишет: <<Мы слишком мало знаем о стыде. Он был игнорируемой эмоцией в психологии>>. Это пренебрежение серьёзно повлияло на успех психотерапии в лечении стыдливых людей. Dr. Donald Nathanson утверждает: <<Стыд замалчивается в психотерапии, но его важность была распознаны многими терапевтами>>.

Helen Block Lewis, психолог из Йельского университета, была одной из первых, кто изучала проблему стыда в психотерапии. В её исследовании 180 психотерапевтических сессий, она выяснила, что: <<Когда терапевт не распознавал у пациента чувство стыда, проблемы пациента продлевались или ухудшались. Когда терапевт распознавал стыд и помогал пациенту справится с ним, лечение было короче>>.

Работа Dr. Lewis подтверждает мой опыт как пациента и терапевта. В определённый момент моей жизни, когда я находился в глубоком смятении и отчаянии, я провел три месяца терапии, работая с моим психиатром, покорно принимая транквилизаторы и снотворное, которое он прописал. Шесть месяцев после прекращения этого лечения, я был помещён в госпиталь Остина.

У меня были серьезные сомнения в эффективности традиционной модели лечения, относительно различных синдромов стыда. Возможно, более точное понимание токсического стыда может открыть новые инновационные подходы.

Следующая диаграмма это вклад в формирующийся диалог. Он сравнивает и противопоставляет токсический стыд и токсическую вину.


\end{document}
